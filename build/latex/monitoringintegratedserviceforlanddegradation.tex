%% Generated by Sphinx.
\def\sphinxdocclass{report}
\documentclass[letterpaper,10pt,english]{sphinxmanual}
\ifdefined\pdfpxdimen
   \let\sphinxpxdimen\pdfpxdimen\else\newdimen\sphinxpxdimen
\fi \sphinxpxdimen=.75bp\relax
\ifdefined\pdfimageresolution
    \pdfimageresolution= \numexpr \dimexpr1in\relax/\sphinxpxdimen\relax
\fi
%% let collapsible pdf bookmarks panel have high depth per default
\PassOptionsToPackage{bookmarksdepth=5}{hyperref}

\PassOptionsToPackage{warn}{textcomp}
\usepackage[utf8]{inputenc}
\ifdefined\DeclareUnicodeCharacter
% support both utf8 and utf8x syntaxes
  \ifdefined\DeclareUnicodeCharacterAsOptional
    \def\sphinxDUC#1{\DeclareUnicodeCharacter{"#1}}
  \else
    \let\sphinxDUC\DeclareUnicodeCharacter
  \fi
  \sphinxDUC{00A0}{\nobreakspace}
  \sphinxDUC{2500}{\sphinxunichar{2500}}
  \sphinxDUC{2502}{\sphinxunichar{2502}}
  \sphinxDUC{2514}{\sphinxunichar{2514}}
  \sphinxDUC{251C}{\sphinxunichar{251C}}
  \sphinxDUC{2572}{\textbackslash}
\fi
\usepackage{cmap}
\usepackage[T1]{fontenc}
\usepackage{amsmath,amssymb,amstext}
\usepackage{babel}



\usepackage{tgtermes}
\usepackage{tgheros}
\renewcommand{\ttdefault}{txtt}



\usepackage[Bjarne]{fncychap}
\usepackage{sphinx}

\fvset{fontsize=auto}
\usepackage{geometry}


% Include hyperref last.
\usepackage{hyperref}
% Fix anchor placement for figures with captions.
\usepackage{hypcap}% it must be loaded after hyperref.
% Set up styles of URL: it should be placed after hyperref.
\urlstyle{same}

\addto\captionsenglish{\renewcommand{\contentsname}{About}}

\usepackage{sphinxmessages}
\setcounter{tocdepth}{2}



\title{Monitoring Integrated Service for Land Degradation}
\date{Feb 06, 2023}
\release{1.0.0}
\author{LocateIT Kenya Ltd}
\newcommand{\sphinxlogo}{\vbox{}}
\renewcommand{\releasename}{Release}
\makeindex
\begin{document}

\pagestyle{empty}
\sphinxmaketitle
\pagestyle{plain}
\sphinxtableofcontents
\pagestyle{normal}
\phantomsection\label{\detokenize{index::doc}}




\sphinxAtStartPar
The Monitoring Integrated System for Land Degradation MISLAND was developed under
The Monitoring Integrated System for Land Degradation MISLAND was developed under
GMES \& Africa programme through a collaboration between the OSS and LocateIT  as a Decision Support System (DSS) utilizing Earth Observation data to deliver information, promote awareness, and support decision making toward achieving Land Degradation Neutrality (LDN) in African countries.

\sphinxAtStartPar
At the very core, the service provides information to monitor SDG indicator 15.3.1
(Proportion of land that is degraded over the total land area). In addition, and to
improve the understanding and the multi\sphinxhyphen{}faceted nature of the active processes
behind land degradation, MISLAND service also provides information on vegetation
loss and gain hotspots, forest change, forest fires and the Mediterranean Desertification
and Land Use Model (MEDALUS), to assess desertification indicators.

\begin{sphinxadmonition}{note}{Note:}
\sphinxAtStartPar
You can download the \sphinxhref{https://locateit-oss-ldms.readthedocs.io/\_/downloads/en/latest/pdf/}{PDF Version of this doucument} here.
\end{sphinxadmonition}

\sphinxstepscope


\chapter{General Infortmation}
\label{\detokenize{Introduction/general_information:general-infortmation}}\label{\detokenize{Introduction/general_information::doc}}
\sphinxAtStartPar
MISLAND\sphinxhyphen{}North Africa is an operational instrument relying on the international standards for reporting SDG 15.3.1 and technical approaches allowing the delivery of regular information on vegetation cover gain/loss to decision makers and environmental agencies at the first place.

\sphinxAtStartPar
The core\sphinxhyphen{}service provides land degradation indicators for six North African Countries at two levels:
\begin{itemize}
\item {} 
\sphinxAtStartPar
At the continental level(Africa) and five regional levels(North Africa, West Africa, East Africa, Central Africa, and South Africa) where low and medium resolution EO are used.

\item {} 
\sphinxAtStartPar
At the pilot site level, where(customized indicators) can be developed, using medium resoultion data(landsat time series imagery and derived vegetation indices, combined with different satellite\sphinxhyphen{}derived climate data)

\end{itemize}


\section{Get in touch with the team}
\label{\detokenize{Introduction/general_information:get-in-touch-with-the-team}}
\sphinxAtStartPar
Contact the MISLAND\sphinxhyphen{}Africa team with any comments or suggestions. If you have specific bugs to report or improvements to the tool that you would like to suggest, you can also submit them in the issue tracker on Github for MISLAND\sphinxhyphen{}Africa.


\section{Authors}
\label{\detokenize{Introduction/general_information:authors}}
\sphinxAtStartPar
The MISLAND\sphinxhyphen{}Africa is a project by the OSS under the Global Monitoring for Environment and Security and Africa(GMES \& Africa) framework that is co\sphinxhyphen{}funded by the African Union and the European Union.

\sphinxAtStartPar
Contributors to the documentation and to MISLAND\sphinxhyphen{}Africa include Mustapha MIMOUNI, Nabil KHATRA, Amjed TAIEB, Haithem REJEB, Toure SOULEYMANE, Anthony ODONGO, Vivianne META, Derick ONGERI, Grace AMONDI, and Brian CHELOTI.
\begin{quote}

\noindent{\hspace*{\fill}\sphinxincludegraphics[width=550\sphinxpxdimen,height=94\sphinxpxdimen]{{partners}.png}\hspace*{\fill}}
\end{quote}


\section{Aknowledgement}
\label{\detokenize{Introduction/general_information:aknowledgement}}
\sphinxAtStartPar
Special appreciation to the Trends.Earth. Conservation International. Available online at \sphinxurl{http://trends.earth}
.2018. for providing input on the implementation of the SDG 15.3 and LDN indicators in MISLAND\sphinxhyphen{}North Africa, on the UNCCD reporting process, and also provided early input and testing of the tool.

\sphinxAtStartPar
The project also acknowledges the contribution of national and regional stakeholders; Algerian Space Agency(ASAL), ASAL (Algeria), DRC (Egypt), LCRSSS (Libya), CRTS (Morocco), AL\sphinxhyphen{}Aasriya University of Nouakchott (Mauritania) and CNCT (Tunisia) for the national level and CRTEAN and CRASTE\sphinxhyphen{}LF for the regional level


\section{Linces}
\label{\detokenize{Introduction/general_information:linces}}
\sphinxAtStartPar
MISLAND\sphinxhyphen{}Africa is free and open\sphinxhyphen{}source. It is licensed under the GNU General Public License, version 2.0 or later.

\sphinxAtStartPar
This site and the products of MISLAND\sphinxhyphen{}Africa are made available under the terms of the Creative Commons Attribution 4.0 International License (CC BY 4.0). The boundaries and names used, and the designations used, in MISLAND\sphinxhyphen{}North Africa do not imply official endorsement or acceptance by OSS, or its partner organizations and contributors.

\sphinxstepscope


\chapter{Data sources}
\label{\detokenize{Introduction/data:data-sources}}\label{\detokenize{Introduction/data::doc}}
\sphinxAtStartPar
MISLAND\sphinxhyphen{}North Africa draws on a number of data sources. The data sets listed below are
owned/made available by the following organizations and individuals under
separate terms as indicated in their respective metadata.


\section{NDVI}
\label{\detokenize{Introduction/data:ndvi}}

\begin{savenotes}\sphinxattablestart
\centering
\begin{tabulary}{\linewidth}[t]{|T|T|T|T|T|}
\hline
\sphinxstyletheadfamily 
\sphinxAtStartPar
Sensor/Dataset
&\sphinxstyletheadfamily 
\sphinxAtStartPar
Temporal
&\sphinxstyletheadfamily 
\sphinxAtStartPar
Spatial
&\sphinxstyletheadfamily 
\sphinxAtStartPar
Extent
&\sphinxstyletheadfamily 
\sphinxAtStartPar
License
\\
\hline
\sphinxAtStartPar
\sphinxhref{https://developers.google.com/earth-engine/datasets/catalog/landsat}{LANDSAT7}
&
\sphinxAtStartPar
2001\sphinxhyphen{}2020
&
\sphinxAtStartPar
30 m
&
\sphinxAtStartPar
Global
&
\sphinxAtStartPar
\sphinxhref{https://creativecommons.org/publicdomain/zero/1.0}{Public Domain}
\\
\hline
\sphinxAtStartPar
\sphinxhref{https://glam1.gsfc.nasa.gov}{AVHRR/GIMMS}
&
\sphinxAtStartPar
1982\sphinxhyphen{}2015
&
\sphinxAtStartPar
8 km
&
\sphinxAtStartPar
Global
&
\sphinxAtStartPar
\sphinxhref{https://creativecommons.org/publicdomain/zero/1.0}{Public Domain}
\\
\hline
\sphinxAtStartPar
\sphinxhref{https://lpdaac.usgs.gov/dataset\_discovery/modis/modis\_products\_table/mod13q1\_v006}{MOD13Q1\sphinxhyphen{}coll6}
&
\sphinxAtStartPar
2001\sphinxhyphen{}2016
&
\sphinxAtStartPar
250 m
&
\sphinxAtStartPar
Global
&
\sphinxAtStartPar
\sphinxhref{https://creativecommons.org/publicdomain/zero/1.0}{Public Domain}
\\
\hline
\end{tabulary}
\par
\sphinxattableend\end{savenotes}


\section{Soil moisture}
\label{\detokenize{Introduction/data:soil-moisture}}

\begin{savenotes}\sphinxattablestart
\centering
\begin{tabulary}{\linewidth}[t]{|T|T|T|T|T|}
\hline
\sphinxstyletheadfamily 
\sphinxAtStartPar
Sensor/Dataset
&\sphinxstyletheadfamily 
\sphinxAtStartPar
Temporal
&\sphinxstyletheadfamily 
\sphinxAtStartPar
Spatial
&\sphinxstyletheadfamily 
\sphinxAtStartPar
Extent
&\sphinxstyletheadfamily 
\sphinxAtStartPar
License
\\
\hline
\sphinxAtStartPar
\sphinxhref{https://gmao.gsfc.nasa.gov/reanalysis/MERRA-Land}{MERRA 2}
&
\sphinxAtStartPar
1980\sphinxhyphen{}2016
&
\sphinxAtStartPar
0.5° x 0.625°
&
\sphinxAtStartPar
Global
&
\sphinxAtStartPar
\sphinxhref{https://creativecommons.org/publicdomain/zero/1.0}{Public Domain}
\\
\hline
\sphinxAtStartPar
\sphinxhref{https://www.ecmwf.int/en/forecasts/datasets/reanalysis-datasets/era-interim-land}{ERA I}
&
\sphinxAtStartPar
1979\sphinxhyphen{}2016
&
\sphinxAtStartPar
0.75° x 0.75°
&
\sphinxAtStartPar
Global
&
\sphinxAtStartPar
\sphinxhref{https://creativecommons.org/publicdomain/zero/1.0}{Public Domain}
\\
\hline
\end{tabulary}
\par
\sphinxattableend\end{savenotes}


\section{Precipitation}
\label{\detokenize{Introduction/data:precipitation}}

\begin{savenotes}\sphinxattablestart
\centering
\begin{tabulary}{\linewidth}[t]{|T|T|T|T|T|}
\hline
\sphinxstyletheadfamily 
\sphinxAtStartPar
Sensor/Dataset
&\sphinxstyletheadfamily 
\sphinxAtStartPar
Temporal
&\sphinxstyletheadfamily 
\sphinxAtStartPar
Spatial
&\sphinxstyletheadfamily 
\sphinxAtStartPar
Extent
&\sphinxstyletheadfamily 
\sphinxAtStartPar
License
\\
\hline
\sphinxAtStartPar
\sphinxhref{https://www.esrl.noaa.gov/psd/data/gridded/data.gpcp.html}{GPCP v2.3 1 month}
&
\sphinxAtStartPar
1979\sphinxhyphen{}2019
&
\sphinxAtStartPar
2.5° x 2.5°
&
\sphinxAtStartPar
Global
&
\sphinxAtStartPar
\sphinxhref{https://creativecommons.org/publicdomain/zero/1.0}{Public Domain}
\\
\hline
\sphinxAtStartPar
\sphinxhref{https://www.esrl.noaa.gov/psd/data/gridded/data.gpcc.html}{GPCC V6}
&
\sphinxAtStartPar
1891\sphinxhyphen{}2019
&
\sphinxAtStartPar
1° x 1°
&
\sphinxAtStartPar
Global
&
\sphinxAtStartPar
\sphinxhref{https://creativecommons.org/publicdomain/zero/1.0}{Public Domain}
\\
\hline
\sphinxAtStartPar
\sphinxhref{http://chg.geog.ucsb.edu/data/chirps}{CHIRPS}
&
\sphinxAtStartPar
1981\sphinxhyphen{}2016
&
\sphinxAtStartPar
5 km
&
\sphinxAtStartPar
50N\sphinxhyphen{}50S
&
\sphinxAtStartPar
\sphinxhref{https://creativecommons.org/publicdomain/zero/1.0}{Public Domain}
\\
\hline
\sphinxAtStartPar
\sphinxhref{http://chrsdata.eng.uci.edu}{PERSIANN\sphinxhyphen{}CDR}
&
\sphinxAtStartPar
1983\sphinxhyphen{}2015
&
\sphinxAtStartPar
25 km
&
\sphinxAtStartPar
60N\sphinxhyphen{}60S
&
\sphinxAtStartPar
\sphinxhref{https://creativecommons.org/publicdomain/zero/1.0}{Public Domain}
\\
\hline
\end{tabulary}
\par
\sphinxattableend\end{savenotes}


\section{Evapotranspiration}
\label{\detokenize{Introduction/data:evapotranspiration}}

\begin{savenotes}\sphinxattablestart
\centering
\begin{tabulary}{\linewidth}[t]{|T|T|T|T|T|}
\hline
\sphinxstyletheadfamily 
\sphinxAtStartPar
Sensor/Dataset
&\sphinxstyletheadfamily 
\sphinxAtStartPar
Temporal
&\sphinxstyletheadfamily 
\sphinxAtStartPar
Spatial
&\sphinxstyletheadfamily 
\sphinxAtStartPar
Extent
&\sphinxstyletheadfamily 
\sphinxAtStartPar
License
\\
\hline
\sphinxAtStartPar
\sphinxhref{https://lpdaac.usgs.gov/dataset\_discovery/modis/modis\_products\_table/mod16a2\_v006}{MOD16A2}
&
\sphinxAtStartPar
2000\sphinxhyphen{}2014
&
\sphinxAtStartPar
1 km
&
\sphinxAtStartPar
Global
&
\sphinxAtStartPar
\sphinxhref{https://creativecommons.org/publicdomain/zero/1.0}{Public Domain}
\\
\hline
\end{tabulary}
\par
\sphinxattableend\end{savenotes}


\section{Land cover}
\label{\detokenize{Introduction/data:land-cover}}

\begin{savenotes}\sphinxattablestart
\centering
\begin{tabulary}{\linewidth}[t]{|T|T|T|T|T|}
\hline
\sphinxstyletheadfamily 
\sphinxAtStartPar
Sensor/Dataset
&\sphinxstyletheadfamily 
\sphinxAtStartPar
Temporal
&\sphinxstyletheadfamily 
\sphinxAtStartPar
Spatial
&\sphinxstyletheadfamily 
\sphinxAtStartPar
Extent
&\sphinxstyletheadfamily 
\sphinxAtStartPar
License
\\
\hline
\sphinxAtStartPar
\sphinxhref{https://www.esa-landcover-cci.org}{ESA CCI Land Cover}
&
\sphinxAtStartPar
1992\sphinxhyphen{}2018
&
\sphinxAtStartPar
300 m
&
\sphinxAtStartPar
Global
&
\sphinxAtStartPar
\sphinxhref{https://creativecommons.org/licenses/by-sa/3.0/igo}{CC by\sphinxhyphen{}SA 3.0}
\\
\hline
\end{tabulary}
\par
\sphinxattableend\end{savenotes}


\section{Soil carbon}
\label{\detokenize{Introduction/data:soil-carbon}}

\begin{savenotes}\sphinxattablestart
\centering
\begin{tabulary}{\linewidth}[t]{|T|T|T|T|T|}
\hline
\sphinxstyletheadfamily 
\sphinxAtStartPar
Sensor/Dataset
&\sphinxstyletheadfamily 
\sphinxAtStartPar
Temporal
&\sphinxstyletheadfamily 
\sphinxAtStartPar
Spatial
&\sphinxstyletheadfamily 
\sphinxAtStartPar
Extent
&\sphinxstyletheadfamily 
\sphinxAtStartPar
License
\\
\hline
\sphinxAtStartPar
\sphinxhref{https://www.soilgrids.org/}{Soil Grids (ISRIC)}
&
\sphinxAtStartPar
Present
&
\sphinxAtStartPar
250 m
&
\sphinxAtStartPar
Global
&
\sphinxAtStartPar
\sphinxhref{https://creativecommons.org/licenses/by-sa/4.0}{CC by\sphinxhyphen{}SA 4.0}
\\
\hline
\end{tabulary}
\par
\sphinxattableend\end{savenotes}


\section{Agroecological Zones}
\label{\detokenize{Introduction/data:agroecological-zones}}

\begin{savenotes}\sphinxattablestart
\centering
\begin{tabulary}{\linewidth}[t]{|T|T|T|T|T|}
\hline
\sphinxstyletheadfamily 
\sphinxAtStartPar
Sensor/Dataset
&\sphinxstyletheadfamily 
\sphinxAtStartPar
Temporal
&\sphinxstyletheadfamily 
\sphinxAtStartPar
Spatial
&\sphinxstyletheadfamily 
\sphinxAtStartPar
Extent
&\sphinxstyletheadfamily 
\sphinxAtStartPar
License
\\
\hline
\sphinxAtStartPar
\sphinxhref{http://www.fao.org/nr/gaez/en}{FAO \sphinxhyphen{} IIASA Global Agroecological Zones (GAEZ)}
&
\sphinxAtStartPar
2000
&
\sphinxAtStartPar
8 km
&
\sphinxAtStartPar
Global
&
\sphinxAtStartPar
\sphinxhref{https://creativecommons.org/publicdomain/zero/1.0}{Public Domain}
\\
\hline
\end{tabulary}
\par
\sphinxattableend\end{savenotes}


\section{Soil Quality}
\label{\detokenize{Introduction/data:soil-quality}}

\begin{savenotes}\sphinxattablestart
\centering
\begin{tabulary}{\linewidth}[t]{|T|T|T|T|T|}
\hline
\sphinxstyletheadfamily 
\sphinxAtStartPar
Sensor/Dataset
&\sphinxstyletheadfamily 
\sphinxAtStartPar
Temporal
&\sphinxstyletheadfamily 
\sphinxAtStartPar
Spatial
&\sphinxstyletheadfamily 
\sphinxAtStartPar
Extent
&\sphinxstyletheadfamily 
\sphinxAtStartPar
License
\\
\hline
\sphinxAtStartPar
\sphinxhref{https://cmr.earthdata.nasa.gov/search/concepts/C1000000240-LPDAAC\_ECS.html}{Soil Texture and Depth}
&
\sphinxAtStartPar
Present
&
\sphinxAtStartPar
250 m
&
\sphinxAtStartPar
Global
&
\sphinxAtStartPar
\sphinxhref{https://creativecommons.org/licenses/by-sa/4.0}{CC by\sphinxhyphen{}SA 4.0}
\\
\hline
\sphinxAtStartPar
\sphinxhref{https://doi.pangaea.de/10.1594/PANGAEA.788537}{Parent Material}
&
\sphinxAtStartPar
Present
&
\sphinxAtStartPar
N/A
&
\sphinxAtStartPar
Global
&
\sphinxAtStartPar
\sphinxhref{https://creativecommons.org/licenses/by-sa/4.0}{CC by\sphinxhyphen{}SA 4.0}
\\
\hline
\sphinxAtStartPar
\sphinxhref{https://developers.google.com/earth-engine/datasets/catalog/OpenLandMap\_SOL\_SOL\_TEXTURE-CLASS\_USDA-TT\_M\_v02}{Slope}
&
\sphinxAtStartPar
Present
&
\sphinxAtStartPar
30 m
&\sphinxstartmulticolumn{2}%
\begin{varwidth}[t]{\sphinxcolwidth{2}{5}}
\sphinxAtStartPar
Global  | \sphinxhref{https://www.jpl.nasa.gov/imagepolicy/}{JPL public}
\par
\vskip-\baselineskip\vbox{\hbox{\strut}}\end{varwidth}%
\sphinxstopmulticolumn
\\
\hline
\end{tabulary}
\par
\sphinxattableend\end{savenotes}


\section{Climate}
\label{\detokenize{Introduction/data:climate}}

\begin{savenotes}\sphinxattablestart
\centering
\begin{tabulary}{\linewidth}[t]{|T|T|T|T|T|}
\hline
\sphinxstyletheadfamily 
\sphinxAtStartPar
Sensor/Dataset
&\sphinxstyletheadfamily 
\sphinxAtStartPar
Temporal
&\sphinxstyletheadfamily 
\sphinxAtStartPar
Spatial
&\sphinxstyletheadfamily 
\sphinxAtStartPar
Extent
&\sphinxstyletheadfamily 
\sphinxAtStartPar
License
\\
\hline
\sphinxAtStartPar
\sphinxhref{https://developers.google.com/earth-engine/datasets/catalog/IDAHO\_EPSCOR\_TERRACLIMATE\#description}{Terra Climate}
&
\sphinxAtStartPar
1985\sphinxhyphen{}2019
&
\sphinxAtStartPar
30 m
&
\sphinxAtStartPar
Global
&
\sphinxAtStartPar
\sphinxhref{https://creativecommons.org/publicdomain/zero/1.0}{Public Domain}
\\
\hline
\end{tabulary}
\par
\sphinxattableend\end{savenotes}


\section{Administrative Boundaries}
\label{\detokenize{Introduction/data:administrative-boundaries}}

\begin{savenotes}\sphinxattablestart
\centering
\begin{tabulary}{\linewidth}[t]{|T|T|T|T|T|}
\hline
\sphinxstyletheadfamily 
\sphinxAtStartPar
Sensor/Dataset
&\sphinxstyletheadfamily 
\sphinxAtStartPar
Temporal
&\sphinxstyletheadfamily 
\sphinxAtStartPar
Spatial
&\sphinxstyletheadfamily 
\sphinxAtStartPar
Extent
&\sphinxstyletheadfamily 
\sphinxAtStartPar
License
\\
\hline
\sphinxAtStartPar
\sphinxhref{http://www.naturalearthdata.com}{Natural Earth Administrative Boundaries}
&
\sphinxAtStartPar
Present
&
\sphinxAtStartPar
10/50m
&
\sphinxAtStartPar
Global
&
\sphinxAtStartPar
\sphinxhref{https://creativecommons.org/publicdomain/zero/1.0}{Public Domain}
\\
\hline
\end{tabulary}
\par
\sphinxattableend\end{savenotes}

\begin{sphinxadmonition}{note}{Note:}
\sphinxAtStartPar
The \sphinxhref{http://www.naturalearthdata.com}{Natural Earth Administrative Boundaries} provided in MISLAND\sphinxhyphen{}North Africa
are in the \sphinxhref{https://creativecommons.org/publicdomain/zero/1.0}{public domain}. The boundaries and names used, and the
designations used, in MISLAND\sphinxhyphen{}North Africa do not imply official endorsement or
acceptance by Conservation International Foundation, or by its partner
organizations and contributors.

\sphinxAtStartPar
If using MISLAND\sphinxhyphen{}North Africa for official purposes, it is recommended that users
choose an official boundary provided by the designated office of their
country.
\end{sphinxadmonition}

\sphinxstepscope


\chapter{Dataset coding}
\label{\detokenize{Introduction/Layers:dataset-coding}}\label{\detokenize{Introduction/Layers::doc}}
\sphinxAtStartPar
The spatial data produced by MISLAND\sphinxhyphen{}North Africa is in GeoTiff format. This is a
widely supported format, so theese datasets can be used within QGIS as well as
within any other GIS software.

\sphinxAtStartPar
If you wish to use MISLAND\sphinxhyphen{}North Africa data outside of the tool itself, you will need to
know how the data is coded. The tables below provide guidance on what the exact
layers are that are produced by each analysis in MISLAND\sphinxhyphen{}North Africa.

\sphinxAtStartPar
To see which of the below layers is contained within a MISLAND\sphinxhyphen{}North Africa output
file, use the \sphinxhref{../documentation/load\_data.html}{load data} tool. When you
choose a file with that toool, it will show you a list of the layers within
that file, as well as the band number for each layer.


\section{SDG 15.3.1 Indicator}
\label{\detokenize{Introduction/Layers:sdg-15-3-1-indicator}}

\begin{savenotes}\sphinxattablestart
\centering
\begin{tabulary}{\linewidth}[t]{|T|T|}
\hline
\sphinxstyletheadfamily 
\sphinxAtStartPar
Value
&\sphinxstyletheadfamily 
\sphinxAtStartPar
Meaning
\\
\hline
\sphinxAtStartPar
\sphinxhyphen{}32768
&
\sphinxAtStartPar
No data
\\
\hline
\sphinxAtStartPar
\sphinxhyphen{}1
&
\sphinxAtStartPar
Degradation
\\
\hline
\sphinxAtStartPar
0
&
\sphinxAtStartPar
No change
\\
\hline
\sphinxAtStartPar
1
&
\sphinxAtStartPar
Improvement
\\
\hline
\end{tabulary}
\par
\sphinxattableend\end{savenotes}


\section{Productivity}
\label{\detokenize{Introduction/Layers:productivity}}

\subsection{Productivity trajectory (trend)}
\label{\detokenize{Introduction/Layers:productivity-trajectory-trend}}

\begin{savenotes}\sphinxattablestart
\centering
\begin{tabulary}{\linewidth}[t]{|T|T|}
\hline
\sphinxstyletheadfamily 
\sphinxAtStartPar
Value
&\sphinxstyletheadfamily 
\sphinxAtStartPar
Meaning
\\
\hline
\sphinxAtStartPar
\sphinxhyphen{}32768
&
\sphinxAtStartPar
No data
\\
\hline
\sphinxAtStartPar
Any other value
&
\sphinxAtStartPar
Linear trend of annually
integrated NDVI, scaled by 10,000
\\
\hline
\end{tabulary}
\par
\sphinxattableend\end{savenotes}


\subsection{Productivity trajectory (significance)}
\label{\detokenize{Introduction/Layers:productivity-trajectory-significance}}

\begin{savenotes}\sphinxattablestart
\centering
\begin{tabulary}{\linewidth}[t]{|T|T|}
\hline
\sphinxstyletheadfamily 
\sphinxAtStartPar
Value
&\sphinxstyletheadfamily 
\sphinxAtStartPar
Meaning
\\
\hline
\sphinxAtStartPar
\sphinxhyphen{}32768
&
\sphinxAtStartPar
No data
\\
\hline
\sphinxAtStartPar
\sphinxhyphen{}3
&
\sphinxAtStartPar
Significant decline (p \textgreater{} .99)
\\
\hline
\sphinxAtStartPar
\sphinxhyphen{}2
&
\sphinxAtStartPar
Significant decline (p \textgreater{} .95)
\\
\hline
\sphinxAtStartPar
\sphinxhyphen{}1
&
\sphinxAtStartPar
Significant decline (p \textgreater{} .90)
\\
\hline
\sphinxAtStartPar
0
&
\sphinxAtStartPar
No significant change
\\
\hline
\sphinxAtStartPar
1
&
\sphinxAtStartPar
Significant increase (p \textgreater{} .90)
\\
\hline
\sphinxAtStartPar
2
&
\sphinxAtStartPar
Significant increase (p \textgreater{} .95)
\\
\hline
\sphinxAtStartPar
3
&
\sphinxAtStartPar
Significant increase (p \textgreater{} .99)
\\
\hline
\end{tabulary}
\par
\sphinxattableend\end{savenotes}


\subsection{Productivity performance (degradation)}
\label{\detokenize{Introduction/Layers:productivity-performance-degradation}}

\begin{savenotes}\sphinxattablestart
\centering
\begin{tabulary}{\linewidth}[t]{|T|T|}
\hline
\sphinxstyletheadfamily 
\sphinxAtStartPar
Value
&\sphinxstyletheadfamily 
\sphinxAtStartPar
Meaning
\\
\hline
\sphinxAtStartPar
\sphinxhyphen{}32768
&
\sphinxAtStartPar
No data
\\
\hline
\sphinxAtStartPar
\sphinxhyphen{}1
&
\sphinxAtStartPar
Degradation
\\
\hline
\sphinxAtStartPar
0
&
\sphinxAtStartPar
No change
\\
\hline
\end{tabulary}
\par
\sphinxattableend\end{savenotes}


\subsection{Productivity performance (ratio)}
\label{\detokenize{Introduction/Layers:productivity-performance-ratio}}

\begin{savenotes}\sphinxattablestart
\centering
\begin{tabulary}{\linewidth}[t]{|T|T|}
\hline
\sphinxstyletheadfamily 
\sphinxAtStartPar
Value
&\sphinxstyletheadfamily 
\sphinxAtStartPar
Meaning
\\
\hline
\sphinxAtStartPar
\sphinxhyphen{}32768
&
\sphinxAtStartPar
No data
\\
\hline
\sphinxAtStartPar
0
&
\sphinxAtStartPar
Ratio of mean NDVI and maximum productivity. See background on
\sphinxhref{../background/understanding\_indicators15.html\#productivity-performance}{performance}.
\\
\hline
\end{tabulary}
\par
\sphinxattableend\end{savenotes}


\subsection{Productivity performance (units)}
\label{\detokenize{Introduction/Layers:productivity-performance-units}}

\begin{savenotes}\sphinxattablestart
\centering
\begin{tabulary}{\linewidth}[t]{|T|T|}
\hline
\sphinxstyletheadfamily 
\sphinxAtStartPar
Value
&\sphinxstyletheadfamily 
\sphinxAtStartPar
Meaning
\\
\hline
\sphinxAtStartPar
\sphinxhyphen{}32768
&
\sphinxAtStartPar
No data
\\
\hline
\sphinxAtStartPar
Any other value
&
\sphinxAtStartPar
ID number of unit used to calculate performance. See background on
\sphinxhref{../background/understanding\_indicators15.html\#productivity-performance}{performance}.
\\
\hline
\end{tabulary}
\par
\sphinxattableend\end{savenotes}


\subsection{Productivity state (degradation)}
\label{\detokenize{Introduction/Layers:productivity-state-degradation}}

\begin{savenotes}\sphinxattablestart
\centering
\begin{tabulary}{\linewidth}[t]{|T|T|}
\hline
\sphinxstyletheadfamily 
\sphinxAtStartPar
Value
&\sphinxstyletheadfamily 
\sphinxAtStartPar
Meaning
\\
\hline
\sphinxAtStartPar
\sphinxhyphen{}32768
&
\sphinxAtStartPar
No data
\\
\hline
\sphinxAtStartPar
Any other value
&
\sphinxAtStartPar
Change in productivity state classes between baseline and
target period, calculated as the rank in the target period
minus the rank in the baseline period. Positive values
indicate improvement, negative values indicate decline.
\\
\hline
\end{tabulary}
\par
\sphinxattableend\end{savenotes}


\subsection{Productivity state classes}
\label{\detokenize{Introduction/Layers:productivity-state-classes}}

\begin{savenotes}\sphinxattablestart
\centering
\begin{tabulary}{\linewidth}[t]{|T|T|}
\hline
\sphinxstyletheadfamily 
\sphinxAtStartPar
Value
&\sphinxstyletheadfamily 
\sphinxAtStartPar
Meaning
\\
\hline
\sphinxAtStartPar
\sphinxhyphen{}32768
&
\sphinxAtStartPar
No data
\\
\hline
\sphinxAtStartPar
Any other value
&
\sphinxAtStartPar
Percentile class for productivity state. See background on
\sphinxhref{../background/understanding\_indicators15.html\#productivity-state}{productivity state}.
\\
\hline
\end{tabulary}
\par
\sphinxattableend\end{savenotes}


\subsection{Productivity state NDVI mean}
\label{\detokenize{Introduction/Layers:productivity-state-ndvi-mean}}

\begin{savenotes}\sphinxattablestart
\centering
\begin{tabulary}{\linewidth}[t]{|T|T|}
\hline
\sphinxstyletheadfamily 
\sphinxAtStartPar
Value
&\sphinxstyletheadfamily 
\sphinxAtStartPar
Meaning
\\
\hline
\sphinxAtStartPar
\sphinxhyphen{}32768
&
\sphinxAtStartPar
No data
\\
\hline
\sphinxAtStartPar
Any other value
&
\sphinxAtStartPar
Mean annually integrated NDVI for the baseline period chosen for
productivity state, scaled by 10,000. See background on
\sphinxhref{../background/understanding\_indicators15.html\#productivity-state}{productivity state}.
\\
\hline
\end{tabulary}
\par
\sphinxattableend\end{savenotes}


\subsection{SDG 15.3.1 productivity indicator}
\label{\detokenize{Introduction/Layers:sdg-15-3-1-productivity-indicator}}

\begin{savenotes}\sphinxattablestart
\centering
\begin{tabulary}{\linewidth}[t]{|T|T|}
\hline
\sphinxstyletheadfamily 
\sphinxAtStartPar
Value
&\sphinxstyletheadfamily 
\sphinxAtStartPar
Meaning
\\
\hline
\sphinxAtStartPar
\sphinxhyphen{}32768
&
\sphinxAtStartPar
No data
\\
\hline
\sphinxAtStartPar
1
&
\sphinxAtStartPar
Declining
\\
\hline
\sphinxAtStartPar
2
&
\sphinxAtStartPar
Early signs of decline
\\
\hline
\sphinxAtStartPar
3
&
\sphinxAtStartPar
Stable but stressed
\\
\hline
\sphinxAtStartPar
4
&
\sphinxAtStartPar
Stable
\\
\hline
\sphinxAtStartPar
5
&
\sphinxAtStartPar
Increasing
\\
\hline
\end{tabulary}
\par
\sphinxattableend\end{savenotes}


\subsection{Land productivity dynamics}
\label{\detokenize{Introduction/Layers:land-productivity-dynamics}}

\begin{savenotes}\sphinxattablestart
\centering
\begin{tabulary}{\linewidth}[t]{|T|T|}
\hline
\sphinxstyletheadfamily 
\sphinxAtStartPar
Value
&\sphinxstyletheadfamily 
\sphinxAtStartPar
Meaning
\\
\hline
\sphinxAtStartPar
\sphinxhyphen{}32768
&
\sphinxAtStartPar
No data
\\
\hline
\sphinxAtStartPar
1
&
\sphinxAtStartPar
Declining
\\
\hline
\sphinxAtStartPar
2
&
\sphinxAtStartPar
Moderate decline
\\
\hline
\sphinxAtStartPar
3
&
\sphinxAtStartPar
Stressed
\\
\hline
\sphinxAtStartPar
4
&
\sphinxAtStartPar
Stable
\\
\hline
\sphinxAtStartPar
5
&
\sphinxAtStartPar
Increasing
\\
\hline
\end{tabulary}
\par
\sphinxattableend\end{savenotes}


\section{Land cover}
\label{\detokenize{Introduction/Layers:land-cover}}

\subsection{Land cover (degradation)}
\label{\detokenize{Introduction/Layers:land-cover-degradation}}

\begin{savenotes}\sphinxattablestart
\centering
\begin{tabulary}{\linewidth}[t]{|T|T|}
\hline
\sphinxstyletheadfamily 
\sphinxAtStartPar
Value
&\sphinxstyletheadfamily 
\sphinxAtStartPar
Meaning
\\
\hline
\sphinxAtStartPar
\sphinxhyphen{}32768
&
\sphinxAtStartPar
No data
\\
\hline
\sphinxAtStartPar
\sphinxhyphen{}1
&
\sphinxAtStartPar
Degradation
\\
\hline
\sphinxAtStartPar
0
&
\sphinxAtStartPar
No change
\\
\hline
\sphinxAtStartPar
1
&
\sphinxAtStartPar
Improvement
\\
\hline
\end{tabulary}
\par
\sphinxattableend\end{savenotes}


\subsection{Land cover (7 class)}
\label{\detokenize{Introduction/Layers:land-cover-7-class}}

\begin{savenotes}\sphinxattablestart
\centering
\begin{tabulary}{\linewidth}[t]{|T|T|}
\hline
\sphinxstyletheadfamily 
\sphinxAtStartPar
Value
&\sphinxstyletheadfamily 
\sphinxAtStartPar
Meaning
\\
\hline
\sphinxAtStartPar
\sphinxhyphen{}32768
&
\sphinxAtStartPar
No data
\\
\hline
\sphinxAtStartPar
1
&
\sphinxAtStartPar
Tree\sphinxhyphen{}covered
\\
\hline
\sphinxAtStartPar
2
&
\sphinxAtStartPar
Grasslands
\\
\hline
\sphinxAtStartPar
3
&
\sphinxAtStartPar
Cropland
\\
\hline
\sphinxAtStartPar
4
&
\sphinxAtStartPar
Wetland
\\
\hline
\sphinxAtStartPar
5
&
\sphinxAtStartPar
Artificial
\\
\hline
\sphinxAtStartPar
6
&
\sphinxAtStartPar
Other land
\\
\hline
\sphinxAtStartPar
7
&
\sphinxAtStartPar
Water body
\\
\hline
\end{tabulary}
\par
\sphinxattableend\end{savenotes}


\subsection{Land cover (ESA classes)}
\label{\detokenize{Introduction/Layers:land-cover-esa-classes}}

\begin{savenotes}\sphinxatlongtablestart\begin{longtable}[c]{|l|l|}
\hline
\sphinxstyletheadfamily 
\sphinxAtStartPar
Value
&\sphinxstyletheadfamily 
\sphinxAtStartPar
Meaning
\\
\hline
\endfirsthead

\multicolumn{2}{c}%
{\makebox[0pt]{\sphinxtablecontinued{\tablename\ \thetable{} \textendash{} continued from previous page}}}\\
\hline
\sphinxstyletheadfamily 
\sphinxAtStartPar
Value
&\sphinxstyletheadfamily 
\sphinxAtStartPar
Meaning
\\
\hline
\endhead

\hline
\multicolumn{2}{r}{\makebox[0pt][r]{\sphinxtablecontinued{continues on next page}}}\\
\endfoot

\endlastfoot

\sphinxAtStartPar
\sphinxhyphen{}32768
&
\sphinxAtStartPar
No data
\\
\hline
\sphinxAtStartPar
10
&
\sphinxAtStartPar
Cropland, rainfed
\\
\hline
\sphinxAtStartPar
11
&
\sphinxAtStartPar
Herbaceous cover
\\
\hline
\sphinxAtStartPar
12
&
\sphinxAtStartPar
Tree or shrub cover
\\
\hline
\sphinxAtStartPar
20
&
\sphinxAtStartPar
Cropland, irrigated or post‐flooding
\\
\hline
\sphinxAtStartPar
30
&
\sphinxAtStartPar
Mosaic cropland (\textgreater{}50\%) / natural vegetation (tree, shrub, herbaceous cover) (\textless{}50\%)
\\
\hline
\sphinxAtStartPar
40
&
\sphinxAtStartPar
Mosaic natural vegetation (tree, shrub, herbaceous cover) (\textgreater{}50\%) / cropland (\textless{}50\%)
\\
\hline
\sphinxAtStartPar
50
&
\sphinxAtStartPar
Tree cover, broadleaved, evergreen, closed to open (\textgreater{}15\%)
\\
\hline
\sphinxAtStartPar
60
&
\sphinxAtStartPar
Tree cover, broadleaved, deciduous, closed to open (\textgreater{}15\%)
\\
\hline
\sphinxAtStartPar
61
&
\sphinxAtStartPar
Tree cover, broadleaved, deciduous, closed (\textgreater{}40\%)
\\
\hline
\sphinxAtStartPar
62
&
\sphinxAtStartPar
Tree cover, broadleaved, deciduous, open (15‐40\%)
\\
\hline
\sphinxAtStartPar
70
&
\sphinxAtStartPar
Tree cover, needleleaved, evergreen, closed to open (\textgreater{}15\%)
\\
\hline
\sphinxAtStartPar
71
&
\sphinxAtStartPar
Tree cover, needleleaved, evergreen, closed (\textgreater{}40\%)
\\
\hline
\sphinxAtStartPar
72
&
\sphinxAtStartPar
Tree cover, needleleaved, evergreen, open (15‐40\%)
\\
\hline
\sphinxAtStartPar
80
&
\sphinxAtStartPar
Tree cover, needleleaved, deciduous, closed to open (\textgreater{}15\%)
\\
\hline
\sphinxAtStartPar
81
&
\sphinxAtStartPar
Tree cover, needleleaved, deciduous, closed (\textgreater{}40\%)
\\
\hline
\sphinxAtStartPar
82
&
\sphinxAtStartPar
Tree cover, needleleaved, deciduous, open (15‐40\%)
\\
\hline
\sphinxAtStartPar
90
&
\sphinxAtStartPar
Tree cover, mixed leaf type (broadleaved and needleleaved)
\\
\hline
\sphinxAtStartPar
100
&
\sphinxAtStartPar
Mosaic tree and shrub (\textgreater{}50\%) / herbaceous cover (\textless{}50\%)
\\
\hline
\sphinxAtStartPar
110
&
\sphinxAtStartPar
Mosaic herbaceous cover (\textgreater{}50\%) / tree and shrub (\textless{}50\%)
\\
\hline
\sphinxAtStartPar
120
&
\sphinxAtStartPar
Shrubland
\\
\hline
\sphinxAtStartPar
121
&
\sphinxAtStartPar
Evergreen shrubland
\\
\hline
\sphinxAtStartPar
122
&
\sphinxAtStartPar
Deciduous shrubland
\\
\hline
\sphinxAtStartPar
130
&
\sphinxAtStartPar
Grassland
\\
\hline
\sphinxAtStartPar
140
&
\sphinxAtStartPar
Lichens and mosses
\\
\hline
\sphinxAtStartPar
150
&
\sphinxAtStartPar
Sparse vegetation (tree, shrub, herbaceous cover) (\textless{}15\%)
\\
\hline
\sphinxAtStartPar
151
&
\sphinxAtStartPar
Sparse tree (\textless{}15\%)
\\
\hline
\sphinxAtStartPar
152
&
\sphinxAtStartPar
Sparse shrub (\textless{}15\%)
\\
\hline
\sphinxAtStartPar
153
&
\sphinxAtStartPar
Sparse herbaceous cover (\textless{}15\%)
\\
\hline
\sphinxAtStartPar
160
&
\sphinxAtStartPar
Tree cover, flooded, fresh or brakish water
\\
\hline
\sphinxAtStartPar
170
&
\sphinxAtStartPar
Tree cover, flooded, saline water
\\
\hline
\sphinxAtStartPar
180
&
\sphinxAtStartPar
Shrub or herbaceous cover, flooded, fresh/saline/brakish water
\\
\hline
\sphinxAtStartPar
190
&
\sphinxAtStartPar
Urban areas
\\
\hline
\sphinxAtStartPar
200
&
\sphinxAtStartPar
Bare areas
\\
\hline
\sphinxAtStartPar
201
&
\sphinxAtStartPar
Consolidated bare areas
\\
\hline
\sphinxAtStartPar
202
&
\sphinxAtStartPar
Unconsolidated bare areas
\\
\hline
\sphinxAtStartPar
210
&
\sphinxAtStartPar
Water bodies
\\
\hline
\sphinxAtStartPar
220
&
\sphinxAtStartPar
Permanent snow and ice
\\
\hline
\end{longtable}\sphinxatlongtableend\end{savenotes}


\subsection{Land cover (transitions)}
\label{\detokenize{Introduction/Layers:land-cover-transitions}}

\begin{savenotes}\sphinxatlongtablestart\begin{longtable}[c]{|l|l|}
\hline
\sphinxstyletheadfamily 
\sphinxAtStartPar
Value
&\sphinxstyletheadfamily 
\sphinxAtStartPar
Meaning
\\
\hline
\endfirsthead

\multicolumn{2}{c}%
{\makebox[0pt]{\sphinxtablecontinued{\tablename\ \thetable{} \textendash{} continued from previous page}}}\\
\hline
\sphinxstyletheadfamily 
\sphinxAtStartPar
Value
&\sphinxstyletheadfamily 
\sphinxAtStartPar
Meaning
\\
\hline
\endhead

\hline
\multicolumn{2}{r}{\makebox[0pt][r]{\sphinxtablecontinued{continues on next page}}}\\
\endfoot

\endlastfoot

\sphinxAtStartPar
\sphinxhyphen{}32768
&
\sphinxAtStartPar
No data
\\
\hline
\sphinxAtStartPar
1
&
\sphinxAtStartPar
Tree\sphinxhyphen{}covered \sphinxhyphen{} Tree\sphinxhyphen{}covered (no change)
\\
\hline
\sphinxAtStartPar
2
&
\sphinxAtStartPar
Grassland \sphinxhyphen{} Grassland (no change)
\\
\hline
\sphinxAtStartPar
3
&
\sphinxAtStartPar
Cropland \sphinxhyphen{} Cropland (no change)
\\
\hline
\sphinxAtStartPar
4
&
\sphinxAtStartPar
Wetland \sphinxhyphen{} Wetland (no change)
\\
\hline
\sphinxAtStartPar
5
&
\sphinxAtStartPar
Artificial \sphinxhyphen{} Artificial (no change)
\\
\hline
\sphinxAtStartPar
6
&
\sphinxAtStartPar
Other land \sphinxhyphen{} Other land (no change)
\\
\hline
\sphinxAtStartPar
7
&
\sphinxAtStartPar
Water body \sphinxhyphen{} Water body (no change)
\\
\hline
\sphinxAtStartPar
12
&
\sphinxAtStartPar
Forest \sphinxhyphen{} Grassland
\\
\hline
\sphinxAtStartPar
13
&
\sphinxAtStartPar
Forest \sphinxhyphen{} Cropland
\\
\hline
\sphinxAtStartPar
14
&
\sphinxAtStartPar
Forest \sphinxhyphen{} Wetland
\\
\hline
\sphinxAtStartPar
15
&
\sphinxAtStartPar
Forest \sphinxhyphen{} Artificial
\\
\hline
\sphinxAtStartPar
16
&
\sphinxAtStartPar
Forest \sphinxhyphen{} Other land
\\
\hline
\sphinxAtStartPar
17
&
\sphinxAtStartPar
Forest \sphinxhyphen{} Water body
\\
\hline
\sphinxAtStartPar
21
&
\sphinxAtStartPar
Grassland \sphinxhyphen{} Forest
\\
\hline
\sphinxAtStartPar
23
&
\sphinxAtStartPar
Grassland \sphinxhyphen{} Cropland
\\
\hline
\sphinxAtStartPar
24
&
\sphinxAtStartPar
Grassland \sphinxhyphen{} Wetland
\\
\hline
\sphinxAtStartPar
25
&
\sphinxAtStartPar
Grassland \sphinxhyphen{} Artificial
\\
\hline
\sphinxAtStartPar
26
&
\sphinxAtStartPar
Grassland \sphinxhyphen{} Other land
\\
\hline
\sphinxAtStartPar
27
&
\sphinxAtStartPar
Grassland \sphinxhyphen{} Water body
\\
\hline
\sphinxAtStartPar
31
&
\sphinxAtStartPar
Cropland \sphinxhyphen{} Forest
\\
\hline
\sphinxAtStartPar
32
&
\sphinxAtStartPar
Cropland \sphinxhyphen{} Grassland
\\
\hline
\sphinxAtStartPar
34
&
\sphinxAtStartPar
Cropland \sphinxhyphen{} Wetland
\\
\hline
\sphinxAtStartPar
35
&
\sphinxAtStartPar
Cropland \sphinxhyphen{} Artificial
\\
\hline
\sphinxAtStartPar
36
&
\sphinxAtStartPar
Cropland \sphinxhyphen{} Other land
\\
\hline
\sphinxAtStartPar
37
&
\sphinxAtStartPar
Cropland \sphinxhyphen{} Water body
\\
\hline
\sphinxAtStartPar
41
&
\sphinxAtStartPar
Wetland \sphinxhyphen{} Forest
\\
\hline
\sphinxAtStartPar
42
&
\sphinxAtStartPar
Wetland \sphinxhyphen{} Grassland
\\
\hline
\sphinxAtStartPar
43
&
\sphinxAtStartPar
Wetland \sphinxhyphen{} Cropland
\\
\hline
\sphinxAtStartPar
45
&
\sphinxAtStartPar
Wetland \sphinxhyphen{} Artificial
\\
\hline
\sphinxAtStartPar
46
&
\sphinxAtStartPar
Wetland \sphinxhyphen{} Other land
\\
\hline
\sphinxAtStartPar
47
&
\sphinxAtStartPar
Wetland \sphinxhyphen{} Water body
\\
\hline
\sphinxAtStartPar
51
&
\sphinxAtStartPar
Artificial \sphinxhyphen{} Forest
\\
\hline
\sphinxAtStartPar
52
&
\sphinxAtStartPar
Artificial \sphinxhyphen{} Grassland
\\
\hline
\sphinxAtStartPar
53
&
\sphinxAtStartPar
Artificial \sphinxhyphen{} Cropland
\\
\hline
\sphinxAtStartPar
54
&
\sphinxAtStartPar
Artificial \sphinxhyphen{} Wetland
\\
\hline
\sphinxAtStartPar
56
&
\sphinxAtStartPar
Artificial \sphinxhyphen{} Other land
\\
\hline
\sphinxAtStartPar
57
&
\sphinxAtStartPar
Artificial \sphinxhyphen{} Water body
\\
\hline
\sphinxAtStartPar
61
&
\sphinxAtStartPar
Other land \sphinxhyphen{} Forest
\\
\hline
\sphinxAtStartPar
62
&
\sphinxAtStartPar
Other land \sphinxhyphen{} Grassland
\\
\hline
\sphinxAtStartPar
63
&
\sphinxAtStartPar
Other land \sphinxhyphen{} Cropland
\\
\hline
\sphinxAtStartPar
64
&
\sphinxAtStartPar
Other land \sphinxhyphen{} Wetland
\\
\hline
\sphinxAtStartPar
65
&
\sphinxAtStartPar
Other land \sphinxhyphen{} Artificial
\\
\hline
\sphinxAtStartPar
67
&
\sphinxAtStartPar
Other land \sphinxhyphen{} Water body
\\
\hline
\sphinxAtStartPar
71
&
\sphinxAtStartPar
Water body \sphinxhyphen{} Forest
\\
\hline
\sphinxAtStartPar
72
&
\sphinxAtStartPar
Water body \sphinxhyphen{} Grassland
\\
\hline
\sphinxAtStartPar
73
&
\sphinxAtStartPar
Water body \sphinxhyphen{} Cropland
\\
\hline
\sphinxAtStartPar
74
&
\sphinxAtStartPar
Water body \sphinxhyphen{} Wetland
\\
\hline
\sphinxAtStartPar
75
&
\sphinxAtStartPar
Water body \sphinxhyphen{} Artificial
\\
\hline
\sphinxAtStartPar
76
&
\sphinxAtStartPar
Water body \sphinxhyphen{} Other land
\\
\hline
\end{longtable}\sphinxatlongtableend\end{savenotes}


\section{Soil organic carbon}
\label{\detokenize{Introduction/Layers:soil-organic-carbon}}

\subsection{Soil organic carbon (degradation)}
\label{\detokenize{Introduction/Layers:soil-organic-carbon-degradation}}

\begin{savenotes}\sphinxattablestart
\centering
\begin{tabulary}{\linewidth}[t]{|T|T|}
\hline
\sphinxstyletheadfamily 
\sphinxAtStartPar
Value
&\sphinxstyletheadfamily 
\sphinxAtStartPar
Meaning
\\
\hline
\sphinxAtStartPar
\sphinxhyphen{}32768
&
\sphinxAtStartPar
No data
\\
\hline
\sphinxAtStartPar
Any other value
&
\sphinxAtStartPar
Percentage change in soil organic carbon content (0 \sphinxhyphen{} 30 cm depth)
from baseline to target year. Positive values indicate increase,
negative values indicate decrease.
\\
\hline
\end{tabulary}
\par
\sphinxattableend\end{savenotes}


\subsection{Soil organic carbon}
\label{\detokenize{Introduction/Layers:id3}}

\begin{savenotes}\sphinxattablestart
\centering
\begin{tabulary}{\linewidth}[t]{|T|T|}
\hline
\sphinxstyletheadfamily 
\sphinxAtStartPar
Value
&\sphinxstyletheadfamily 
\sphinxAtStartPar
Meaning
\\
\hline
\sphinxAtStartPar
\sphinxhyphen{}32768
&
\sphinxAtStartPar
No data
\\
\hline
\sphinxAtStartPar
Any other value
&
\sphinxAtStartPar
Soil organic carbon content (0 \sphinxhyphen{} 30 cm depth) in metric tons per hectare
\\
\hline
\end{tabulary}
\par
\sphinxattableend\end{savenotes}


\subsection{Population}
\label{\detokenize{Introduction/Layers:population}}

\begin{savenotes}\sphinxattablestart
\centering
\begin{tabulary}{\linewidth}[t]{|T|T|}
\hline
\sphinxstyletheadfamily 
\sphinxAtStartPar
Value
&\sphinxstyletheadfamily 
\sphinxAtStartPar
Meaning
\\
\hline
\sphinxAtStartPar
\sphinxhyphen{}32768
&
\sphinxAtStartPar
No data
\\
\hline
\sphinxAtStartPar
Any other value
&
\sphinxAtStartPar
Total population within grid cell
\\
\hline
\end{tabulary}
\par
\sphinxattableend\end{savenotes}


\subsection{Delta Normalized Burnt Ratio}
\label{\detokenize{Introduction/Layers:delta-normalized-burnt-ratio}}

\begin{savenotes}\sphinxattablestart
\centering
\begin{tabulary}{\linewidth}[t]{|T|T|}
\hline
\sphinxstyletheadfamily 
\sphinxAtStartPar
Value
&\sphinxstyletheadfamily 
\sphinxAtStartPar
Meaning
\\
\hline
\sphinxAtStartPar
\sphinxhyphen{}500
&
\sphinxAtStartPar
No data
\\
\hline
\sphinxAtStartPar
\sphinxhyphen{}350
&
\sphinxAtStartPar
Hight Severety
\\
\hline
\sphinxAtStartPar
\sphinxhyphen{}300
&
\sphinxAtStartPar
Moderate High Severety
\\
\hline
\sphinxAtStartPar
\sphinxhyphen{}200
&
\sphinxAtStartPar
Moderate Low Severety
\\
\hline
\sphinxAtStartPar
\sphinxhyphen{}100
&
\sphinxAtStartPar
Low Severety
\\
\hline
\sphinxAtStartPar
100
&
\sphinxAtStartPar
Unburned
\\
\hline
\sphinxAtStartPar
300
&
\sphinxAtStartPar
Enhanced Growth Low
\\
\hline
\sphinxAtStartPar
1000
&
\sphinxAtStartPar
Enhanced Growth High
\\
\hline
\end{tabulary}
\par
\sphinxattableend\end{savenotes}

\sphinxstepscope


\chapter{Frequently asked questions}
\label{\detokenize{Introduction/faq:frequently-asked-questions}}\label{\detokenize{Introduction/faq::doc}}
\sphinxAtStartPar
This page lists some Frequently Asked Questions (FAQs) for the MISLAND\sphinxhyphen{}North Africa
tool.


\section{Installation of MISLAND\sphinxhyphen{}North Africa}
\label{\detokenize{Introduction/faq:installation-of-misland-north-africa}}

\subsection{What version of Quantum GIS (QGIS) do I need for the toolbox?}
\label{\detokenize{Introduction/faq:what-version-of-quantum-gis-qgis-do-i-need-for-the-toolbox}}
\sphinxAtStartPar
To download QGIS, please go to the QGIS Downloads page. As of February 2018,
version 3.0 was released.


\subsection{Do I need to download a 32\sphinxhyphen{}bit or 64 bit version of QGIS?}
\label{\detokenize{Introduction/faq:do-i-need-to-download-a-32-bit-or-64-bit-version-of-qgis}}
\sphinxAtStartPar
We recommend downloading 64\sphinxhyphen{}bit version (2.18), but you may need to download
the 32\sphinxhyphen{}bit version for 32\sphinxhyphen{}bit operating systems. To find out if your computer
is running a 32\sphinxhyphen{}bit or 64\sphinxhyphen{}bit version of Windows,  search for System or
msinfo32. This is found in the Control Panel and will bring up a window that
says the system type e.g. System type: 64\sphinxhyphen{}bit Operating System, x64\sphinxhyphen{}based
processor.

\sphinxAtStartPar
Windows 7 or Windows Vista:
\begin{enumerate}
\sphinxsetlistlabels{\arabic}{enumi}{enumii}{}{.}%
\item {} 
\sphinxAtStartPar
Open System by clicking the Start button , right\sphinxhyphen{}clicking Computer, and then
clicking Properties.

\item {} 
\sphinxAtStartPar
Under System, you can view the system type.

\end{enumerate}

\sphinxAtStartPar
Windows 8 or Windows 10:
\begin{enumerate}
\sphinxsetlistlabels{\arabic}{enumi}{enumii}{}{.}%
\item {} 
\sphinxAtStartPar
From the Start screen, type This PC.

\item {} 
\sphinxAtStartPar
Right Click (or tap and hold) This PC, and click Properties.

\end{enumerate}

\sphinxAtStartPar
Mac:
\begin{enumerate}
\sphinxsetlistlabels{\arabic}{enumi}{enumii}{}{.}%
\item {} 
\sphinxAtStartPar
Click the Apple icon in the top left and select “About this Mac”.

\item {} 
\sphinxAtStartPar
For more advanced details click “More Info…” in the About This Mac window.

\end{enumerate}


\subsection{How do I install the plugin?}
\label{\detokenize{Introduction/faq:how-do-i-install-the-plugin}}
\sphinxAtStartPar
Open QGIS, navigate to Plugins on the menu bar, and select Install From Zipfile and
install the Zipfile provided.


\subsection{How do I upgrade the plugin?}
\label{\detokenize{Introduction/faq:how-do-i-upgrade-the-plugin}}
\sphinxAtStartPar
If you have already installed the plugin, navigate to Plugins on the menu bar,
and select Manage and install plugins. On the side menu, select Installed to
view the plugins that you have installed in your computer. At the bottom of the
window, select Upgrade all to upgrade the toolbox to the latest version.


\subsection{How do I uninstall the plugin?}
\label{\detokenize{Introduction/faq:how-do-i-uninstall-the-plugin}}
\sphinxAtStartPar
If you would like to uninstall the plugin, normally you can do so with the QGIS
plugins manager. To access the tool, choose “Plugins” and then “Manage and
Install Plugins…” from the QGIS menu bar. From the plugin manager screen,
select “Installed” from the menu on the left\sphinxhyphen{}hand side. Then click on
“MISLAND\sphinxhyphen{}North Africa” in the list of plugins, and on “Uninstall Plugin” to uninstall
it.

\sphinxAtStartPar
If you encounter an error uninstalling the plugin, it is also possible to
remove it manually. To manually remove the plugin:
\begin{enumerate}
\sphinxsetlistlabels{\arabic}{enumi}{enumii}{}{.}%
\item {} 
\sphinxAtStartPar
Open QGIS

\item {} 
\sphinxAtStartPar
Navigate to where the plugin is installed by selecting “Open Active Profile
Folder” from the menu under “Settings” \sphinxhyphen{} “User Profiles” on the menu bar.

\item {} 
\sphinxAtStartPar
Quit QGIS. You may not be able to uninstall the plugin if QGIS is not
closed.

\item {} 
\sphinxAtStartPar
In the file browser window that opened, double click on “python”, and then
double click on “plugins”. Delete the LDMP folder within that directory.

\item {} 
\sphinxAtStartPar
Restart QGIS.

\end{enumerate}


\section{Datasets}
\label{\detokenize{Introduction/faq:datasets}}

\subsection{When will you update datasets for the current year?}
\label{\detokenize{Introduction/faq:when-will-you-update-datasets-for-the-current-year}}
\sphinxAtStartPar
MISLAND\sphinxhyphen{}North Africa uses publicly available data, as such the most up to date datasets
will be added to the toolbox as soon as the original data providers make them
public. If you notice any update that we missed, please do let us know.


\subsection{Is there an option to download the original data?}
\label{\detokenize{Introduction/faq:is-there-an-option-to-download-the-original-data}}
\sphinxAtStartPar
Users can download the original data using the Download option within the
toolbox.


\subsection{Will the toolbox support higher resolution datasets?}
\label{\detokenize{Introduction/faq:will-the-toolbox-support-higher-resolution-datasets}}
\sphinxAtStartPar
The toolbox currently supports AVHRR (8km), LANDSAT 7 (30m) and MODIS (250m) data for primary
productivity analysis, and ESA LCC CCI (300m) for land cover change analysis.


\section{Methods}
\label{\detokenize{Introduction/faq:methods}}

\subsection{Who was the default time period for the analysis determined?}
\label{\detokenize{Introduction/faq:who-was-the-default-time-period-for-the-analysis-determined}}
\sphinxAtStartPar
The default time period of analysis is from years 2001 to 2015. These were
recommended by the \sphinxhref{http://www2.unccd.int/sites/default/files/relevant-links/2017-10/Good\%20Practice\%20Guidance\_SDG\%20Indicator\%2015.3.1\_Version\%201.0.pdf}{Good Practice Guidelines}.,
a document that provides detailed recommendations for measuring land
degradation and has been adopted by the UNCCD.


\subsection{Productivity}
\label{\detokenize{Introduction/faq:productivity}}

\subsubsection{How does the result provided by state differs from trajectory?}
\label{\detokenize{Introduction/faq:how-does-the-result-provided-by-state-differs-from-trajectory}}
\sphinxAtStartPar
The trajectory analysis uses linear regressions and non\sphinxhyphen{}parametric tests to
identify long term significant trends in primary productivity. This method
however, is not able to capture more recent changes in primary productivity,
which could be signals of short term processes of improvement or degradation.
By comparing a long term mean to the most recent period, state is able to
capture such recent changes.


\subsection{Land cover}
\label{\detokenize{Introduction/faq:land-cover}}

\subsubsection{Currently, the land cover aggregation is done following the UNCCD guidelines, but that classification does not take into account country level characteristics. Could it be possible to allow the user to define the aggregation criteria?}
\label{\detokenize{Introduction/faq:currently-the-land-cover-aggregation-is-done-following-the-unccd-guidelines-but-that-classification-does-not-take-into-account-country-level-characteristics-could-it-be-possible-to-allow-the-user-to-define-the-aggregation-criteria}}
\sphinxAtStartPar
Users are able to make these changes using the advanced settings in the land
cover GUI so that appropriate aggregations occur depending on the context of
your country.


\subsubsection{How can we isolate woody plant encroachment within the toolbox?}
\label{\detokenize{Introduction/faq:how-can-we-isolate-woody-plant-encroachment-within-the-toolbox}}
\sphinxAtStartPar
This can be altered using the land cover change matrix in the toolbox. For
every transition, the user can mark the change as stable, improvement or
degraded. The transition from grassland/rangeland to shrubland may indicate
woody encroachment and this transition can be marked as an indicator of
degradation.


\subsection{Carbon stocks}
\label{\detokenize{Introduction/faq:carbon-stocks}}

\subsubsection{Why use soil organic carbon (SOC) instead of above and below\sphinxhyphen{}ground carbon to  measure carbon stocks?}
\label{\detokenize{Introduction/faq:why-use-soil-organic-carbon-soc-instead-of-above-and-below-ground-carbon-to-measure-carbon-stocks}}
\sphinxAtStartPar
The original proposed indicator is Carbon Stocks, which would include above and
below ground biomass. However, given the lack of consistently generated and
comparable dataset which assess carbon stocks in woody plants (including
shrubs), grasses, croplands, and other land cover types both above and below
ground, the \sphinxhref{http://www2.unccd.int/sites/default/files/relevant-links/2017-10/Good\%20Practice\%20Guidance\_SDG\%20Indicator\%2015.3.1\_Version\%201.0.pdf}{Good Practice Guidelines}
published by the UNCCD recommends for the time being to use SOC as a proxy.


\subsubsection{Is it possible to measure identify processes of degradation linked to salinization using this tool?}
\label{\detokenize{Introduction/faq:is-it-possible-to-measure-identify-processes-of-degradation-linked-to-salinization-using-this-tool}}
\sphinxAtStartPar
Not directly. If salinization caused a reduction in primary productivity, that
decrease would be identified by the productivity indicators, but the users
would have to use their local knowledge to assign the causes.


\section{Land degradation outputs}
\label{\detokenize{Introduction/faq:land-degradation-outputs}}

\subsection{How were the layers combined to define the final land degradation layer?}
\label{\detokenize{Introduction/faq:how-were-the-layers-combined-to-define-the-final-land-degradation-layer}}
\sphinxAtStartPar
Performance, state, and trajectory (the three indicators of change in
{\hyperref[\detokenize{Introduction/faq:productivity}]{\sphinxcrossref{productivity}}}) are combined following a modified version of the good practice
guidance developed by the UNCCD (in section SDG Indicator 15.3.1 of this manual
a table is presented). Productivity, soil carbon, and land cover chance (the
three sub\sphinxhyphen{}indicators of SDG 15.3.1) are combined using a “one out, all out”
principle. In other words: if there is a decline in any of the three indicators
at a particular pixel, then that pixel is mapped as being “degraded”.


\subsection{Why do I see areas the data says are improving or degrading when I know they are not?}
\label{\detokenize{Introduction/faq:why-do-i-see-areas-the-data-says-are-improving-or-degrading-when-i-know-they-are-not}}
\sphinxAtStartPar
The final output should be interpreted as showing areas potentially degraded.
The indicator of land degradation is based on changes in productivity, land
cover and soil organic carbon. Several factor could lead to the identification
of patterns of degradation which do not seem to correlate to what is happening
on the ground, the date of analysis being a very important one. If the climatic
conditions at the beginning of the analysis were particularly wet, for example,
trends from that moment on could show significant decreases in primary
productivity, and degradation. The user can use LMDS to address some of
this issues correcting by the effect of climate. The resolution of the data
could potentially be another limitation. MISLAND\sphinxhyphen{}North Africa by default uses global
datasets which will not be the most relevant at all scales and geographies. A
functionality to use local data will be added shortly.


\subsection{All of the sub\sphinxhyphen{}indicators are measuring vegetation: how does this contribute to understanding and identifying land degradation?}
\label{\detokenize{Introduction/faq:all-of-the-sub-indicators-are-measuring-vegetation-how-does-this-contribute-to-understanding-and-identifying-land-degradation}}
\sphinxAtStartPar
Vegetation is a key component of most ecosystems, and serve as a good proxy for
their overall functioning and health. The three subindicators used for SDG
15.3.1 measure different aspects of land cover, which do relate to vegetation.
Primary productivity directly measures the change in amount of biomass present
in one area, but it does not inform us if that change is positive or not (not
all increases in plant biomass should be interpreted as improvement). Land
cover fills that gap by interpreting the landscape from a thematic perspective
looking at what was there before and what is there now. It does include
vegetation, but also bare land, urban and water. Finally, the soil organic
carbon indicator uses the land cover map to inform the changes in soil organic
carbon over time. This method is not ideal, but given the current state of
global soil science and surveying, there is consensus that it this point in
time and globally, this is the best approach.


\section{Future plans}
\label{\detokenize{Introduction/faq:future-plans}}

\subsection{When will there be an offline version of the toolbox?}
\label{\detokenize{Introduction/faq:when-will-there-be-an-offline-version-of-the-toolbox}}
\sphinxAtStartPar
The final toolbox will be available as both as an offline and online version.
The online version allows users to access current datasets more easily, while
also allowing users to leverage Google Earth Engine to provide computing in the
cloud. An offline version allows users to access data and perform analyses
where internet connectivity may be limited, but it does have the disadvantage
of requiring users to have enough local computing capacity to run analyses
locally. The technical team intends to build the offline version of the toolbox
and provide countries with data relevant for reporting at the national level
within the pilot project countries.

\sphinxstepscope


\chapter{Land Degradation}
\label{\detokenize{Background/LD_indicators:land-degradation}}\label{\detokenize{Background/LD_indicators::doc}}
\begin{sphinxShadowBox}
\sphinxstyletopictitle{Contents}
\begin{itemize}
\item {} 
\sphinxAtStartPar
\phantomsection\label{\detokenize{Background/LD_indicators:id7}}{\hyperref[\detokenize{Background/LD_indicators:land-degradation}]{\sphinxcrossref{Land Degradation}}}
\begin{itemize}
\item {} 
\sphinxAtStartPar
\phantomsection\label{\detokenize{Background/LD_indicators:id8}}{\hyperref[\detokenize{Background/LD_indicators:land-degradation-indicators}]{\sphinxcrossref{Land Degradation Indicators}}}
\begin{itemize}
\item {} 
\sphinxAtStartPar
\phantomsection\label{\detokenize{Background/LD_indicators:id9}}{\hyperref[\detokenize{Background/LD_indicators:sdg15-3-1-indicator}]{\sphinxcrossref{SDG15.3.1 Indicator}}}
\begin{itemize}
\item {} 
\sphinxAtStartPar
\phantomsection\label{\detokenize{Background/LD_indicators:id10}}{\hyperref[\detokenize{Background/LD_indicators:sdg-15-3-1-sub-indicators}]{\sphinxcrossref{SDG 15.3.1 Sub\sphinxhyphen{}indicators}}}
\begin{itemize}
\item {} 
\sphinxAtStartPar
\phantomsection\label{\detokenize{Background/LD_indicators:id11}}{\hyperref[\detokenize{Background/LD_indicators:productivity}]{\sphinxcrossref{Productivity}}}
\begin{itemize}
\item {} 
\sphinxAtStartPar
\phantomsection\label{\detokenize{Background/LD_indicators:id12}}{\hyperref[\detokenize{Background/LD_indicators:trajectory}]{\sphinxcrossref{Trajectory}}}

\item {} 
\sphinxAtStartPar
\phantomsection\label{\detokenize{Background/LD_indicators:id13}}{\hyperref[\detokenize{Background/LD_indicators:state}]{\sphinxcrossref{State}}}

\item {} 
\sphinxAtStartPar
\phantomsection\label{\detokenize{Background/LD_indicators:id14}}{\hyperref[\detokenize{Background/LD_indicators:perfomance}]{\sphinxcrossref{Perfomance}}}

\end{itemize}

\item {} 
\sphinxAtStartPar
\phantomsection\label{\detokenize{Background/LD_indicators:id15}}{\hyperref[\detokenize{Background/LD_indicators:landcover}]{\sphinxcrossref{Landcover}}}

\item {} 
\sphinxAtStartPar
\phantomsection\label{\detokenize{Background/LD_indicators:id16}}{\hyperref[\detokenize{Background/LD_indicators:carbon-stocks}]{\sphinxcrossref{Carbon\sphinxhyphen{}stocks}}}

\end{itemize}

\item {} 
\sphinxAtStartPar
\phantomsection\label{\detokenize{Background/LD_indicators:id17}}{\hyperref[\detokenize{Background/LD_indicators:combining-productivity-indicators}]{\sphinxcrossref{Combining Productivity Indicators}}}

\end{itemize}

\item {} 
\sphinxAtStartPar
\phantomsection\label{\detokenize{Background/LD_indicators:id18}}{\hyperref[\detokenize{Background/LD_indicators:vegetation-loss-gain-hotspots}]{\sphinxcrossref{Vegetation Loss/Gain hotspots}}}

\item {} 
\sphinxAtStartPar
\phantomsection\label{\detokenize{Background/LD_indicators:id19}}{\hyperref[\detokenize{Background/LD_indicators:forest-change}]{\sphinxcrossref{Forest Change}}}
\begin{itemize}
\item {} 
\sphinxAtStartPar
\phantomsection\label{\detokenize{Background/LD_indicators:id20}}{\hyperref[\detokenize{Background/LD_indicators:forest-gain-loss}]{\sphinxcrossref{Forest Gain/Loss}}}

\item {} 
\sphinxAtStartPar
\phantomsection\label{\detokenize{Background/LD_indicators:id21}}{\hyperref[\detokenize{Background/LD_indicators:forest-fires}]{\sphinxcrossref{Forest Fires}}}

\end{itemize}

\end{itemize}

\end{itemize}

\end{itemize}
\end{sphinxShadowBox}


\section{Land Degradation Indicators}
\label{\detokenize{Background/LD_indicators:land-degradation-indicators}}
\sphinxAtStartPar
Land degradation, as defined by the United Nations Convention to Combat Desertification (UNCCD), is a complex process that refers to the long\sphinxhyphen{}lasting reduction or loss of biological and economic productivity of lands, caused by human activities, sometimes exacerbated by natural phenomena. Terrestrial vegetation including crops depend on appropriate soil which is the substrate on which vegetation/crops grow, besides other climatic factor requirements.

\sphinxAtStartPar
Different land masses are however affected by different factors at different levels. The factors pan from the climatic to soil properties, from land use land cover and to surface roughness which depends on the conditions that a given land mass is exposed to. Apart from the natural and geophysical causes, land degradation may also be influenced by anthropogenic factors which yield conditions for land degradation to take place, these activities may span from uncouth agricultural practices, desertification through illegal logging, top soil harvesting, mining activities among others.

\sphinxAtStartPar
The OSS.LDMS focuses on provision of evidence\sphinxhyphen{}based proofs on land degradation and its spatiotemporal distribution and therefore the hotspots where priority actions should be taken or awareness\sphinxhyphen{}raising campaigns should be planed. The figure below show key land degradation indicators included in the OSS.LDMS service

\begin{figure}[H]
\centering
\capstart

\noindent\sphinxincludegraphics[width=351\sphinxpxdimen,height=341\sphinxpxdimen]{{service}.png}
\caption{Summary of included services on the OSS.LDMS service platform}\label{\detokenize{Background/LD_indicators:id1}}\end{figure}


\subsection{SDG15.3.1 Indicator}
\label{\detokenize{Background/LD_indicators:sdg15-3-1-indicator}}
\sphinxAtStartPar
As part of the Sustainable development Goals(SDGs), SDG 15 is to: “Protect, restore and promote sustainable use of terrestrial ecosystems, sustainably manage forest, combat desertification, and halt and reverse land degradation and halt biodiversity loss”

\sphinxAtStartPar
Target 15.3 aims to: “By 2030, combat desertification, restore degraded land and soil, including land affected by desertification, drought, floods, and strive to achieve land degradation\sphinxhyphen{}neutral world”

\sphinxAtStartPar
The indicator used to assess the progress of each SDG target is the 15.3.1 indicator: “Proportion of land that is degraded over total land area’’

\sphinxAtStartPar
The basic land degradation indicators include three main sub\sphinxhyphen{}indicators of the SDG target 15.3.1 (proportion of land that is degraded over the total land area). As the custodian agency of SDG 15.3, the United Nations Convention to Combat Desertification (UNCCD) has developed recommendations/Good practice guide on how to compute SDG indicator 15.3.1  from 3 sub\sphinxhyphen{}indicators:
\begin{itemize}
\item {} 
\sphinxAtStartPar
Vegetation productivity

\item {} 
\sphinxAtStartPar
Landcover

\item {} 
\sphinxAtStartPar
Soil Organic carbon

\end{itemize}

\begin{figure}[H]
\centering
\capstart

\noindent\sphinxincludegraphics[width=500\sphinxpxdimen,height=300\sphinxpxdimen]{{sdg}.png}
\caption{SDG 15.3.1 Indicators}\label{\detokenize{Background/LD_indicators:id2}}\end{figure}


\subsubsection{SDG 15.3.1 Sub\sphinxhyphen{}indicators}
\label{\detokenize{Background/LD_indicators:sdg-15-3-1-sub-indicators}}

\paragraph{Productivity}
\label{\detokenize{Background/LD_indicators:productivity}}
\sphinxAtStartPar
Land productivity is the biological productive capacity of the land (i.e. the ability to produce food, fibre and fuel that sustain life). For easy interpretation the annual mean vegetation indices values at the pixel level will be used to assess three measures of change (trajectory, state and performance) as summarized in the figure below and explained in the subsequent sub\sphinxhyphen{}sections:

\begin{figure}[H]
\centering
\capstart

\noindent\sphinxincludegraphics[width=700\sphinxpxdimen,height=500\sphinxpxdimen]{{sdgmethodology}.png}
\caption{Sammury methodology for conputing Land Productivity}\label{\detokenize{Background/LD_indicators:id3}}\end{figure}


\subparagraph{Trajectory}
\label{\detokenize{Background/LD_indicators:trajectory}}
\sphinxAtStartPar
The rate of change in primary productivity over time which will be computed using linear regression at the pixel level for various Landsat derived vegetation indices (NDVI, MSAVI2, SAVI). To identify areas experiencing changes in the primary productivity, a non\sphinxhyphen{}parametric significance test will be performed to show the significant changes (p\sphinxhyphen{}value of 0.05). Positive significant trends in the vegetation indices will indicate potential improvement while a negative significant trend will indicate potential degradation.

\sphinxAtStartPar
The annual integrals of the vegetation indices are interpreted alongside historical precipitation data as a context. The climatic correction method that will be applied is the Rain Use Efficiency (RUE). The rain use efficiency is the ratio of annual NPP to annual precipitation. After the RUE is computed, linear regression and nonparametric significance testing will be applied to the RUE over time. Positive significance in RUE indicates improvement while negative significance will indicate potential degradation.


\subparagraph{State}
\label{\detokenize{Background/LD_indicators:state}}
\sphinxAtStartPar
The Productivity State indicator will be used to show recent changes in primary productivity compared to a baseline period. The indicator is computed from (NDVI, MSAVI2, SAVI) derived from medium resolution Landsat imagery following the steps outlined below:
\begin{enumerate}
\sphinxsetlistlabels{\arabic}{enumi}{enumii}{}{.}%
\item {} 
\sphinxAtStartPar
A baseline period (historical period for comparison to recent primary productivity) will be defined. (This will be left open for selection of different periods by the users).

\item {} 
\sphinxAtStartPar
A comparison period (recent years for which the state is being analysed) will be defined. (The definition of this period will also be left open for the users of the service)

\item {} 
\sphinxAtStartPar
The annual integrals of the selected vegetation index for the baseline period will be used to compute a frequency distribution at the selected pixel. That frequency distribution curve will then be used to classify the values to the 10th percentile(1 to 10).

\item {} 
\sphinxAtStartPar
The next step would involve computing the mean of the selected vegetation index for the baseline period, and to determine the percentile class it belongs to. The computed mean value for the baseline period is then assigned a number which corresponds to that percentile class if falls between 1 and 10.

\item {} 
\sphinxAtStartPar
The mean value of the selected index for the comparison period is the computed and percentile class to which it belongs to. It is determined and placed in a class corresponding to its percentile class.

\item {} 
\sphinxAtStartPar
The difference between the assigned class number for the comparison and the baseline period (comparison minus baseline) will be computed and thresholded to show the productivity state of the land.

\end{enumerate}


\subparagraph{Perfomance}
\label{\detokenize{Background/LD_indicators:perfomance}}
\sphinxAtStartPar
The Productivity Performance indicator will measure the local productivity relative to other similar vegetation types in similar ecological units. A combination of soil units (based on Soil Grids data at 250m resolution) and land cover (ESA CCI at 300m resolution) will be used to define the ecological units. The indicator will be computed as follows:
\begin{enumerate}
\sphinxsetlistlabels{\arabic}{enumi}{enumii}{}{.}%
\item {} 
\sphinxAtStartPar
The analysis period is defined, and time series data is used to compute mean value for the selected vegetation index at pixel level.

\item {} 
\sphinxAtStartPar
Similar ecological units are derived as the unique intersections of different land cover types and soil types.

\item {} 
\sphinxAtStartPar
For each ecological unit, the frequency distribution of the mean pixel values obtained in step 1 shall be computed. From the distribution the value representing the 90th percentile will be considered the maximum productivity for that unit.

\item {} 
\sphinxAtStartPar
The ratio of mean NDVI and maximum productivity (in each case compare the mean observed value to the maximum for its corresponding unit) is computed. If the computed ratio is less than 50 \%, the pixel shall be considered potentially degraded for this indicator.

\end{enumerate}

\begin{figure}[H]
\centering
\capstart

\noindent\sphinxincludegraphics[width=700\sphinxpxdimen,height=500\sphinxpxdimen]{{sdgmethodology}.png}
\caption{Sammury methodology for conputing Land Productivity}\label{\detokenize{Background/LD_indicators:id4}}\end{figure}


\paragraph{Landcover}
\label{\detokenize{Background/LD_indicators:landcover}}
\sphinxAtStartPar
Monitoring of Land Use and Land Cover Changes (LULCCs) at both regional and local scales presents a major opportunity for identifying areas threatened by land degradation where mitigation measures should be taken. Traditionally, LULCCs have been interpreted by distinguishing between two transformation types: conversion and modification.

\sphinxAtStartPar
To assess changes in land cover users need land cover maps covering the study
area for the baseline and target years. These maps need to be of acceptable
accuracy and created in such a way which allows for valid comparisons.
LDMS uses ESA CCI land cover maps as the default dataset, but local
maps can also be used. The indicator is computed as follows:
\begin{enumerate}
\sphinxsetlistlabels{\arabic}{enumi}{enumii}{}{.}%
\item {} 
\sphinxAtStartPar
Reclassify both land cover maps to the 7 land cover classes needed for
reporting to the UNCCD (forest, grassland, cropland, wetland, artificial
area, bare land and water).

\item {} 
\sphinxAtStartPar
Perform a land cover transition analysis to identify which pixels remained
in the same land cover class, and which ones changed.

\item {} 
\sphinxAtStartPar
Based on your local knowledge of the conditions in the study area and the
land degradation processed occurring there, use the table below to identify
which transitions correspond to degradation (\sphinxhyphen{} sign), improvement (+ sign),
or no change in terms of land condition (zero).

\end{enumerate}

\noindent{\hspace*{\fill}\sphinxincludegraphics{{lc_matrix}.PNG}\hspace*{\fill}}
\begin{enumerate}
\sphinxsetlistlabels{\arabic}{enumi}{enumii}{}{.}%
\setcounter{enumi}{3}
\item {} 
\sphinxAtStartPar
LDMS will combine the information from the land cover maps and the
table of degradation typologies by land cover transition to compute the land
cover sub\sphinxhyphen{}indicator.

\end{enumerate}

\begin{figure}[H]
\centering
\capstart

\noindent\sphinxincludegraphics[width=621\sphinxpxdimen,height=251\sphinxpxdimen]{{lulc}.png}
\caption{Sammary methodology for computing land cover change}\label{\detokenize{Background/LD_indicators:id5}}\end{figure}


\paragraph{Carbon\sphinxhyphen{}stocks}
\label{\detokenize{Background/LD_indicators:carbon-stocks}}
\sphinxAtStartPar
The third sub\sphinxhyphen{}indicator for monitoring land degradation as part of the SDG
process quantifies changes in soil organic carbon (SOC) over the reporting
period. Changes in SOC are particularly difficult to assess for several
reasons, some of them being the high spatial variability of soil properties,
the time and cost intensiveness of conducting representative soil surveys and
the lack of time series data on SOC for most regions of the world. To address
some of the limitations, a combined land cover/SOC method is used in
LDMS to estimate changes in SOC and identify potentially degraded
areas. The indicator is computed as follows:
\begin{enumerate}
\sphinxsetlistlabels{\arabic}{enumi}{enumii}{}{.}%
\item {} 
\sphinxAtStartPar
Determine the SOC reference values. LDMS uses SoilGrids 250m
carbon stocks for the first 30 cm of the soil profile as the reference
values for calculation (NOTE: SoilGrids uses information from a variety of
data sources and ranging from many years to produce this product, therefore
assigning a date for calculations purposes could cause inaccuracies in the
stock change calculations).

\item {} 
\sphinxAtStartPar
Reclassify the land cover maps to the 7 land cover classes needed for
reporting to the UNCCD (forest, grassland, cropland, wetland, artificial
area, bare land and water). Ideally annual land cover maps are preferred,
but at least land cover maps for the starting and end years are needed.

\item {} 
\sphinxAtStartPar
To estimate the changes in C stocks for the reporting period C conversion
coefficients for changes in land use, management and inputs are recommended
by the IPCC and the UNCCD. However, spatially explicit information on
management and C inputs is not available for most regions. As such, only
land use conversion coefficient can be applied for estimating changes in C
stocks (using land cover as a proxy for land use). The coefficients used
were the result of a literature review performed by the UNCCD and are
presented in the table below. Those coefficients represent the proportional
in C stocks after 20 years of land cover change.

\end{enumerate}

\noindent{\hspace*{\fill}\sphinxincludegraphics{{soc_coeff}.PNG}\hspace*{\fill}}

\sphinxAtStartPar
Changes in SOC are better studied for land cover transitions involving
agriculture, and for that reason there is a different set of coefficients for
each of the main global climatic regions: Temperate Dry (f = 0.80), Temperate
Moist (f = 0.69), Tropical Dry (f = 0.58), Tropical Moist (f = 0.48), and
Tropical Montane (f = 0.64).
\begin{enumerate}
\sphinxsetlistlabels{\arabic}{enumi}{enumii}{}{.}%
\setcounter{enumi}{3}
\item {} 
\sphinxAtStartPar
Compute relative different in SOC between the baseline and the target
period, areas which experienced a loss in SOC of 10\% of more during the
reporting period will be considered potentially degraded, and areas
experiencing a gain of 10\% or more as potentially improved.

\end{enumerate}

\noindent{\hspace*{\fill}\sphinxincludegraphics{{soc}.PNG}\hspace*{\fill}}


\subsubsection{Combining Productivity Indicators}
\label{\detokenize{Background/LD_indicators:combining-productivity-indicators}}
\sphinxAtStartPar
The three productivity sub\sphinxhyphen{}indicators are then combined as indicated in the
tables below. For SDG 15.3.1 reporting, the 3\sphinxhyphen{}class indicator is required, but
LDMS also produces a 5\sphinxhyphen{}class one which takes advantage of the
information provided by State to inform the type of degradation occurring in
the area.

\noindent{\hspace*{\fill}\sphinxincludegraphics{{lp_aggregation}.PNG}\hspace*{\fill}}


\subsection{Vegetation Loss/Gain hotspots}
\label{\detokenize{Background/LD_indicators:vegetation-loss-gain-hotspots}}
\sphinxAtStartPar
Land degradation hotsports (LDH) are produced via the analysis of time\sphinxhyphen{}series vegetation indices data and are used to characterize areas of different sizes, where the vegetation cover and the soil types are severely degraded.

\sphinxAtStartPar
Vegetation loss/gain hotspots will be calculated based on time series observation of selected suit of vegetation indices depending on the climatic zones and terrain morphology of the North African countries. The selected indices derived from Landsat data are as listed below:
\begin{quote}

\sphinxAtStartPar
\sphinxhyphen{}NDVI for humid zones, sub\sphinxhyphen{}humid and semi\sphinxhyphen{}arid zones
\sphinxhyphen{}MSAVI2 for arid and stepic zones
\sphinxhyphen{}SAVI for desert areas
\end{quote}


\subsection{Forest Change}
\label{\detokenize{Background/LD_indicators:forest-change}}

\subsubsection{Forest Gain/Loss}
\label{\detokenize{Background/LD_indicators:forest-gain-loss}}
\sphinxAtStartPar
The quantification of the forest gain/loss hotspots will be based on pre\sphinxhyphen{}existing high\sphinxhyphen{}resolution global maps derived from Hansen Global Forest change dataset that can be accessed using \sphinxhref{https://earthenginepartners.appspot.com/science-2013-global-forest}{Google Earth Engine API}.
\begin{quote}
\end{quote}

\sphinxAtStartPar
The maps are produced from time\sphinxhyphen{}series analysis of Landsat images characterizing forest extent and change over time.


\subsubsection{Forest Fires}
\label{\detokenize{Background/LD_indicators:forest-fires}}
\sphinxAtStartPar
Burnt areas and forest fires will be highlighted and mapped out form remotely sensed Landsat/Sentinel data using the Normalized Burn Ratio (NBR). NBR is designed to highlight burned areas and estimate burn severity. It uses near\sphinxhyphen{}infrared (NIR) and shortwave\sphinxhyphen{}infrared (SWIR) wavelengths. Before fire events, healthy vegetation has very high NIR reflectance and a low SWIR reflectance. In contrast, recently burned areas show low reflectance in the NIR and high reflectance in the SWIR band.

\sphinxAtStartPar
The NBR will be calculated for Landsat/Sentinel images before the fire (pre\sphinxhyphen{}fire NBR) and after the fire (post\sphinxhyphen{}fire NBR). The difference between the pre\sphinxhyphen{}fire NBR and the post\sphinxhyphen{}fire NBR referred to as delta NBR (dNBR) is computed to highlight the areas of forest disturbance by fire event.

\sphinxAtStartPar
Classification of the dNBR will be used for burn severity assessment, as areas with higher dNBR values indicate more severe damage whereas areas with negative dNBR values might show increased vegetation productivity. dNBR will be classified according to burn severity ranges proposed by the United States Geological Survey(USGS)

\begin{figure}[H]
\centering
\capstart

\noindent\sphinxincludegraphics[width=550\sphinxpxdimen,height=300\sphinxpxdimen]{{Forest_fires}.png}
\caption{Sammury methodology for conputing Burnt Areas}\label{\detokenize{Background/LD_indicators:id6}}\end{figure}

\sphinxstepscope


\chapter{SDG15.3.1 Indicator}
\label{\detokenize{Background/SDG_indicators:sdg15-3-1-indicator}}\label{\detokenize{Background/SDG_indicators::doc}}
\sphinxAtStartPar
As part of the Sustainable development Goals(SDGs), SDG 15 is to: “Protect, restore and promote sustainable use of terrestrial ecosystems, sustainably manage forest, combat desertification, and halt and reverse land degradation and halt biodiversity loss”

\sphinxAtStartPar
Target 15.3 aims to: “By 2030, combat desertification, restore degraded land and soil, including land affected by desertification, drought, floods, and strive to achieve land degradation\sphinxhyphen{}neutral world”

\sphinxAtStartPar
The indicator used to assess the progress of each SDG target is the 15.3.1 indicator: “Proportion of land that is degraded over total land area’’

\sphinxAtStartPar
The basic land degradation indicators include three main sub\sphinxhyphen{}indicators of the SDG target 15.3.1 (proportion of land that is degraded over the total land area). As the custodian agency of SDG 15.3, the United Nations Convention to Combat Desertification (UNCCD) has developed recommendations/Good practice guide on how to compute SDG indicator 15.3.1  from 3 sub\sphinxhyphen{}indicators:
\begin{itemize}
\item {} 
\sphinxAtStartPar
Vegetation productivity

\item {} 
\sphinxAtStartPar
Landcover

\item {} 
\sphinxAtStartPar
Soil Organic carbon

\end{itemize}

\begin{figure}[H]
\centering
\capstart

\noindent\sphinxincludegraphics[width=500\sphinxpxdimen,height=300\sphinxpxdimen]{{sdg}.png}
\caption{SDG 15.3.1 Indicators}\label{\detokenize{Background/SDG_indicators:id1}}\end{figure}


\section{SDG 15.3.1 Sub\sphinxhyphen{}indicators}
\label{\detokenize{Background/SDG_indicators:sdg-15-3-1-sub-indicators}}

\subsection{Productivity}
\label{\detokenize{Background/SDG_indicators:productivity}}
\sphinxAtStartPar
Land productivity is the biological productive capacity of the land (i.e. the ability to produce food, fibre and fuel that sustain life). For easy interpretation the annual mean vegetation indices values at the pixel level will be used to assess three measures of change (trajectory, state and performance) as summarized in the figure below and explained in the subsequent sub\sphinxhyphen{}sections:

\begin{figure}[H]
\centering
\capstart

\noindent\sphinxincludegraphics[width=700\sphinxpxdimen,height=500\sphinxpxdimen]{{sdgmethodology}.png}
\caption{Sammury methodology for conputing Land Productivity}\label{\detokenize{Background/SDG_indicators:id2}}\end{figure}

\sphinxAtStartPar
Trajectory;

\sphinxAtStartPar
The rate of change in primary productivity over time which will be computed using linear regression at the pixel level for various Landsat derived vegetation indices (NDVI, MSAVI2, SAVI). To identify areas experiencing changes in the primary productivity, a non\sphinxhyphen{}parametric significance test will be performed to show the significant changes (p\sphinxhyphen{}value of 0.05). Positive significant trends in the vegetation indices will indicate potential improvement while a negative significant trend will indicate potential degradation.

\sphinxAtStartPar
The annual integrals of the vegetation indices are interpreted alongside historical precipitation data as a context. The climatic correction method that will be applied is the Rain Use Efficiency (RUE). The rain use efficiency is the ratio of annual NPP to annual precipitation. After the RUE is computed, linear regression and nonparametric significance testing will be applied to the RUE over time. Positive significance in RUE indicates improvement while negative significance will indicate potential degradation.

\sphinxAtStartPar
State;

\sphinxAtStartPar
The Productivity State indicator will be used to show recent changes in primary productivity compared to a baseline period. The indicator is computed from (NDVI, MSAVI2, SAVI) derived from medium resolution Landsat imagery following the steps outlined below:
\begin{enumerate}
\sphinxsetlistlabels{\arabic}{enumi}{enumii}{}{.}%
\item {} 
\sphinxAtStartPar
A baseline period (historical period for comparison to recent primary productivity) will be defined. (This will be left open for selection of different periods by the users).

\item {} 
\sphinxAtStartPar
A comparison period (recent years for which the state is being analysed) will be defined. (The definition of this period will also be left open for the users of the service)

\item {} 
\sphinxAtStartPar
The annual integrals of the selected vegetation index for the baseline period will be used to compute a frequency distribution at the selected pixel. That frequency distribution curve will then be used to classify the values to the 10th percentile(1 to 10).

\item {} 
\sphinxAtStartPar
The next step would involve computing the mean of the selected vegetation index for the baseline period, and to determine the percentile class it belongs to. The computed mean value for the baseline period is then assigned a number which corresponds to that percentile class if falls between 1 and 10.

\item {} 
\sphinxAtStartPar
The mean value of the selected index for the comparison period is the computed and percentile class to which it belongs to. It is determined and placed in a class corresponding to its percentile class.

\item {} 
\sphinxAtStartPar
The difference between the assigned class number for the comparison and the baseline period (comparison minus baseline) will be computed and thresholded to show the productivity state of the land.

\end{enumerate}

\sphinxAtStartPar
Perfomance;

\sphinxAtStartPar
The Productivity Performance indicator will measure the local productivity relative to other similar vegetation types in similar ecological units. A combination of soil units (based on Soil Grids data at 250m resolution) and land cover (ESA CCI at 300m resolution) will be used to define the ecological units. The indicator will be computed as follows:
\begin{enumerate}
\sphinxsetlistlabels{\arabic}{enumi}{enumii}{}{.}%
\item {} 
\sphinxAtStartPar
The analysis period is defined, and time series data is used to compute mean value for the selected vegetation index at pixel level.

\item {} 
\sphinxAtStartPar
Similar ecological units are derived as the unique intersections of different land cover types and soil types.

\item {} 
\sphinxAtStartPar
For each ecological unit, the frequency distribution of the mean pixel values obtained in step 1 shall be computed. From the distribution the value representing the 90th percentile will be considered the maximum productivity for that unit.

\item {} 
\sphinxAtStartPar
The ratio of mean NDVI and maximum productivity (in each case compare the mean observed value to the maximum for its corresponding unit) is computed. If the computed ratio is less than 50 \%, the pixel shall be considered potentially degraded for this indicator.

\end{enumerate}

\begin{figure}[H]
\centering
\capstart

\noindent\sphinxincludegraphics[width=700\sphinxpxdimen,height=500\sphinxpxdimen]{{sdgmethodology}.png}
\caption{Sammury methodology for conputing Land Productivity}\label{\detokenize{Background/SDG_indicators:id3}}\end{figure}


\subsection{Combining Productivity Indicators}
\label{\detokenize{Background/SDG_indicators:combining-productivity-indicators}}
\sphinxAtStartPar
The three productivity sub\sphinxhyphen{}indicators are then combined as indicated in the
tables below. For SDG 15.3.1 reporting, the 3\sphinxhyphen{}class indicator is required, but
LDMS also produces a 5\sphinxhyphen{}class one which takes advantage of the
information provided by State to inform the type of degradation occurring in
the area.

\noindent{\hspace*{\fill}\sphinxincludegraphics{{lp_aggregation}.PNG}\hspace*{\fill}}


\section{Landcover}
\label{\detokenize{Background/SDG_indicators:landcover}}
\sphinxAtStartPar
Monitoring of Land Use and Land Cover Changes (LULCCs) at both regional and local scales presents a major opportunity for identifying areas threatened by land degradation where mitigation measures should be taken. Traditionally, LULCCs have been interpreted by distinguishing between two transformation types: conversion and modification.

\sphinxAtStartPar
To assess changes in land cover users need land cover maps covering the study
area for the baseline and target years. These maps need to be of acceptable
accuracy and created in such a way which allows for valid comparisons.
LDMS uses ESA CCI land cover maps as the default dataset, but local
maps can also be used. The indicator is computed as follows:
\begin{enumerate}
\sphinxsetlistlabels{\arabic}{enumi}{enumii}{}{.}%
\item {} 
\sphinxAtStartPar
Reclassify both land cover maps to the 7 land cover classes needed for
reporting to the UNCCD (forest, grassland, cropland, wetland, artificial
area, bare land and water).

\item {} 
\sphinxAtStartPar
Perform a land cover transition analysis to identify which pixels remained
in the same land cover class, and which ones changed.

\item {} 
\sphinxAtStartPar
Based on your local knowledge of the conditions in the study area and the
land degradation processed occurring there, use the table below to identify
which transitions correspond to degradation (\sphinxhyphen{} sign), improvement (+ sign),
or no change in terms of land condition (zero).

\end{enumerate}

\noindent{\hspace*{\fill}\sphinxincludegraphics{{lc_matrix}.PNG}\hspace*{\fill}}
\begin{enumerate}
\sphinxsetlistlabels{\arabic}{enumi}{enumii}{}{.}%
\item {} 
\sphinxAtStartPar
LDMS will combine the information from the land cover maps and the
table of degradation typologies by land cover transition to compute the land
cover sub\sphinxhyphen{}indicator.

\end{enumerate}

\begin{figure}[H]
\centering
\capstart

\noindent\sphinxincludegraphics[width=621\sphinxpxdimen,height=251\sphinxpxdimen]{{lulc}.png}
\caption{Sammary methodology for computing land cover change}\label{\detokenize{Background/SDG_indicators:id4}}\end{figure}


\section{Carbon\sphinxhyphen{}stocks}
\label{\detokenize{Background/SDG_indicators:carbon-stocks}}
\sphinxAtStartPar
The third sub\sphinxhyphen{}indicator for monitoring land degradation as part of the SDG
process quantifies changes in soil organic carbon (SOC) over the reporting
period. Changes in SOC are particularly difficult to assess for several
reasons, some of them being the high spatial variability of soil properties,
the time and cost intensiveness of conducting representative soil surveys and
the lack of time series data on SOC for most regions of the world. To address
some of the limitations, a combined land cover/SOC method is used in
LDMS to estimate changes in SOC and identify potentially degraded
areas. The indicator is computed as follows:
\begin{enumerate}
\sphinxsetlistlabels{\arabic}{enumi}{enumii}{}{.}%
\item {} 
\sphinxAtStartPar
Determine the SOC reference values. LDMS uses SoilGrids 250m
carbon stocks for the first 30 cm of the soil profile as the reference
values for calculation (NOTE: SoilGrids uses information from a variety of
data sources and ranging from many years to produce this product, therefore
assigning a date for calculations purposes could cause inaccuracies in the
stock change calculations).

\item {} 
\sphinxAtStartPar
Reclassify the land cover maps to the 7 land cover classes needed for
reporting to the UNCCD (forest, grassland, cropland, wetland, artificial
area, bare land and water). Ideally annual land cover maps are preferred,
but at least land cover maps for the starting and end years are needed.

\item {} 
\sphinxAtStartPar
To estimate the changes in C stocks for the reporting period C conversion
coefficients for changes in land use, management and inputs are recommended
by the IPCC and the UNCCD. However, spatially explicit information on
management and C inputs is not available for most regions. As such, only
land use conversion coefficient can be applied for estimating changes in C
stocks (using land cover as a proxy for land use). The coefficients used
were the result of a literature review performed by the UNCCD and are
presented in the table below. Those coefficients represent the proportional
in C stocks after 20 years of land cover change.

\end{enumerate}

\noindent{\hspace*{\fill}\sphinxincludegraphics{{soc_coeff}.PNG}\hspace*{\fill}}

\sphinxAtStartPar
Changes in SOC are better studied for land cover transitions involving
agriculture, and for that reason there is a different set of coefficients for
each of the main global climatic regions: Temperate Dry (f = 0.80), Temperate
Moist (f = 0.69), Tropical Dry (f = 0.58), Tropical Moist (f = 0.48), and
Tropical Montane (f = 0.64).
\begin{enumerate}
\sphinxsetlistlabels{\arabic}{enumi}{enumii}{}{.}%
\setcounter{enumi}{3}
\item {} 
\sphinxAtStartPar
Compute relative different in SOC between the baseline and the target
period, areas which experienced a loss in SOC of 10\% of more during the
reporting period will be considered potentially degraded, and areas
experiencing a gain of 10\% or more as potentially improved.

\end{enumerate}

\noindent{\hspace*{\fill}\sphinxincludegraphics{{soc}.PNG}\hspace*{\fill}}
\phantomsection\label{\detokenize{Background/SDG_indicators:indicator-land-cover}}
\sphinxstepscope


\chapter{Vegetation Loss/Gain hotspots}
\label{\detokenize{Background/vegetationloss:vegetation-loss-gain-hotspots}}\label{\detokenize{Background/vegetationloss::doc}}
\sphinxAtStartPar
Land degradation hotsports (LDH) are produced via the analysis of time\sphinxhyphen{}series vegetation indices data and are used to characterize areas of different sizes, where the vegetation cover and the soil types are severely degraded.

\sphinxAtStartPar
Vegetation loss/gain hotspots will be calculated based on time series observation of selected suit of vegetation indices depending on the climatic zones and terrain morphology of the North African countries. The selected indices derived from Landsat data are as listed below:
\begin{quote}

\sphinxAtStartPar
\sphinxhyphen{}NDVI for humid zones, sub\sphinxhyphen{}humid and semi\sphinxhyphen{}arid zones
\sphinxhyphen{}MSAVI2 for arid and stepic zones
\sphinxhyphen{}SAVI for desert areas
\end{quote}

\sphinxstepscope


\chapter{Forest Change}
\label{\detokenize{Background/Forest_change:forest-change}}\label{\detokenize{Background/Forest_change::doc}}

\section{Forest Gain/Loss}
\label{\detokenize{Background/Forest_change:forest-gain-loss}}
\sphinxAtStartPar
The quantification of the forest gain/loss hotspots will be based on pre\sphinxhyphen{}existing high\sphinxhyphen{}resolution global maps derived from Hansen Global Forest change dataset that can be accessed using \sphinxhref{https://earthenginepartners.appspot.com/science-2013-global-forest}{Google Earth Engine API}.
\begin{quote}
\end{quote}

\sphinxAtStartPar
The maps are produced from time\sphinxhyphen{}series analysis of Landsat images characterizing forest extent and change over time.


\section{Forest Carbon Emission}
\label{\detokenize{Background/Forest_change:forest-carbon-emission}}

\section{Forest Fire Risk}
\label{\detokenize{Background/Forest_change:forest-fire-risk}}

\section{Forest Fires}
\label{\detokenize{Background/Forest_change:forest-fires}}
\sphinxAtStartPar
Burnt areas and forest fires will be highlighted and mapped out form remotely sensed Landsat/Sentinel data using the Normalized Burn Ratio (NBR). NBR is designed to highlight burned areas and estimate burn severity. It uses near\sphinxhyphen{}infrared (NIR) and shortwave\sphinxhyphen{}infrared (SWIR) wavelengths. Before fire events, healthy vegetation has very high NIR reflectance and a low SWIR reflectance. In contrast, recently burned areas show low reflectance in the NIR and high reflectance in the SWIR band.

\sphinxAtStartPar
The NBR will be calculated for Landsat/Sentinel images before the fire (pre\sphinxhyphen{}fire NBR) and after the fire (post\sphinxhyphen{}fire NBR). The difference between the pre\sphinxhyphen{}fire NBR and the post\sphinxhyphen{}fire NBR referred to as delta NBR (dNBR) is computed to highlight the areas of forest disturbance by fire event.

\sphinxAtStartPar
Classification of the dNBR will be used for burn severity assessment, as areas with higher dNBR values indicate more severe damage whereas areas with negative dNBR values might show increased vegetation productivity. dNBR will be classified according to burn severity ranges proposed by the United States Geological Survey(USGS)

\begin{figure}[H]
\centering
\capstart

\noindent\sphinxincludegraphics[width=700\sphinxpxdimen,height=500\sphinxpxdimen]{{Forest_fires}.png}
\caption{Sammury methodology for conputing Burnt Areas}\label{\detokenize{Background/Forest_change:id1}}\end{figure}

\sphinxstepscope


\chapter{Serivice Guide}
\label{\detokenize{Service/Service_Guide:serivice-guide}}\label{\detokenize{Service/Service_Guide::doc}}

\section{MISLAND Site Tour}
\label{\detokenize{Service/Service_Guide:misland-site-tour}}



\section{Registration and Log in}
\label{\detokenize{Service/Service_Guide:registration-and-log-in}}
\sphinxAtStartPar
New users to the service will be required to register a new account to use the service. Registering a new account is a simple two step process
\begin{enumerate}
\sphinxsetlistlabels{\arabic}{enumi}{enumii}{}{.}%
\item {} 
\sphinxAtStartPar
Click on the log\sphinxhyphen{}in icon on the right hand side of the navigation\sphinxhyphen{}bar

\end{enumerate}

\begin{figure}[H]
\centering
\capstart

\noindent\sphinxincludegraphics[width=516\sphinxpxdimen,height=305\sphinxpxdimen]{{login1}.png}
\caption{Finding the log in option}\label{\detokenize{Service/Service_Guide:id1}}\end{figure}
\begin{enumerate}
\sphinxsetlistlabels{\arabic}{enumi}{enumii}{}{.}%
\setcounter{enumi}{1}
\item {} 
\sphinxAtStartPar
Choose the ‘Not a user? Sing up’ option on the log in menu that pops up

\end{enumerate}

\begin{figure}[H]
\centering
\capstart

\noindent\sphinxincludegraphics[width=582\sphinxpxdimen,height=545\sphinxpxdimen]{{register}.png}
\caption{new user registration}\label{\detokenize{Service/Service_Guide:id2}}\end{figure}

\sphinxAtStartPar
Registered users can proceed to log in with their Email and password.


\section{User profile and custom uploads}
\label{\detokenize{Service/Service_Guide:user-profile-and-custom-uploads}}
\sphinxAtStartPar
Once loged in, users can select their profile to edit their information and for custom uploads. For custom area uploads, follow the simple steps as outlined below:
\begin{enumerate}
\sphinxsetlistlabels{\arabic}{enumi}{enumii}{}{.}%
\item {} 
\sphinxAtStartPar
Click on the user\sphinxhyphen{}name that appears on the navingation bar items

\end{enumerate}

\begin{figure}[H]
\centering
\capstart

\noindent\sphinxincludegraphics[width=570\sphinxpxdimen,height=591\sphinxpxdimen]{{profile}.png}
\caption{User profile and options}\label{\detokenize{Service/Service_Guide:id3}}\end{figure}

\sphinxstepscope


\chapter{SDG 15.3.1 indicator}
\label{\detokenize{Service/Calculate_SDG:sdg-15-3-1-indicator}}\label{\detokenize{Service/Calculate_SDG::doc}}

\section{Compute SDG 15.3.1 Sub\sphinxhyphen{}indicators}
\label{\detokenize{Service/Calculate_SDG:compute-sdg-15-3-1-sub-indicators}}
\sphinxAtStartPar
On the Service Menu bar, users can select the \sphinxincludegraphics{{sdgicon}.png} option which appears as the first item in the menu to compute SDG 15.3.1 indicator and its sub\sphinxhyphen{}indicators forllowing the steps described below:


\subsection{Computing Land Productivity}
\label{\detokenize{Service/Calculate_SDG:computing-land-productivity}}
\sphinxAtStartPar
Land productivity is computed form vegetation indices using three measures of change i.e trajectory, state and performace. Any of the three sub\sphinxhyphen{}indicators measures of change as well as the productivity can be combuted as illustrated below
\begin{enumerate}
\sphinxsetlistlabels{\arabic}{enumi}{enumii}{}{.}%
\item {} 
\sphinxAtStartPar
On the indicator menu bar that appears below the area selection panel click on the \sphinxincludegraphics{{landproductivitybutton}.png} option. This should pop\sphinxhyphen{}up a selection panel as in the diagram below:

\end{enumerate}

\begin{figure}[H]
\centering
\capstart

\noindent\sphinxincludegraphics[width=697\sphinxpxdimen,height=482\sphinxpxdimen]{{productivity1}.png}
\caption{selecting the land productivity option}\label{\detokenize{Service/Calculate_SDG:id1}}\end{figure}
\begin{enumerate}
\sphinxsetlistlabels{\arabic}{enumi}{enumii}{}{.}%
\setcounter{enumi}{1}
\item {} 
\sphinxAtStartPar
Under the productivity indicator option on the pop\sphinxhyphen{}up menu, users can select either of the three land productivity sub\sphinxhyphen{}indicators i.e. State, performance, and trajectory or the final aggregated land productivity for their selected area of interest.

\end{enumerate}

\begin{figure}[H]
\centering
\capstart

\noindent\sphinxincludegraphics[width=702\sphinxpxdimen,height=489\sphinxpxdimen]{{productivity2}.png}
\caption{Computing land productivity by combining state, trajectory and performance}\label{\detokenize{Service/Calculate_SDG:id2}}\end{figure}
\begin{enumerate}
\sphinxsetlistlabels{\arabic}{enumi}{enumii}{}{.}%
\setcounter{enumi}{2}
\item {} 
\sphinxAtStartPar
Complete the selection by selecting the data source and the reporting period.

\end{enumerate}

\begin{sphinxadmonition}{note}{Note:}
\sphinxAtStartPar
MISLAND allows users to asses vegetation using high resolution Landsat derived vegetation indices. If the selection of the dataset is landsat the option to specify the vegetation index .i.e NDVI, MSAVI or SAVI will appear under the \sphinxincludegraphics{{advanceparams}.png} options.
\begin{quote}

\begin{figure}[H]
\centering
\capstart

\noindent\sphinxincludegraphics[width=701\sphinxpxdimen,height=480\sphinxpxdimen]{{productivity3}.png}
\caption{Vegetation index selection}\label{\detokenize{Service/Calculate_SDG:id3}}\end{figure}
\end{quote}
\end{sphinxadmonition}
\begin{enumerate}
\sphinxsetlistlabels{\arabic}{enumi}{enumii}{}{.}%
\setcounter{enumi}{3}
\item {} 
\sphinxAtStartPar
Once the selection of datasets and reporting period is complete click on the \sphinxincludegraphics{{submit}.png} button at the pottom of the selection pop\sphinxhyphen{}up window to compute the selected indicator. The map and statistics should be dispalyed as shown below.

\end{enumerate}

\begin{figure}[H]
\centering
\capstart

\noindent\sphinxincludegraphics[width=650\sphinxpxdimen,height=312\sphinxpxdimen]{{productivity4}.png}
\caption{Land productivity output}\label{\detokenize{Service/Calculate_SDG:id4}}\end{figure}


\subsection{Computing Landcover Change}
\label{\detokenize{Service/Calculate_SDG:computing-landcover-change}}
\sphinxAtStartPar
MISLAND allows the user to view land cover state for a particular year or to compute landcover changes between two years. The landcover change
can be accessed from the \sphinxincludegraphics{{landcoverbutton}.png} option under the SDG indicator menu as described in the steps below
\begin{enumerate}
\sphinxsetlistlabels{\arabic}{enumi}{enumii}{}{.}%
\item {} 
\sphinxAtStartPar
Select \sphinxincludegraphics{{sdgicon}.png} option on the services menu\sphinxhyphen{}bar and clic on the \sphinxincludegraphics{{landcoverbutton}.png} option. This should pop up a selection panel as the one shown below

\end{enumerate}

\begin{figure}[H]
\centering
\capstart

\noindent\sphinxincludegraphics[width=682\sphinxpxdimen,height=482\sphinxpxdimen]{{lulc1}.png}
\caption{Selecting the Land cover change under SDG 15.3.1 sub\sphinxhyphen{}indicators}\label{\detokenize{Service/Calculate_SDG:id5}}\end{figure}
\begin{enumerate}
\sphinxsetlistlabels{\arabic}{enumi}{enumii}{}{.}%
\setcounter{enumi}{1}
\item {} 
\sphinxAtStartPar
To view the land cover data for a particular year, select ‘Land Cover’ option under the \sphinxtitleref{LANDCOVER ANALYSIS OPTIONS} folled by a selection of the landcover data source and the year as shown below

\end{enumerate}

\begin{figure}[H]
\centering
\capstart

\noindent\sphinxincludegraphics[width=695\sphinxpxdimen,height=482\sphinxpxdimen]{{lulc2}.png}
\caption{Viewing the Land cover data for a particular year}\label{\detokenize{Service/Calculate_SDG:id6}}\end{figure}

\sphinxAtStartPar
Click on the \sphinxincludegraphics{{submit}.png} button and the Land cover map for the chosen year and the summary statistics will be displayed on the map panel and the summary panel as shown below

\begin{figure}[H]
\centering
\capstart

\noindent\sphinxincludegraphics[width=665\sphinxpxdimen,height=385\sphinxpxdimen]{{lulc3}.png}
\caption{Viewing the Land cover data for a particular year}\label{\detokenize{Service/Calculate_SDG:id7}}\end{figure}
\begin{enumerate}
\sphinxsetlistlabels{\arabic}{enumi}{enumii}{}{.}%
\setcounter{enumi}{2}
\item {} 
\sphinxAtStartPar
To compute landcover change, selec the ‘Landcover change’ option option under the \sphinxtitleref{LANDCOVER ANALYSIS OPTIONS}. select the data landcover data source and the baseline and reporting year for comparison as shown

\end{enumerate}

\begin{figure}[H]
\centering
\capstart

\noindent\sphinxincludegraphics[width=697\sphinxpxdimen,height=482\sphinxpxdimen]{{lulc4}.png}
\caption{Selecting the Landcover change option}\label{\detokenize{Service/Calculate_SDG:id8}}\end{figure}

\sphinxAtStartPar
The results will be displayed on the map panel and the summary statistics panel as shown below

\begin{figure}[H]
\centering
\capstart

\noindent\sphinxincludegraphics[width=650\sphinxpxdimen,height=312\sphinxpxdimen]{{carbonstocks2}.png}
\caption{landcover change map and statistics}\label{\detokenize{Service/Calculate_SDG:id9}}\end{figure}


\subsection{Carbon Stocks}
\label{\detokenize{Service/Calculate_SDG:carbon-stocks}}
\sphinxAtStartPar
To compute changes in carbon stocks,
\begin{enumerate}
\sphinxsetlistlabels{\arabic}{enumi}{enumii}{}{.}%
\item {} 
\sphinxAtStartPar
Select \sphinxincludegraphics{{sdgicon}.png} option on the services menu\sphinxhyphen{}bar. Choose \sphinxincludegraphics{{carbonstockbutton}.png} option and under the SDG indicator menu\sphinxhyphen{}bar. This should pop\sphinxhyphen{}up a dialog as the one shown below

\end{enumerate}

\begin{figure}[H]
\centering
\capstart

\noindent\sphinxincludegraphics[width=366\sphinxpxdimen,height=335\sphinxpxdimen]{{carbonstocks}.png}
\caption{Selecting the Carbon stock change SDG 15.3.1 sub\sphinxhyphen{}indicator}\label{\detokenize{Service/Calculate_SDG:id10}}\end{figure}

\sphinxAtStartPar
select the data source and the reporting period and click on the \sphinxincludegraphics{{submit}.png} button to view the carbon stock change for the selected reporting period

\begin{figure}[H]
\centering
\capstart

\noindent\sphinxincludegraphics[width=650\sphinxpxdimen,height=312\sphinxpxdimen]{{carbonstocks2}.png}
\caption{Carbon stock change map and statistics}\label{\detokenize{Service/Calculate_SDG:id11}}\end{figure}


\section{Compute SDG 15.3.1 Indicator}
\label{\detokenize{Service/Calculate_SDG:compute-sdg-15-3-1-indicator}}
\sphinxAtStartPar
The SDG 15.3.1 Indicator combines the three sub\sphinxhyphen{}indicators .i.e changes in land productivity, landcover and carbon stocks discussed previously to asses the land degradation status of the selected area and period. The one\sphinxhyphen{}out, all\sphinxhyphen{}out (1OAO) approach is used to combine the results from the three sub\sphinxhyphen{}indicators, to assess degradation status for each monitoring period at the Indicator level. Within the study region, degradation is considered to have occurred if degradation is reported in any one of the sub\sphinxhyphen{}indicators.

\sphinxAtStartPar
To compute the SDG 15.3.1 indicator, follow these simple steps,
\begin{enumerate}
\sphinxsetlistlabels{\arabic}{enumi}{enumii}{}{.}%
\item {} 
\sphinxAtStartPar
Select the \sphinxincludegraphics{{sdgicon}.png} service, on the services menu\sphinxhyphen{}bar and click on the \sphinxincludegraphics{{sdgoption}.png} option. This should show a pop\sphinxhyphen{}up as the one shown below

\end{enumerate}

\begin{figure}[H]
\centering
\capstart

\noindent\sphinxincludegraphics[width=692\sphinxpxdimen,height=482\sphinxpxdimen]{{sdg1}.png}
\caption{SDG 15.3.1 indicator}\label{\detokenize{Service/Calculate_SDG:id12}}\end{figure}
\begin{enumerate}
\sphinxsetlistlabels{\arabic}{enumi}{enumii}{}{.}%
\setcounter{enumi}{1}
\item {} 
\sphinxAtStartPar
On the dialog that pops up, select the datasource and the reporting period. and click on the \sphinxincludegraphics{{submit}.png} button to get the results

\end{enumerate}

\begin{sphinxadmonition}{note}{Note:}
\sphinxAtStartPar
Clicking on the \sphinxincludegraphics{{advanceparams}.png} option provides more options to select the vegetation index of choise, the ecological unit dataset and the soil organic carbon reference raster as shown below.
\begin{quote}

\begin{figure}[H]
\centering
\capstart

\noindent\sphinxincludegraphics[width=693\sphinxpxdimen,height=482\sphinxpxdimen]{{sdg2}.png}
\caption{Setting advanced options for the SDG 15.3.1 indicator}\label{\detokenize{Service/Calculate_SDG:id13}}\end{figure}
\end{quote}
\end{sphinxadmonition}

\sphinxAtStartPar
The map and computed statistics will be displayed on the map panel and summary pannel respectively.

\begin{figure}[H]
\centering

\noindent\sphinxincludegraphics[width=770\sphinxpxdimen,height=393\sphinxpxdimen]{{sdg4}.png}
\end{figure}

\sphinxstepscope


\chapter{Calculate Vegetation Loss/Gain indicators}
\label{\detokenize{Service/Calculate_vegetationloss:calculate-vegetation-loss-gain-indicators}}\label{\detokenize{Service/Calculate_vegetationloss::doc}}
\sphinxAtStartPar
To compute vegetation loss/gain on the service platform,
\begin{enumerate}
\sphinxsetlistlabels{\arabic}{enumi}{enumii}{}{.}%
\item {} 
\sphinxAtStartPar
On the services menu, select the \sphinxincludegraphics{{veglossicon}.png} option

\end{enumerate}

\begin{figure}[H]
\centering
\capstart

\noindent\sphinxincludegraphics[width=705\sphinxpxdimen,height=374\sphinxpxdimen]{{vegetation_gain_loss}.png}
\caption{Selecting the vegetation Loss/Gain Service}\label{\detokenize{Service/Calculate_vegetationloss:id1}}\end{figure}
\begin{enumerate}
\sphinxsetlistlabels{\arabic}{enumi}{enumii}{}{.}%
\setcounter{enumi}{1}
\item {} 
\sphinxAtStartPar
Click on the \sphinxincludegraphics{{cog}.png} icon to open the layer settings dialog and select the data source and reporting year as shown below.

\end{enumerate}

\begin{figure}[H]
\centering
\capstart

\noindent\sphinxincludegraphics[width=711\sphinxpxdimen,height=234\sphinxpxdimen]{{vegetation_gain_loss3}.png}
\caption{Vegetation gain/loss outputs}\label{\detokenize{Service/Calculate_vegetationloss:id2}}\end{figure}

\sphinxAtStartPar
To compute vegetation indices using Landsat derived vegetation indices(NDVI, MSAVI, SAVI),
\begin{enumerate}
\sphinxsetlistlabels{\arabic}{enumi}{enumii}{}{.}%
\item {} 
\sphinxAtStartPar
On the Vegetation loss/gain dialog, select Landsat under the SELECT DATA SOURCE dropdown and click on the \sphinxincludegraphics{{advancedparameters}.png} options to access the list of indices.

\end{enumerate}

\begin{figure}[H]
\centering
\capstart

\noindent\sphinxincludegraphics[width=709\sphinxpxdimen,height=381\sphinxpxdimen]{{landsat}.png}
\caption{Selecting the Landsat\sphinxhyphen{}derived vegetation index option}\label{\detokenize{Service/Calculate_vegetationloss:id3}}\end{figure}
\begin{enumerate}
\sphinxsetlistlabels{\arabic}{enumi}{enumii}{}{.}%
\setcounter{enumi}{1}
\item {} 
\sphinxAtStartPar
select the vegetation index form the SELECT VEGEATATION INDEX dropdown that is revealed. Select the reporting period before clicking on the \sphinxincludegraphics{{submit}.png} button.

\end{enumerate}

\begin{figure}[H]
\centering
\capstart

\noindent\sphinxincludegraphics[width=349\sphinxpxdimen,height=372\sphinxpxdimen]{{vegindex}.png}
\caption{Choosing the vegetation index to compute}\label{\detokenize{Service/Calculate_vegetationloss:id4}}\end{figure}

\sphinxAtStartPar
The map and computed statistics will be displayed on the map panel and summary pannel respectively.

\begin{figure}[H]
\centering
\capstart

\noindent\sphinxincludegraphics[width=712\sphinxpxdimen,height=394\sphinxpxdimen]{{landsat_vegetation_loss}.png}
\caption{Landsat derived vegetation loss and gain output}\label{\detokenize{Service/Calculate_vegetationloss:id5}}\end{figure}

\sphinxstepscope


\chapter{Calculate Forest Change}
\label{\detokenize{Service/Calculate_Forestloss:calculate-forest-change}}\label{\detokenize{Service/Calculate_Forestloss::doc}}

\section{Computing Forest Loss}
\label{\detokenize{Service/Calculate_Forestloss:computing-forest-loss}}
\begin{sphinxadmonition}{note}{Note:}
\sphinxAtStartPar
The current release of the MISLAND\sphinxhyphen{}North Africa uses the High resolution Hansen Global forest Change data to compute forest loss for selected area and year.
\end{sphinxadmonition}

\sphinxAtStartPar
To compute forest loss using the Hansen Global forest change dataset;
\begin{enumerate}
\sphinxsetlistlabels{\arabic}{enumi}{enumii}{}{.}%
\item {} 
\sphinxAtStartPar
On the top left conner of the Map pannel, click on the \sphinxincludegraphics{{drowicon}.png} tool to toggle the drowing tools. Once the drawing tools are revealed, click on the \sphinxincludegraphics{{drawpolygonicon}.png} tool to start drowing a custom area on the map where you wish to compute the forest loss

\end{enumerate}

\begin{figure}[H]
\centering
\capstart

\noindent\sphinxincludegraphics[width=384\sphinxpxdimen,height=390\sphinxpxdimen]{{draw}.png}
\caption{Draw a polygon tool}\label{\detokenize{Service/Calculate_Forestloss:id1}}\end{figure}

\begin{figure}[H]
\centering
\capstart

\noindent\sphinxincludegraphics[width=431\sphinxpxdimen,height=438\sphinxpxdimen]{{drawarea}.png}
\caption{Draw a polygon tool}\label{\detokenize{Service/Calculate_Forestloss:id2}}\end{figure}
\begin{enumerate}
\sphinxsetlistlabels{\arabic}{enumi}{enumii}{}{.}%
\setcounter{enumi}{1}
\item {} 
\sphinxAtStartPar
Select \sphinxincludegraphics{{forestlossicon}.png} and click on the layer settings icon \sphinxincludegraphics{{cog}.png} . On the layer settings dialog, under the ‘SELECT DATA SOURCE’ options, choose ‘Hansen’ and select the year you wish to compute the forest loss.

\end{enumerate}

\begin{figure}[H]
\centering
\capstart

\noindent\sphinxincludegraphics[width=683\sphinxpxdimen,height=482\sphinxpxdimen]{{Hansen}.png}
\caption{Selecting the Hansen Forest loss data}\label{\detokenize{Service/Calculate_Forestloss:id3}}\end{figure}

\sphinxAtStartPar
On clicking \sphinxincludegraphics{{submit}.png} The map and computed statistics will be displayed on the map panel and summary pannel respectively.

\begin{figure}[H]
\centering
\capstart

\noindent\sphinxincludegraphics[width=705\sphinxpxdimen,height=325\sphinxpxdimen]{{hansenoutput}.png}
\caption{Foret change outputs}\label{\detokenize{Service/Calculate_Forestloss:id4}}\end{figure}


\section{Forest Fires Assesment}
\label{\detokenize{Service/Calculate_Forestloss:forest-fires-assesment}}\begin{enumerate}
\sphinxsetlistlabels{\arabic}{enumi}{enumii}{}{.}%
\item {} 
\sphinxAtStartPar
Select the \sphinxincludegraphics{{forestchangeicon}.png} option from the service menu. Under the \sphinxincludegraphics{{forestfireassesmenticon}.png} click on the \sphinxincludegraphics{{cog}.png} icon to toggle the layer settings as shown below.

\end{enumerate}

\begin{figure}[H]
\centering
\capstart

\noindent\sphinxincludegraphics[width=354\sphinxpxdimen,height=482\sphinxpxdimen]{{forestfireselection}.png}
\caption{Selecting the Forest\sphinxhyphen{}fires option from the service menu.}\label{\detokenize{Service/Calculate_Forestloss:id5}}\end{figure}
\begin{enumerate}
\sphinxsetlistlabels{\arabic}{enumi}{enumii}{}{.}%
\setcounter{enumi}{1}
\item {} 
\sphinxAtStartPar
On the output layer options, select the pre and post fire dates using the calender

\end{enumerate}

\begin{figure}[H]
\centering
\capstart

\noindent\sphinxincludegraphics[width=384\sphinxpxdimen,height=405\sphinxpxdimen]{{calender}.png}
\caption{Selecting the date from the calender tool.}\label{\detokenize{Service/Calculate_Forestloss:id6}}\end{figure}

\begin{figure}[H]
\centering
\capstart

\noindent\sphinxincludegraphics[width=308\sphinxpxdimen,height=178\sphinxpxdimen]{{firedates}.png}
\caption{Pre\sphinxhyphen{}fire and Post\sphinxhyphen{}fire dates}\label{\detokenize{Service/Calculate_Forestloss:id7}}\end{figure}
\begin{enumerate}
\sphinxsetlistlabels{\arabic}{enumi}{enumii}{}{.}%
\setcounter{enumi}{2}
\item {} 
\sphinxAtStartPar
Choose the platform to use to compute the burnt area

\end{enumerate}

\begin{figure}[H]
\centering
\capstart

\noindent\sphinxincludegraphics[width=326\sphinxpxdimen,height=336\sphinxpxdimen]{{platform}.png}
\caption{Choosing the Platform/Sensor for computing forest fires.}\label{\detokenize{Service/Calculate_Forestloss:id8}}\end{figure}

\sphinxAtStartPar
The output showing the extent and severity of the fire will be as shown below

\begin{figure}[H]
\centering
\capstart

\noindent\sphinxincludegraphics[width=793\sphinxpxdimen,height=350\sphinxpxdimen]{{fireoutput}.png}
\caption{Forest\sphinxhyphen{}fire Output.}\label{\detokenize{Service/Calculate_Forestloss:id9}}\end{figure}

\sphinxstepscope


\chapter{Calculating Sensitivity to Desertification (MEDALUS)}
\label{\detokenize{Service/Calculate_Medalus:calculating-sensitivity-to-desertification-medalus}}\label{\detokenize{Service/Calculate_Medalus::doc}}
\sphinxAtStartPar
Land degradation and desertification (LDD) analysis is done using the MEDALUS\textendash{}(Mediterranean Desertification and Land Use Model, a series of international cooperation research projects funded by the European Union) is used worldwide to identify ‘sensitive areas’ that are potentially threatened by land degradation and desertification (LDD). The distinctive outcome of the approach is a multidimensional index (the ESAI) composed of partial indicators of climate, soil, vegetation, and management quality that are derived from the elaboration of 14 elementary variables.

\sphinxAtStartPar
All the varialbes are grouped into four Quality Indicators (Soil quality, SQI; vegetation quality, VQI; climate quality, CQI; and management quality, MQI), which were estimated as the geometric mean of the respective scores of the elementary variables.


\section{Calculating Individual Quality Indicators}
\label{\detokenize{Service/Calculate_Medalus:calculating-individual-quality-indicators}}
\sphinxAtStartPar
The current implementation of the MEDALUS model in MISLAND overcomes the problem of no data by computing the geometric mean of Individual Quality Indicators by using the variables with available information for any of the elementary variables.

\begin{sphinxadmonition}{note}{Note:}
\sphinxAtStartPar
To upscale the model for regional analysis, the following cosiderations were made for the selection of variables to be used in the computation of individual Quality indices: (a) Consistancy with the original MEDALUS Approach; (b) Time\sphinxhyphen{}series data availability and regularity for multi\sphinxhyphen{}temporal analysis; and (c) data source quality and reliability for future updates.
\end{sphinxadmonition}

\sphinxAtStartPar
To compute the individual quality indicators(Soil quality, SQI; vegetation quality, VQI; climate quality, CQI; and management quality, MQI), Follow the following simple steps:
\begin{enumerate}
\sphinxsetlistlabels{\arabic}{enumi}{enumii}{}{.}%
\item {} 
\sphinxAtStartPar
On the service menu\sphinxhyphen{}bar select the \sphinxincludegraphics{{deserticon}.png} option and click on the \sphinxincludegraphics{{cog}.png} icon to togle the layer settings dialog as shown below:

\end{enumerate}

\begin{figure}[H]
\centering
\capstart

\noindent\sphinxincludegraphics[width=700\sphinxpxdimen,height=396\sphinxpxdimen]{{medalus1}.png}
\caption{MEDALUS layer settings dialog.}\label{\detokenize{Service/Calculate_Medalus:id1}}\end{figure}
\begin{enumerate}
\sphinxsetlistlabels{\arabic}{enumi}{enumii}{}{.}%
\setcounter{enumi}{1}
\item {} 
\sphinxAtStartPar
On the layer settings dialog select the Quality index to compute from the dropdown list and the year you wish to compute:

\end{enumerate}

\begin{figure}[H]
\centering
\capstart

\noindent\sphinxincludegraphics[width=705\sphinxpxdimen,height=515\sphinxpxdimen]{{medalus2}.png}
\caption{Selecting the Quality index to compute}\label{\detokenize{Service/Calculate_Medalus:id2}}\end{figure}

\sphinxAtStartPar
The resultant layer and statistics will be computed and visualized as shown

\begin{figure}[H]
\centering
\capstart

\noindent\sphinxincludegraphics[width=700\sphinxpxdimen,height=395\sphinxpxdimen]{{medalus3}.png}
\caption{Example of results for Management Quality Index computation}\label{\detokenize{Service/Calculate_Medalus:id3}}\end{figure}


\section{Calculate the Environmental Sensitivity Areas Index(ESAI)}
\label{\detokenize{Service/Calculate_Medalus:calculate-the-environmental-sensitivity-areas-index-esai}}
\sphinxAtStartPar
To coumpute the Environmental Sensitivity Index select MEDALUS option from the service menu. On the layers selection option dropdown, selsect the ESAI option as shown below

\begin{figure}[H]
\centering
\capstart

\noindent\sphinxincludegraphics[width=700\sphinxpxdimen,height=500\sphinxpxdimen]{{medalus4}.png}
\caption{Selecting the ESAI option from the layer selection dropdown}\label{\detokenize{Service/Calculate_Medalus:id4}}\end{figure}

\sphinxstepscope


\chapter{Exporting Outputs}
\label{\detokenize{Service/Download_results:exporting-outputs}}\label{\detokenize{Service/Download_results::doc}}
\sphinxAtStartPar
Exporting outputs on the service is as simple and intuitive. Users can download the maps, charts and data following the steps highligted in this section of the document.


\section{Export Map}
\label{\detokenize{Service/Download_results:export-map}}
\sphinxAtStartPar
Map outputs can be exported in .png format. To export the map, users can click on the export map tool that is found on the map navigation tools as shown below.

\begin{figure}[H]
\centering
\capstart

\noindent\sphinxincludegraphics[width=800\sphinxpxdimen,height=634\sphinxpxdimen]{{exportmap}.png}
\caption{Image file of exported map}\label{\detokenize{Service/Download_results:id1}}\end{figure}


\section{Export Chart}
\label{\detokenize{Service/Download_results:export-chart}}
\sphinxAtStartPar
The chart image can be exported from the statistics tab by clicking on the export image icon \sphinxincludegraphics{{exportimageicon}.png} on the list of icons at the top\sphinxhyphen{}right conner of the chart area.

\begin{figure}[H]
\centering
\capstart

\noindent\sphinxincludegraphics[width=170\sphinxpxdimen,height=80\sphinxpxdimen]{{export1}.png}
\caption{Export chart as image}\label{\detokenize{Service/Download_results:id2}}\end{figure}

\sphinxAtStartPar
An example of an image file export is shown below.

\begin{figure}[H]
\centering
\capstart

\noindent\sphinxincludegraphics[width=429\sphinxpxdimen,height=214\sphinxpxdimen]{{chartimage}.png}
\caption{Export chart as image}\label{\detokenize{Service/Download_results:id3}}\end{figure}

\sphinxstepscope


\chapter{Download data}
\label{\detokenize{Service/Download_data:download-data}}\label{\detokenize{Service/Download_data::doc}}
\sphinxAtStartPar
In addition to exporting charts as an option, users can also downlad the data and create custom charts or perform further analyses.

\sphinxAtStartPar
The download data icon \sphinxincludegraphics{{downloaddataicon}.png} icon can be found just below the summary text under the statistics panel as in the figure below.

\begin{figure}[H]
\centering
\capstart

\noindent\sphinxincludegraphics[width=411\sphinxpxdimen,height=530\sphinxpxdimen]{{downloaddata}.png}
\caption{Download CSV file option}\label{\detokenize{Service/Download_data:id1}}\end{figure}

\sphinxAtStartPar
The downloaded data is in .csv format and can be open in microsoft exel or similar software

\begin{figure}[H]
\centering
\capstart

\noindent\sphinxincludegraphics[width=355\sphinxpxdimen,height=458\sphinxpxdimen]{{download1}.png}
\caption{Exported data as CSV}\label{\detokenize{Service/Download_data:id2}}\end{figure}


\section{Downloading Raster Data}
\label{\detokenize{Service/Download_data:downloading-raster-data}}
\sphinxAtStartPar
MISLAND Service users can also download the data in GeoTIF format for further analysis or visualization

\sphinxAtStartPar
To download the raster data, Click on the download tiff just below the service menu\sphinxhyphen{}bar as shown below

\begin{figure}[H]
\centering
\capstart

\noindent\sphinxincludegraphics[width=209\sphinxpxdimen,height=85\sphinxpxdimen]{{download2}.png}
\caption{Download tiff file option}\label{\detokenize{Service/Download_data:id3}}\end{figure}

\sphinxAtStartPar
This will prompt you to save the file in your desired location. The downloaded raster can be visualized and analysed with your desired software or tools.

\begin{figure}[H]
\centering
\capstart

\noindent\sphinxincludegraphics[width=653\sphinxpxdimen,height=522\sphinxpxdimen]{{download3}.png}
\caption{Saving the downloaded tiff}\label{\detokenize{Service/Download_data:id4}}\end{figure}

\sphinxstepscope


\chapter{Plugin Development}
\label{\detokenize{Qgis_Plugin/plugin_development:plugin-development}}\label{\detokenize{Qgis_Plugin/plugin_development::doc}}
\sphinxAtStartPar
MISLAND is free and open\sphinxhyphen{}source software, licensed under the \sphinxhref{https://www.gnu.org/licenses/old-licenses/gpl-2.0.en.html}{GNU
General Public License, version 2.0 or later}.

\sphinxAtStartPar
There are a number of components to the MISLAND tool. The first is a
QGIS plugin supporting calculation of indicators, access to raw data,
reporting, and production of print maps . The code for the plugin, and further
instructions on installing it if you want to modify the code, are in
\sphinxhref{https://github.com/LocateIT/trends.earth}{MISLAND}
GitHub repository.

\sphinxAtStartPar
The MISLAND QGIS plugin is supported by a number of different Python
scripts that allow calculation of the various indicators on Google Earth Engine
(GEE). These scripts sit in the “gee” sub\sphinxhyphen{}folder of that GitHub repository. The
GEE scripts are supported by the \sphinxtitleref{landdegradation} Python module, which
includes code for processing inputs and outputs for the plugin, as well as
other common functions supporting calculation of NDVI integrals, statistical
significance, and other shared code. The code for this module is available in
the \sphinxhref{https://github.com/ConservationInternational/landdegradation}{landdegradation} repository on
GitHub.

\sphinxAtStartPar
Further details are below on how to contribute to MISLAND by working on
the plugin code, by modifying the processing code, or by contributing to
translating the website and plugin.

\begin{sphinxShadowBox}
\sphinxstyletopictitle{Contents}
\begin{itemize}
\item {} 
\sphinxAtStartPar
\phantomsection\label{\detokenize{Qgis_Plugin/plugin_development:id5}}{\hyperref[\detokenize{Qgis_Plugin/plugin_development:plugin-development}]{\sphinxcrossref{Plugin Development}}}
\begin{itemize}
\item {} 
\sphinxAtStartPar
\phantomsection\label{\detokenize{Qgis_Plugin/plugin_development:id6}}{\hyperref[\detokenize{Qgis_Plugin/plugin_development:modifying-the-qgis-plugin-code}]{\sphinxcrossref{Modifying the QGIS Plugin code}}}
\begin{itemize}
\item {} 
\sphinxAtStartPar
\phantomsection\label{\detokenize{Qgis_Plugin/plugin_development:id7}}{\hyperref[\detokenize{Qgis_Plugin/plugin_development:downloading-the-misland-code}]{\sphinxcrossref{Downloading the MISLAND code}}}

\item {} 
\sphinxAtStartPar
\phantomsection\label{\detokenize{Qgis_Plugin/plugin_development:id8}}{\hyperref[\detokenize{Qgis_Plugin/plugin_development:installing-dependencies}]{\sphinxcrossref{Installing dependencies}}}
\begin{itemize}
\item {} 
\sphinxAtStartPar
\phantomsection\label{\detokenize{Qgis_Plugin/plugin_development:id9}}{\hyperref[\detokenize{Qgis_Plugin/plugin_development:python}]{\sphinxcrossref{Python}}}

\item {} 
\sphinxAtStartPar
\phantomsection\label{\detokenize{Qgis_Plugin/plugin_development:id10}}{\hyperref[\detokenize{Qgis_Plugin/plugin_development:python-dependencies}]{\sphinxcrossref{Python dependencies}}}

\item {} 
\sphinxAtStartPar
\phantomsection\label{\detokenize{Qgis_Plugin/plugin_development:id11}}{\hyperref[\detokenize{Qgis_Plugin/plugin_development:pyqt}]{\sphinxcrossref{PyQt}}}

\end{itemize}

\item {} 
\sphinxAtStartPar
\phantomsection\label{\detokenize{Qgis_Plugin/plugin_development:id12}}{\hyperref[\detokenize{Qgis_Plugin/plugin_development:changing-the-version-of-the-plugin}]{\sphinxcrossref{Changing the version of the plugin}}}

\item {} 
\sphinxAtStartPar
\phantomsection\label{\detokenize{Qgis_Plugin/plugin_development:id13}}{\hyperref[\detokenize{Qgis_Plugin/plugin_development:testing-changes-to-the-plugin}]{\sphinxcrossref{Testing changes to the plugin}}}

\item {} 
\sphinxAtStartPar
\phantomsection\label{\detokenize{Qgis_Plugin/plugin_development:id14}}{\hyperref[\detokenize{Qgis_Plugin/plugin_development:syncing-and-deploying-changes-to-the-binaries}]{\sphinxcrossref{Syncing and deploying changes to the binaries}}}

\item {} 
\sphinxAtStartPar
\phantomsection\label{\detokenize{Qgis_Plugin/plugin_development:id15}}{\hyperref[\detokenize{Qgis_Plugin/plugin_development:building-a-plugin-zip-file}]{\sphinxcrossref{Building a plugin ZIP file}}}

\item {} 
\sphinxAtStartPar
\phantomsection\label{\detokenize{Qgis_Plugin/plugin_development:id16}}{\hyperref[\detokenize{Qgis_Plugin/plugin_development:deploying-the-development-version-zip-file}]{\sphinxcrossref{Deploying the development version ZIP file}}}

\end{itemize}

\item {} 
\sphinxAtStartPar
\phantomsection\label{\detokenize{Qgis_Plugin/plugin_development:id17}}{\hyperref[\detokenize{Qgis_Plugin/plugin_development:modifying-the-earth-engine-processing-code}]{\sphinxcrossref{Modifying the Earth Engine processing code}}}
\begin{itemize}
\item {} 
\sphinxAtStartPar
\phantomsection\label{\detokenize{Qgis_Plugin/plugin_development:id18}}{\hyperref[\detokenize{Qgis_Plugin/plugin_development:setting-up-dependencies}]{\sphinxcrossref{Setting up dependencies}}}
\begin{itemize}
\item {} 
\sphinxAtStartPar
\phantomsection\label{\detokenize{Qgis_Plugin/plugin_development:id19}}{\hyperref[\detokenize{Qgis_Plugin/plugin_development:trends-earth-cli}]{\sphinxcrossref{trends.earth\sphinxhyphen{}CLI}}}

\item {} 
\sphinxAtStartPar
\phantomsection\label{\detokenize{Qgis_Plugin/plugin_development:id20}}{\hyperref[\detokenize{Qgis_Plugin/plugin_development:docker}]{\sphinxcrossref{docker}}}

\end{itemize}

\item {} 
\sphinxAtStartPar
\phantomsection\label{\detokenize{Qgis_Plugin/plugin_development:id21}}{\hyperref[\detokenize{Qgis_Plugin/plugin_development:testing-an-earth-engine-script-locally}]{\sphinxcrossref{Testing an Earth Engine script locally}}}

\item {} 
\sphinxAtStartPar
\phantomsection\label{\detokenize{Qgis_Plugin/plugin_development:id22}}{\hyperref[\detokenize{Qgis_Plugin/plugin_development:deploying-a-gee-script-to-api-trends-earth}]{\sphinxcrossref{Deploying a GEE script to api.trends.earth}}}

\end{itemize}

\end{itemize}

\end{itemize}
\end{sphinxShadowBox}


\section{Modifying the QGIS Plugin code}
\label{\detokenize{Qgis_Plugin/plugin_development:modifying-the-qgis-plugin-code}}

\subsection{Downloading the MISLAND code}
\label{\detokenize{Qgis_Plugin/plugin_development:downloading-the-misland-code}}
\sphinxAtStartPar
The MISLAND code for both the plugin and the Google Earth Engine scripts
that support it are located on GitHub in the \sphinxhref{https://github.com/LocateIT/trends.earth}{MISLAND} repository. Clone
this repository to a convenient place on your machine in order to ensure you
have the latest version of the code.

\sphinxAtStartPar
There are a number of different branches of the MISLAND repository that
are under active development. While the plugin does not yet officially support
QGIS3, however the majority of development is occurring on the “master” branch,
which is aimed at QGIS3. The “qgis2” branch is the older version of the plugin,
and supports QGIS2 version 2.18+.

\sphinxAtStartPar
The first time you download the MISLAND code, you will also need to clone
the “schemas” submodule that is located within it, under “LDMP\textbackslash{}schemas”. If
you are using TortoiseGit on Windows, you can right\sphinxhyphen{}click anywhere within the
MISLAND folder and choose “TortoiseGit” and then “Submodule Update…”.
Clicking ok in the window that comes up will checkout the schemas submodule. If
you prefer, you can also do this from the command line by running the below two
commands in shell:

\begin{sphinxVerbatim}[commandchars=\\\{\}]
\PYG{n}{git} \PYG{n}{submodule} \PYG{n}{init}
\PYG{n}{git} \PYG{n}{submodule} \PYG{n}{update}
\end{sphinxVerbatim}

\sphinxAtStartPar
Once you are done you should see files within the “LDMP\textbackslash{}schemas” folder within
the MISLAND folder.


\subsection{Installing dependencies}
\label{\detokenize{Qgis_Plugin/plugin_development:installing-dependencies}}

\subsubsection{Python}
\label{\detokenize{Qgis_Plugin/plugin_development:python}}
\sphinxAtStartPar
The plugin is coded in Python. In addition to being used to run the plugin
through QGIS, Python is also used to support managing the plugin (changing the
version, installing development versions, etc.). Though Python is included with
QGIS, you will also need a local version of Python that you can setup with the
software needed to manage the plugin. The easiest way to manage multiple
versions of Python is through the \sphinxhref{https://www.anaconda.com}{Anaconda distribution}. For work developing the plugin, Python
3 is required. To download Python 3.7 (recommended) though Anaconda,
\sphinxhref{https://www.anaconda.com/distribution/\#download-section}{see this page}.


\subsubsection{Python dependencies}
\label{\detokenize{Qgis_Plugin/plugin_development:python-dependencies}}
\sphinxAtStartPar
In order to work with the MISLAND code, you need to have Invoke
installed on your machine, as well as a number of other packages that are used
for managing the documentation, translations, etc. These packages are all
listed in the “dev” requirements file for MISLAND, so they can be
installed by navigating in a command prompt to the root of the MISLAND
code folder and typing:

\begin{sphinxVerbatim}[commandchars=\\\{\}]
\PYG{n}{pip} \PYG{n}{install} \PYG{o}{\PYGZhy{}}\PYG{n}{r} \PYG{n}{requirements}\PYG{o}{\PYGZhy{}}\PYG{n}{dev}\PYG{o}{.}\PYG{n}{txt}
\end{sphinxVerbatim}

\begin{sphinxadmonition}{note}{Note:}
\sphinxAtStartPar
If you are using Anaconda, you will first want to activate a Python 3.7
virtual environment before running the above command (and any of the other
invoke commands listed on the page). One way to do this is by starting an
“Anaconda prompt”, by \sphinxhref{https://docs.anaconda.com/anaconda/user-guide/getting-started/\#write-a-python-program-using-anaconda-prompt-or-terminal}{following the instructions on this Anaconda page}.
\end{sphinxadmonition}


\subsubsection{PyQt}
\label{\detokenize{Qgis_Plugin/plugin_development:pyqt}}
\sphinxAtStartPar
PyQt5 is the graphics toolkit used by QGIS3. To compile the user interface for
MISLAND for QGIS3 you need to install PyQt5. This package can be installed
from pip using:

\begin{sphinxVerbatim}[commandchars=\\\{\}]
\PYG{n}{pip} \PYG{n}{install} \PYG{n}{PyQt5}
\end{sphinxVerbatim}

\begin{sphinxadmonition}{note}{Note:}
\sphinxAtStartPar
PyQt4 is the graphics toolkit used by QGIS2. The best source for this
package on Windows is from the set of packages maintained by Christoph
Gohlke at UC Irvine. To download PyQt4, select \sphinxhref{https://www.lfd.uci.edu/~gohlke/pythonlibs/\#pyqt4}{the appropriate package
from this page}.
Choose the appropriate file for the version of Python you are using. For
example, if you are using Python 2.7, choose the version with “cp27” in the
filename. If you are using Python 3.7, choose the version with “cp37” in
the filename. Choose “amd64” for 64\sphinxhyphen{}bit python, and “win32” for 32\sphinxhyphen{}bit
python.

\sphinxAtStartPar
After downloading from the above link, use \sphinxcode{\sphinxupquote{pip}} to install it. For example,
for the 64\sphinxhyphen{}bit wheel for Python 3.7, you would run:

\begin{sphinxVerbatim}[commandchars=\\\{\}]
\PYG{n}{pip} \PYG{n}{install} \PYG{n}{PyQt4}\PYG{o}{\PYGZhy{}}\PYG{l+m+mf}{4.11}\PYG{l+m+mf}{.4}\PYG{o}{\PYGZhy{}}\PYG{n}{cp37}\PYG{o}{\PYGZhy{}}\PYG{n}{cp37m}\PYG{o}{\PYGZhy{}}\PYG{n}{win\PYGZus{}amd64}\PYG{o}{.}\PYG{n}{whl}
\end{sphinxVerbatim}
\end{sphinxadmonition}


\subsection{Changing the version of the plugin}
\label{\detokenize{Qgis_Plugin/plugin_development:changing-the-version-of-the-plugin}}
\sphinxAtStartPar
The convention for MISLAND is that version numbers ending in an odd number
(for example 0.65) are development versions, while versions ending in an even
number (for example (0.66) are release versions. Development versions of the
plugin are never released via the QGIS repository, so they are never seen by
normal users of the plugin. Odd\sphinxhyphen{}numbered development versions are used by the
MISLAND development team while testing new features prior to their public
release.

\sphinxAtStartPar
If you wish to make changes to the code and have downloaded a public release of
the plugin (one ending in an even number), the first step is to update the
version of the plugin to the next sequential odd number. So, for example, if
you downloaded version 0.66 of the plugin, you would need to update the version
to be 0.67 before you started making your changes. There are several places in
the code where the version is mentioned (as well as within every GEE script) so
there is an invoke task to assist with changing the version. To change the
version to be 0.67, you would run:

\begin{sphinxVerbatim}[commandchars=\\\{\}]
\PYG{n}{invoke} \PYG{n+nb}{set}\PYG{o}{\PYGZhy{}}\PYG{n}{version} \PYG{o}{\PYGZhy{}}\PYG{n}{v} \PYG{l+m+mf}{0.67}
\end{sphinxVerbatim}

\sphinxAtStartPar
Running the above command will update the version number every place it is
referenced in the code. To avoid confusion, never change the version to one
that has already been released \sphinxhyphen{} always INCREASE the value of the version tag
to the next odd number.


\subsection{Testing changes to the plugin}
\label{\detokenize{Qgis_Plugin/plugin_development:testing-changes-to-the-plugin}}
\sphinxAtStartPar
After making changes to the plugin code, you will need to test them to ensure
the plugin behaves as expected, and to ensure no bugs or errors come up. The
plugin should go through extensive testing before it is released to the QGIS
repository (where it can be accessed by other users) to ensure that any changes
to the code do not break the plugin.

\sphinxAtStartPar
To test any changes that you have made to the plugin within QGIS, you will need
to install it locally. There are invoke tasks that assist with this process.
The first step prior to installing the plugin is ensuring that you have setup
the plugin with all of the dependencies that it needs in order to run from
within QGIS. To do this, run:

\begin{sphinxVerbatim}[commandchars=\\\{\}]
\PYG{n}{invoke} \PYG{n}{plugin}\PYG{o}{\PYGZhy{}}\PYG{n}{setup}
\end{sphinxVerbatim}

\sphinxAtStartPar
The above task only needs to be run immediately after downloading the
MISLAND code, or if any changes are made to the dependencies for the
plugin. By default \sphinxcode{\sphinxupquote{plugin\sphinxhyphen{}setup}} will re\sphinxhyphen{}use any cached files on your
machine. To start from scratch, add the \sphinxcode{\sphinxupquote{\sphinxhyphen{}c}} (clean) flag to the above
command.

\sphinxAtStartPar
After running \sphinxcode{\sphinxupquote{plugin\sphinxhyphen{}setup}}, you are ready to install the plugin to the QGIS
plugins folder on your machine. To do this, run:

\begin{sphinxVerbatim}[commandchars=\\\{\}]
\PYG{n}{invoke} \PYG{n}{plugin}\PYG{o}{\PYGZhy{}}\PYG{n}{install}
\end{sphinxVerbatim}

\sphinxAtStartPar
After running the above command, you will need to either 1) restart QGIS, or 2)
use the \sphinxhref{https://plugins.qgis.org/plugins/plugin\_reloader/}{plugin reloader}
to reload the MISLAND plugin in order to see the effects of the changes
you have made.

\sphinxAtStartPar
By default \sphinxcode{\sphinxupquote{plugin\sphinxhyphen{}install}} will overwrite any existing plugin files on your
machine, but leave in place any data (administrative boundaries, etc.) that the
plugin might have downloaded. To start from scratch, add the \sphinxcode{\sphinxupquote{\sphinxhyphen{}c}} (clean)
flag to the above command. You may need to close QGIS in order to successfully
perform a clean install of the plugin using the \sphinxcode{\sphinxupquote{\sphinxhyphen{}c}} flag.

\begin{sphinxadmonition}{note}{Note:}
\sphinxAtStartPar
By default plugin\sphinxhyphen{}install assumes you want to install the plugin to be used
in QGIS3. To install the plugin for use in QGIS3, add the flag \sphinxcode{\sphinxupquote{\sphinxhyphen{}v 2}} to
the \sphinxcode{\sphinxupquote{plugin\sphinxhyphen{}install}} command. Remember the plugin may or may not be
entirely functional on QGIS3 \sphinxhyphen{} the plugin was originally designed for QGIS2
and is still being tested on QGIS3.
\end{sphinxadmonition}


\subsection{Syncing and deploying changes to the binaries}
\label{\detokenize{Qgis_Plugin/plugin_development:syncing-and-deploying-changes-to-the-binaries}}
\sphinxAtStartPar
To speed the computations in MISLAND, some of the tools allow making use
of pre\sphinxhyphen{}compiled binaries that have been compiled using \sphinxhref{https://numba.pydata.org}{numba}. Numba is an open source compiler that can compile
Python and NumPy code, making it faster than when it is run as ordinary Python.
To avoid users of MISLAND needing to download Numba and all of its
dependencies, the MISLAND team makes pre\sphinxhyphen{}compiled binaries available for
download if users choose to install them.

\sphinxAtStartPar
To generate pre\sphinxhyphen{}compiled binaries for the OS, bitness (32/64 bit) and Python
version you are running on your machine, use:

\begin{sphinxVerbatim}[commandchars=\\\{\}]
\PYG{n}{invoke} \PYG{n}{binaries}\PYG{o}{\PYGZhy{}}\PYG{n+nb}{compile}
\end{sphinxVerbatim}

\begin{sphinxadmonition}{note}{Note:}
\sphinxAtStartPar
You will need a C++ compiler for the above command to work. On
Windows, see \sphinxhref{https://wiki.python.org/moin/WindowsCompilers\#Which\_Microsoft\_Visual\_C.2B-.2B-\_compiler\_to\_use\_with\_a\_specific\_Python\_version\_.3F}{this github page}
for details on how to
install the Microsoft Visual C++ compiler needed for you Python version. On
MacOS, you will most likely need to install Xcode. On Linux, install the
appropriate version of GCC.
\end{sphinxadmonition}

\sphinxAtStartPar
To make binaries publicly available, they are distributed through an Amazon Web
services S3 bucket. To upload the binaries generated with the above command to
the bucket, run:

\begin{sphinxVerbatim}[commandchars=\\\{\}]
\PYG{n}{invoke} \PYG{n}{binaries}\PYG{o}{\PYGZhy{}}\PYG{n}{sync}
\end{sphinxVerbatim}

\begin{sphinxadmonition}{note}{Note:}
\sphinxAtStartPar
The above command will fail if you do not have keys allowing write
access to the \sphinxcode{\sphinxupquote{MISLAND}} bucket on S3.
\end{sphinxadmonition}

\sphinxAtStartPar
The above command will sync each individual binary file to S3. However, users
of the toolbox download the binaries as a single zipfile tied to the version of
the plugin that they are using. To generate that zipfile so that it can be
accessed by MISLAND users, run:

\begin{sphinxVerbatim}[commandchars=\\\{\}]
\PYG{n}{invoke} \PYG{n}{binaries}\PYG{o}{\PYGZhy{}}\PYG{n}{deploy}
\end{sphinxVerbatim}

\begin{sphinxadmonition}{note}{Note:}
\sphinxAtStartPar
The above command will fail if you do not have keys allowing write
access to the \sphinxcode{\sphinxupquote{MISLAND}} bucket on S3.
\end{sphinxadmonition}


\subsection{Building a plugin ZIP file}
\label{\detokenize{Qgis_Plugin/plugin_development:building-a-plugin-zip-file}}
\sphinxAtStartPar
There are several invoke tasks to help with building a ZIP file to deploy the
plugin to the QGIS repository, or to share the development version of the
plugin with others. To package the plugin and all of its dependencies into a
ZIP file that can be installed following, run:
\begin{quote}

\sphinxAtStartPar
invoke zipfile\sphinxhyphen{}build
\end{quote}

\sphinxAtStartPar
This command will create a folder named \sphinxcode{\sphinxupquote{build}} at the root of the
MISLAND code folder, and in that folder it will create a file called
\sphinxcode{\sphinxupquote{LDMP.zip}}. This file can be shared with others, who can use it to manually
install MISLAND.

\sphinxAtStartPar
This can be useful if there is a need to share the latest features with someone
before they are available in the publicly released version of the plugin.


\subsection{Deploying the development version ZIP file}
\label{\detokenize{Qgis_Plugin/plugin_development:deploying-the-development-version-zip-file}}
\sphinxAtStartPar
The MISLAND GitHub page gives a link a ZIP file that allows users who may
not be developers to access the development version of MISLAND. To create
a ZIP file and make it available on that page (the ZIP file is stored on S3),
run:

\begin{sphinxVerbatim}[commandchars=\\\{\}]
\PYG{n}{invoke} \PYG{n}{zipfile}\PYG{o}{\PYGZhy{}}\PYG{n}{deploy}
\end{sphinxVerbatim}

\begin{sphinxadmonition}{note}{Note:}
\sphinxAtStartPar
The above command will fail if you do not have keys allowing write
access to the \sphinxcode{\sphinxupquote{misland}} bucket on S3.
\end{sphinxadmonition}


\section{Modifying the Earth Engine processing code}
\label{\detokenize{Qgis_Plugin/plugin_development:modifying-the-earth-engine-processing-code}}
\sphinxAtStartPar
The Google Earth Engine (GEE) processing scripts used by MISLAND are all
stored in the “gee” folder under the main MISLAND folder. For these script
to be accessible to users of the MISLAND QGIS plugin, they have to be
deployed to the api.trends.earth serviceThe below
describes how to test and deploy GEE scripts to be used with MISLAND.


\subsection{Setting up dependencies}
\label{\detokenize{Qgis_Plugin/plugin_development:setting-up-dependencies}}

\subsubsection{trends.earth\sphinxhyphen{}CLI}
\label{\detokenize{Qgis_Plugin/plugin_development:trends-earth-cli}}
\sphinxAtStartPar
The “trends.earth\sphinxhyphen{}CLI” Python package is required in order to work with the
api.trends.earth server. This package is located on GitHub in the
\sphinxhref{https://github.com/LocateIT/trends.earth-CLI}{trends.earth\sphinxhyphen{}CLI}
repository.

\sphinxAtStartPar
The first step is to clone this repository onto your machine. We recommend that
you clone the repository into the same folder where you the trends.earth code.
For example, if you had a “Code” folder on your machine, clone both the
\sphinxhref{https://github.com/LocateIT/trends.earth}{trends.earth} repository (the
code for the QGIS plugin and associated GEE scripts) and also the
\sphinxhref{https://github.com/LocateIT/trends.earth-CLI}{trends.earth\sphinxhyphen{}CLI} repository
into that same folder.

\sphinxAtStartPar
When you setup your system as recommended above, trends.earth\sphinxhyphen{}CLI will work
with the invoke tasks used to manage MISLAND without any modifications.
If, however, you download trends.earth\sphinxhyphen{}CLI into a different folder, then you
will need to add a file named “invoke.yaml” file into the root of the
MISLAND repository, and in that file tell MISLAND where to locate the
trends.earth\sphinxhyphen{}CLI code. This YAML file should look something like the below (if
you downloaded the code on Windows into a folder called
“C:/Users/grace/Code/trends.earth\sphinxhyphen{}CLI/tecli”):

\begin{sphinxVerbatim}[commandchars=\\\{\}]
\PYG{n+nt}{gee}\PYG{p}{:}
\PYG{+w}{    }\PYG{n+nt}{tecli}\PYG{p}{:}\PYG{+w}{ }\PYG{l+s}{\PYGZdq{}}\PYG{l+s}{C:/Users/grace/Code/trends.earth\PYGZhy{}CLI/tecli}\PYG{l+s}{\PYGZdq{}}
\end{sphinxVerbatim}

\sphinxAtStartPar
Again, you \sphinxstylestrong{do not} need to add this .yaml file if you setup your system as
recommended above.


\subsubsection{docker}
\label{\detokenize{Qgis_Plugin/plugin_development:docker}}
\sphinxAtStartPar
The trends.earth\sphinxhyphen{}CLI package requires \sphinxhref{http://www.docker.com}{docker} in
order to function. \sphinxhref{https://docs.docker.com/docker-for-windows/install/}{Follow these instructions to install docker on Windows}, and \sphinxhref{https://docs.docker.com/docker-for-mac/install/}{these
instructions to install docker on Mac OS}. If you are running
Linux, \sphinxhref{https://docs.docker.com/install}{follow the instructions on this page} that are appropriate for the Linux
distribution you are using.


\subsection{Testing an Earth Engine script locally}
\label{\detokenize{Qgis_Plugin/plugin_development:testing-an-earth-engine-script-locally}}
\sphinxAtStartPar
After installing the trends.earth\sphinxhyphen{}CLI package, you will need to setup a
.tecli.yml file with an access token to a GEE service account in order to test
scripts on GEE. To setup the GEE service account for tecli, first obtain the
key for your service account in JSON format (from the google cloud console),
then and encode it in base64. Provide that base64 encoded key to tecli with the
following command:

\begin{sphinxVerbatim}[commandchars=\\\{\}]
\PYG{n}{invoke} \PYG{n}{tecli}\PYG{o}{\PYGZhy{}}\PYG{n}{config} \PYG{n+nb}{set} \PYG{n}{EE\PYGZus{}SERVICE\PYGZus{}ACCOUNT\PYGZus{}JSON} \PYG{n}{key}
\end{sphinxVerbatim}

\sphinxAtStartPar
where “key” is the base64 encoded JSON format service account key.

\sphinxAtStartPar
While converting a script specifying code to be run on GEE from JavaScript to
Python, or when making modifications to that code, it can be useful to test the
script locally, without deploying it to the api.trends.earth server. To do
this, use the \sphinxcode{\sphinxupquote{run}} invoke task. For example, to test the “land\_cover”
script, go to the root directory of the code, and, in a command
prompt, run:

\begin{sphinxVerbatim}[commandchars=\\\{\}]
\PYG{n}{invoke} \PYG{n}{tecli}\PYG{o}{\PYGZhy{}}\PYG{n}{run} \PYG{n}{land\PYGZus{}cover}
\end{sphinxVerbatim}

\sphinxAtStartPar
This will use the trends.earth\sphinxhyphen{}CLI package to build and run a docker container
that will attempt to run the “land\_cover” script. If there are any syntax
errors in the script, these will show up when the container is run. Before
submitting a new script to api.trends.earth, always make sure that \sphinxcode{\sphinxupquote{invoke
tecli\sphinxhyphen{}run}} is able to run the script without any errors.

\sphinxAtStartPar
When using \sphinxcode{\sphinxupquote{invoke tecli\sphinxhyphen{}run}} you may get an error saying:

\begin{sphinxVerbatim}[commandchars=\\\{\}]
Invalid\PYG{+w}{ }JWT:\PYG{+w}{ }Token\PYG{+w}{ }must\PYG{+w}{ }be\PYG{+w}{ }a\PYG{+w}{ }short\PYGZhy{}lived\PYG{+w}{ }token\PYG{+w}{ }\PYG{o}{(}\PYG{l+m}{60}\PYG{+w}{ }minutes\PYG{o}{)}\PYG{+w}{ }and\PYG{+w}{ }\PYG{k}{in}\PYG{+w}{ }a
reasonable\PYG{+w}{ }timeframe.\PYG{+w}{ }Check\PYG{+w}{ }your\PYG{+w}{ }iat\PYG{+w}{ }and\PYG{+w}{ }exp\PYG{+w}{ }values\PYG{+w}{ }and\PYG{+w}{ }use\PYG{+w}{ }a\PYG{+w}{ }clock\PYG{+w}{ }with
skew\PYG{+w}{ }to\PYG{+w}{ }account\PYG{+w}{ }\PYG{k}{for}\PYG{+w}{ }clock\PYG{+w}{ }differences\PYG{+w}{ }between\PYG{+w}{ }systems.
\end{sphinxVerbatim}

\sphinxAtStartPar
This error can be caused if the clock on the docker container gets out of sync
with the system clock. Restarting docker should fix this error.


\subsection{Deploying a GEE script to api.trends.earth}
\label{\detokenize{Qgis_Plugin/plugin_development:deploying-a-gee-script-to-api-trends-earth}}
\sphinxAtStartPar
When you have finished testing a GEE script and would like it to be accessible
using the QGIS plugin (and by other users of MISLAND), you can deploy it
to the api.trends.earth server. The first step in the process is logging in to
the api.trends.earth server. To login, run:

\begin{sphinxVerbatim}[commandchars=\\\{\}]
\PYG{n}{invoke} \PYG{n}{tecli}\PYG{o}{\PYGZhy{}}\PYG{n}{login}
\end{sphinxVerbatim}

\sphinxAtStartPar
You will be asked for a username and password. These are the same as the
username and password that you use to login to the MISLAND server from the
QGIS plugin. \sphinxstylestrong{If you are not an administrator, you will be able to login, but
the below command will fail}. To upload a script (for example, the
“land\_cover” script) to the server, run:

\begin{sphinxVerbatim}[commandchars=\\\{\}]
\PYG{n}{invoke} \PYG{n}{tecli}\PYG{o}{\PYGZhy{}}\PYG{n}{publish} \PYG{o}{\PYGZhy{}}\PYG{n}{s} \PYG{n}{land\PYGZus{}cover}
\end{sphinxVerbatim}

\sphinxAtStartPar
If this script already exists on the server, you will be asked if you want to
overwrite the existing script. Be very careful uploading scripts with
even\sphinxhyphen{}numbered versions, as these are publicly available scripts, and any errors
that you make will affect anyone using the plugin. Whenever you are testing be
sure to use development version numbers (odd version numbers).

\sphinxAtStartPar
After publishing a script to the server, you can use the \sphinxtitleref{tecli\sphinxhyphen{}info} task to
check the status of the script (to know whether it deployed successfully \sphinxhyphen{}
though note building the script may take a few minutes). To check the status,
of a deployed script, run:

\begin{sphinxVerbatim}[commandchars=\\\{\}]
\PYG{n}{invoke} \PYG{n}{tecli}\PYG{o}{\PYGZhy{}}\PYG{n}{publish} \PYG{o}{\PYGZhy{}}\PYG{n}{s} \PYG{n}{land\PYGZus{}cover}
\end{sphinxVerbatim}

\sphinxAtStartPar
If you are making a new release of the plugin, and want to upload ALL of the
GEE scripts at once (this is necessary whenever the plugin version number
changes), run:

\begin{sphinxVerbatim}[commandchars=\\\{\}]
\PYG{n}{invoke} \PYG{n}{tecli}\PYG{o}{\PYGZhy{}}\PYG{n}{publish}
\end{sphinxVerbatim}

\sphinxAtStartPar
Again \sphinxhyphen{} never run the above on a publicly released version of the plugin unless
you are intending to overwrite all the publicly available scripts used by the
plugin.

\sphinxstepscope


\chapter{Before installing the toolbox}
\label{\detokenize{Qgis_Plugin/before_installing:before-installing-the-toolbox}}\label{\detokenize{Qgis_Plugin/before_installing:before-installing}}\label{\detokenize{Qgis_Plugin/before_installing::doc}}
\sphinxAtStartPar
Before installing the toolbox, QGIS version {\color{red}\bfseries{}|qgisMinVersion|} or higher
needs to be installed on your computer.


\section{Download QGIS}
\label{\detokenize{Qgis_Plugin/before_installing:download-qgis}}
\sphinxAtStartPar
To install the plugin, first install QGIS 3.10.3+ following the below steps:
\begin{enumerate}
\sphinxsetlistlabels{\arabic}{enumi}{enumii}{}{.}%
\item {} 
\sphinxAtStartPar
Choose either 32 or 64 bit version

\sphinxAtStartPar
You have the option of installing a 32\sphinxhyphen{}bit or 64\sphinxhyphen{}bit version of QGIS. To
know which version to install, check which type of operating system you have
following the below instructions. If you are unsure which you need, try
downloading the 64 bit version first. If that version doesn’t work properly,
un\sphinxhyphen{}install it and then install the 32 bit version.
\begin{itemize}
\item {} 
\sphinxAtStartPar
Windows 8 or Windows 10
\begin{itemize}
\item {} 
\sphinxAtStartPar
From the “Start” screen, type “This PC”.

\item {} 
\sphinxAtStartPar
Right click (or tap and hold) “This PC”, and click “Properties”.

\end{itemize}

\item {} 
\sphinxAtStartPar
Windows 7, or Vista
\begin{itemize}
\item {} 
\sphinxAtStartPar
Open “System” by clicking the “Start” button , right\sphinxhyphen{}clicking
“Computer”, and then clicking “Properties”.

\item {} 
\sphinxAtStartPar
Under System, you can view the system type.

\end{itemize}

\item {} 
\sphinxAtStartPar
Mac: Click the Apple icon in the top left and select “About this Mac”.

\end{itemize}

\item {} 
\sphinxAtStartPar
After determining whether you need the 32 or 64 bit version, download the
appropriate installer:
\begin{itemize}
\item {} 
\sphinxAtStartPar
Windows: \sphinxhref{https://qgis.org/en/site/forusers/download.html\#windows}{Download Windows installer from here}.

\item {} 
\sphinxAtStartPar
MacOS: \sphinxhref{https://qgis.org/en/site/forusers/download.html\#mac}{Download MacOS installer from here}.

\item {} 
\sphinxAtStartPar
Linux: \sphinxhref{https://qgis.org/en/site/forusers/download.html\#linux}{Download Linux installer from here, or from the repository for
your Linux distribution}.

\end{itemize}

\end{enumerate}


\section{Install QGIS}
\label{\detokenize{Qgis_Plugin/before_installing:install-qgis}}
\sphinxAtStartPar
Once the installer is downloaded from the website, it needs to be run (double
click on it). Select the Default settings for all options.

\sphinxstepscope


\chapter{Installing the toolbox}
\label{\detokenize{Qgis_Plugin/installing:installing-the-toolbox}}\label{\detokenize{Qgis_Plugin/installing::doc}}


\sphinxAtStartPar
There are different ways to install MISLAND, depending on whether you want
to install the stable version (recommended) or the development version.


\section{Installing the development version}
\label{\detokenize{Qgis_Plugin/installing:installing-the-development-version}}
\sphinxAtStartPar
To install from within QGIS, first launch QGIS, and then go to \sphinxtitleref{Plugins} in the
menu bar at the top of the program and select \sphinxtitleref{Manage and install plugins}.

\noindent{\hspace*{\fill}\sphinxincludegraphics{{plugin_menu}.png}\hspace*{\fill}}

\sphinxAtStartPar
Then search navigate to Install from ZIP and upload the LDMS plugin zipfile

\noindent{\hspace*{\fill}\sphinxincludegraphics{{install_zip}.png}\hspace*{\fill}}

\sphinxAtStartPar
If your plugin has been installed properly, there will be a menu bar in the top
left of your browser that looks like this:

\noindent{\hspace*{\fill}\sphinxincludegraphics{{plugin_toolbar}.png}\hspace*{\fill}}

\sphinxstepscope


\chapter{Registration and settings}
\label{\detokenize{Qgis_Plugin/Registration:registration-and-settings}}\label{\detokenize{Qgis_Plugin/Registration::doc}}


\noindent{\hspace*{\fill}\sphinxincludegraphics{{plugin_toolbar_settings}.png}\hspace*{\fill}}


\section{Registration}
\label{\detokenize{Qgis_Plugin/Registration:registration}}\label{\detokenize{Qgis_Plugin/Registration:id1}}
\sphinxAtStartPar
The toolbox is free to use, but you must register an email address prior to
using any of the cloud\sphinxhyphen{}based functions.

\sphinxAtStartPar
To register your email address and obtain a free account, select the highlighted
above. This will open up the “Settings” dialog box:

\noindent{\hspace*{\fill}\sphinxincludegraphics{{settings}.png}\hspace*{\fill}}

\sphinxAtStartPar
To Register, click the “Step 1: Register” button. Enter your email, name,
organization and country(within North Africa region) of residence and select “Ok”:

\noindent{\hspace*{\fill}\sphinxincludegraphics{{registration}.png}\hspace*{\fill}}

\sphinxAtStartPar
You will see a meesage indicating your user has been registered:

\noindent{\hspace*{\fill}\sphinxincludegraphics{{registration_success}.png}\hspace*{\fill}}

\sphinxAtStartPar
After registering, you will receive an email from \sphinxhref{mailto:info@promiseevnts.co}{info@promiseevnts.co}.ke(for testing) with your
password. Once you receive this email, click on “Step 2: Enter login”. This
will bring up a dialog asking for your email and password. Enter the password
you received from \sphinxhref{mailto:info@promiseevnts.co}{info@promiseevnts.co}.ke(for testing) and click “Ok”:

\noindent{\hspace*{\fill}\sphinxincludegraphics{{login}.png}\hspace*{\fill}}

\sphinxAtStartPar
You will see a message indicating you have successfully been logged in:

\noindent{\hspace*{\fill}\sphinxincludegraphics{{login_success}.png}\hspace*{\fill}}


\section{Updating your user}
\label{\detokenize{Qgis_Plugin/Registration:updating-your-user}}
\sphinxAtStartPar
If you already are registered for LDMS but want to change your login
information; update your name, organization, or country; or delete your user,
click on “Update user” from the “Settings” dialog.

\noindent{\hspace*{\fill}\sphinxincludegraphics{{settings_update}.png}\hspace*{\fill}}

\sphinxAtStartPar
If you want to change your username, click on “Change user”. Note that this
function is only useful if you already have another existing LDMS
account you want to switch to. To register a new user, see {\hyperref[\detokenize{Qgis_Plugin/Registration:registration}]{\sphinxcrossref{\DUrole{std,std-ref}{Registration}}}}.
To change your user, enter the email and password you wish to change to and
click “Ok”:

\noindent{\hspace*{\fill}\sphinxincludegraphics{{login}.png}\hspace*{\fill}}

\sphinxAtStartPar
If you want to update your profile, click on “Update profile”. Update your
information in the box that appears and click “Save”:

\sphinxAtStartPar
To delete your user, click “Delete user”. A warning message will appear. Click
“Ok” if you are sure you want to delete your user:


\section{Forgot password}
\label{\detokenize{Qgis_Plugin/Registration:forgot-password}}
\sphinxAtStartPar
If you forget your password, click on “Reset password” from the settings dialog
box.

\sphinxAtStartPar
A password will be sent to your email. Please check your Junk folder if you
cannot find it within your inbox. The email will come from \sphinxhref{mailto:info@promiseevnts.co}{info@promiseevnts.co}.ke(for testing).

\sphinxAtStartPar
Once you receive your new password, return to the “Settings” screen and use
“Step 2: Enter login” to enter your new pasword.

\noindent{\hspace*{\fill}\sphinxincludegraphics{{forgot_password}.png}\hspace*{\fill}}


\section{Advanced settings}
\label{\detokenize{Qgis_Plugin/Registration:advanced-settings}}
\sphinxAtStartPar
Click “Edit advanced options” to bring up the advanced settings page:

\noindent{\hspace*{\fill}\sphinxincludegraphics{{advanced}.png}\hspace*{\fill}}

\sphinxstepscope


\chapter{Calculate SDG 15.3.1}
\label{\detokenize{Qgis_Plugin/Calculate_sdg15:calculate-sdg-15-3-1}}\label{\detokenize{Qgis_Plugin/Calculate_sdg15::doc}}


\noindent{\hspace*{\fill}\sphinxincludegraphics{{plugin_toolbar_calculate}.png}\hspace*{\fill}}

\sphinxAtStartPar
Sustainable Development Goal 15.3 intends to combat desertification, restore
degraded land and soil, including land affected by desertification, drought and
floods, and strive to achieve a land degradation\sphinxhyphen{}neutral world by 2030. In
order to assess the progress to this goal, the agreed\sphinxhyphen{}upon indicator for SDG
15.3 (proportion of land area degraded) is a combination of three
sub\sphinxhyphen{}indicators: change in land productivity, change in land cover and change
in soil organic carbon.

\begin{sphinxShadowBox}
\sphinxstyletopictitle{Contents}
\begin{itemize}
\item {} 
\sphinxAtStartPar
\phantomsection\label{\detokenize{Qgis_Plugin/Calculate_sdg15:id1}}{\hyperref[\detokenize{Qgis_Plugin/Calculate_sdg15:calculate-sdg-15-3-1}]{\sphinxcrossref{Calculate SDG 15.3.1}}}
\begin{itemize}
\item {} 
\sphinxAtStartPar
\phantomsection\label{\detokenize{Qgis_Plugin/Calculate_sdg15:id2}}{\hyperref[\detokenize{Qgis_Plugin/Calculate_sdg15:calculate-indicators-with-simplified-tool}]{\sphinxcrossref{Calculate indicators with simplified tool}}}

\item {} 
\sphinxAtStartPar
\phantomsection\label{\detokenize{Qgis_Plugin/Calculate_sdg15:id3}}{\hyperref[\detokenize{Qgis_Plugin/Calculate_sdg15:calculate-productivity}]{\sphinxcrossref{Calculate productivity}}}
\begin{itemize}
\item {} 
\sphinxAtStartPar
\phantomsection\label{\detokenize{Qgis_Plugin/Calculate_sdg15:id4}}{\hyperref[\detokenize{Qgis_Plugin/Calculate_sdg15:productivity-trajectory}]{\sphinxcrossref{Productivity Trajectory}}}

\item {} 
\sphinxAtStartPar
\phantomsection\label{\detokenize{Qgis_Plugin/Calculate_sdg15:id5}}{\hyperref[\detokenize{Qgis_Plugin/Calculate_sdg15:productivity-performance}]{\sphinxcrossref{Productivity Performance}}}

\item {} 
\sphinxAtStartPar
\phantomsection\label{\detokenize{Qgis_Plugin/Calculate_sdg15:id6}}{\hyperref[\detokenize{Qgis_Plugin/Calculate_sdg15:productivity-state}]{\sphinxcrossref{Productivity State}}}

\end{itemize}

\item {} 
\sphinxAtStartPar
\phantomsection\label{\detokenize{Qgis_Plugin/Calculate_sdg15:id7}}{\hyperref[\detokenize{Qgis_Plugin/Calculate_sdg15:calculate-land-cover}]{\sphinxcrossref{Calculate land cover}}}

\item {} 
\sphinxAtStartPar
\phantomsection\label{\detokenize{Qgis_Plugin/Calculate_sdg15:id8}}{\hyperref[\detokenize{Qgis_Plugin/Calculate_sdg15:calculate-soil-carbon}]{\sphinxcrossref{Calculate soil carbon}}}

\item {} 
\sphinxAtStartPar
\phantomsection\label{\detokenize{Qgis_Plugin/Calculate_sdg15:id9}}{\hyperref[\detokenize{Qgis_Plugin/Calculate_sdg15:compute-sdg-indicator-15-3-1}]{\sphinxcrossref{Compute SDG Indicator 15.3.1}}}

\end{itemize}

\end{itemize}
\end{sphinxShadowBox}

\sphinxAtStartPar
To select the methods and datasets to calculate these indicators, indicators
click on the calculator icon highlighted above. This will open up the
“Calculate Indicators” dialog box.

\noindent{\hspace*{\fill}\sphinxincludegraphics{{_static/documentation/calculate/LDindicator}.png}\hspace*{\fill}}

\sphinxAtStartPar
Select the Land degradation indicator (SDG indicator 15.3.1) to open the window for this analysis.

\noindent{\hspace*{\fill}\sphinxincludegraphics{{_static/documentation/calculate/SDG15}.png}\hspace*{\fill}}

\sphinxAtStartPar
There are several options for calculating the SDG 15.3.1 Indicator.
MISLAND supports calculating the indicator using the same process as was
used by the UNCCD for the default data provided to countries for the 2018
reporting process. The tool also supports customizing this data, or even
replacing individual datasets with national\sphinxhyphen{}level or other global datasets.
\begin{itemize}
\item {} 
\sphinxAtStartPar
To calculate all three SDG 15.3.1 indicators in one step, using default
settings for most of the indicators, click “Calculate all three indicators in
one step”.

\item {} 
\sphinxAtStartPar
To calculate one of the three SDG 15.3.1 indicators, using customized
settings, or national\sphinxhyphen{}level data, click “Productivity”, “Land cover”, or
“Soil organic carbon”.

\item {} 
\sphinxAtStartPar
To calculate a summary table showing statistics on each of the three
indicators, click “Calculate final SDG 15.3.1 indicator and summary table”.
Note that you must first compute the indicators using one of the above
options.

\item {} 
\sphinxAtStartPar
To calculate a summary table showing statistics on each of the three
indicators for multiple sub\sphinxhyphen{}divisions, click “Calculate area summaries of a raster on sub\sphinxhyphen{}units”.
Note that you must first compute the indicators using one of the above
options.

\end{itemize}

\sphinxAtStartPar
There are three different indicators that are combined to create the SDG 15.3.1
indicator
\begin{itemize}
\item {} 
\sphinxAtStartPar
Productivity: measures the trajectory, performance and state of primary
productivity

\item {} 
\sphinxAtStartPar
Land cover: calculates land cover change relative to a baseline period, enter
a transition matrix indicating which transitions indicate degradation,
stability or improvement.

\item {} 
\sphinxAtStartPar
Soil carbon: compute changes in soil organic carbon as a consequence of
changes in land cover.

\end{itemize}

\sphinxAtStartPar
There are two ways to calculate the indicators: 1) using a simplified tool that
will calculate all three indicators at once, but with limited options for
customization, or 2) using individual tools for each indicator that offer
complete control over how they are calculated.


\section{Calculate indicators with simplified tool}
\label{\detokenize{Qgis_Plugin/Calculate_sdg15:calculate-indicators-with-simplified-tool}}
\sphinxAtStartPar
This tool allows users to calculate all three sub\sphinxhyphen{}indicators in one step.
Select the “Calculate all three sub\sphinxhyphen{}indicators in one step” button.
\begin{enumerate}
\sphinxsetlistlabels{\arabic}{enumi}{enumii}{}{.}%
\item {} 
\sphinxAtStartPar
Select the parameters for Setup. The Period is the Initial and Final year
for the analysis and select one of the two Land Productivity datasets.
Select Next.

\end{enumerate}

\noindent{\hspace*{\fill}\sphinxincludegraphics{{image022}.png}\hspace*{\fill}}
\begin{enumerate}
\sphinxsetlistlabels{\arabic}{enumi}{enumii}{}{.}%
\setcounter{enumi}{1}
\item {} 
\sphinxAtStartPar
Select the Land Cover dataset. The first option is the default ESA dataset.

\end{enumerate}

\noindent{\hspace*{\fill}\sphinxincludegraphics{{image023}.png}\hspace*{\fill}}
\begin{enumerate}
\sphinxsetlistlabels{\arabic}{enumi}{enumii}{}{.}%
\setcounter{enumi}{2}
\item {} 
\sphinxAtStartPar
Select Edit definition to change the aggregation from the ESA Land Cover
dataset into 7 classes.

\end{enumerate}

\noindent{\hspace*{\fill}\sphinxincludegraphics{{image024}.png}\hspace*{\fill}}

\sphinxAtStartPar
The second option allows users to upload a custom land cover dataset. This
requires two datasets to compare change over time. Select Next.

\noindent{\hspace*{\fill}\sphinxincludegraphics{{image025}.png}\hspace*{\fill}}
\begin{enumerate}
\sphinxsetlistlabels{\arabic}{enumi}{enumii}{}{.}%
\setcounter{enumi}{3}
\item {} 
\sphinxAtStartPar
The user can now define the effects of land cover change and how it is
classified as degrading or improving.

\end{enumerate}

\noindent{\hspace*{\fill}\sphinxincludegraphics{{image026}.png}\hspace*{\fill}}
\begin{enumerate}
\sphinxsetlistlabels{\arabic}{enumi}{enumii}{}{.}%
\setcounter{enumi}{4}
\item {} 
\sphinxAtStartPar
Select an area to run the analysis or upload a shapefile boundary

\end{enumerate}

\begin{sphinxadmonition}{note}{Note:}
\sphinxAtStartPar
The provided boundaries are from \sphinxhref{http://www.naturalearthdata.com}{Natural Earth}, and are in the \sphinxhref{https://creativecommons.org/publicdomain}{public domain}. The boundaries and names
used, and the designations used, in MISLAND do not imply official
endorsement or acceptance by Conservation International Foundation, or by
its partner organizations and contributors.

\sphinxAtStartPar
If using MISLAND for official purposes, it is recommended that users
choose an official boundary provided by the designated office of their
country.
\end{sphinxadmonition}

\noindent{\hspace*{\fill}\sphinxincludegraphics{{image027}.png}\hspace*{\fill}}
\begin{enumerate}
\sphinxsetlistlabels{\arabic}{enumi}{enumii}{}{.}%
\setcounter{enumi}{5}
\item {} 
\sphinxAtStartPar
Name the task and make notes for future reference

\item {} 
\sphinxAtStartPar
Click on “Calculate” to submit your task to Google Earth Engine

\end{enumerate}

\noindent{\hspace*{\fill}\sphinxincludegraphics{{image028}.png}\hspace*{\fill}}


\section{Calculate productivity}
\label{\detokenize{Qgis_Plugin/Calculate_sdg15:calculate-productivity}}
\sphinxAtStartPar
Productivity measures the trajectory, performance and state of primary
productivity using either 8km AVHRR, 250m MODIS or 30m LANDSAT 7 (under development) datasets. The user can select
one or multiple indicators to calculate, the NDVI dataset, name the tasks and
enter in explanatory notes for their intended reporting area.


\subsection{Productivity Trajectory}
\label{\detokenize{Qgis_Plugin/Calculate_sdg15:productivity-trajectory}}
\sphinxAtStartPar
Trajectory assesses the rate of change of productivity over time. To calculate
trajectory:
\begin{enumerate}
\sphinxsetlistlabels{\arabic}{enumi}{enumii}{}{)}%
\item {} 
\sphinxAtStartPar
Select an indicator to calculate

\item {} 
\sphinxAtStartPar
Select NDVI dataset to use and select Next

\end{enumerate}

\begin{sphinxadmonition}{note}{Note:}
\sphinxAtStartPar
The valid date range is set by the NDVI dataset selected within the first
tab: AVHRR dates compare 1982\sphinxhyphen{}2015 and MODIS 2001\sphinxhyphen{}2016.
\end{sphinxadmonition}

\noindent{\hspace*{\fill}\sphinxincludegraphics{{image029}.png}\hspace*{\fill}}
\begin{enumerate}
\sphinxsetlistlabels{\arabic}{enumi}{enumii}{}{)}%
\setcounter{enumi}{2}
\item {} 
\sphinxAtStartPar
In the tab “Advanced”, select the method to be used to compute the
productivity trajectory analysis. The options are:

\end{enumerate}
\begin{itemize}
\item {} 
\sphinxAtStartPar
\sphinxstylestrong{NDVI trend}: This dataset shows the trend in annually integrated NDVI time
series (2001\sphinxhyphen{}2015) using MODIS (250m) dataset (MOD13Q1) or AVHRR (8km;
GIMMS3g.v1). The normalized difference vegetation index (NDVI) is the ratio
of the difference between near\sphinxhyphen{}infrared band (NIR) and the red band (RED) and
the sum of these two bands (Rouse et al., 1974; Deering 1978) and reviewed in
Tucker (1979).

\item {} 
\sphinxAtStartPar
\sphinxstylestrong{RUE}: is defined as the ratio between net primary production (NPP), in
this case annual integrals of NDVI, and rainfall. It has been increasingly
used to analyze the variability of vegetation production in arid and
semi\sphinxhyphen{}arid biomes, where rainfall is a major limiting factor for plant growth

\item {} 
\sphinxAtStartPar
\sphinxstylestrong{RESTREND}: this method attempts to adjust the NDVI signals from the effect
of particular climatic drivers, such as rainfall or soil moisture, using a
pixel\sphinxhyphen{}by\sphinxhyphen{}pixel linear regression on the NDVI time series and the climate
signal. The linear model and the climatic data is used then to predict NDVI,
and to compute the residuals between the observed and climate\sphinxhyphen{}predicted NDVI
annual integrals. The NDVI residual trend is finally plotted to spatially
represent overall trends in primary productivity independent of climate.

\item {} 
\sphinxAtStartPar
\sphinxstylestrong{WUE}: is defined as the ratio between net primary production (NPP), in
this case annual integrals of NDVI, and evapotranspiration.

\end{itemize}

\noindent{\hspace*{\fill}\sphinxincludegraphics{{image030}.png}\hspace*{\fill}}


\subsection{Productivity Performance}
\label{\detokenize{Qgis_Plugin/Calculate_sdg15:productivity-performance}}
\sphinxAtStartPar
Performance is a comparison of how productivity in an area compares to
productivity in similar areas at the same point in time. To calculate
performance:
\begin{enumerate}
\sphinxsetlistlabels{\arabic}{enumi}{enumii}{}{)}%
\item {} 
\sphinxAtStartPar
Select the start and end year of the period of analysis  for comparison.

\end{enumerate}


\subsection{Productivity State}
\label{\detokenize{Qgis_Plugin/Calculate_sdg15:productivity-state}}
\sphinxAtStartPar
State performs a comparison of how current productivity in an area compares to
past productivity. To calculate state:
\begin{enumerate}
\sphinxsetlistlabels{\arabic}{enumi}{enumii}{}{)}%
\item {} 
\sphinxAtStartPar
Define the baseline and comparison periods for the computation of the State
sub\sphinxhyphen{}indicator.

\end{enumerate}

\sphinxAtStartPar
The next step is to define the study area on which to perform the analysis. The
tool allows selecting the area of interest in one of two ways:
\begin{enumerate}
\sphinxsetlistlabels{\arabic}{enumi}{enumii}{}{.}%
\item {} 
\sphinxAtStartPar
Selects first (i.e. country) and/or second (i.e. province or state)
administrative boundary from a drop\sphinxhyphen{}down menu.

\item {} 
\sphinxAtStartPar
The user can provide a shapefile, KML, or geojson defining an area of
interest. Once this is done, Select Next.

\end{enumerate}

\noindent{\hspace*{\fill}\sphinxincludegraphics{{image031}.png}\hspace*{\fill}}
\begin{enumerate}
\sphinxsetlistlabels{\arabic}{enumi}{enumii}{}{.}%
\setcounter{enumi}{2}
\item {} 
\sphinxAtStartPar
The next step is to write a Task name and some notes to indicate which
options were selected for the analysis.

\end{enumerate}

\noindent{\hspace*{\fill}\sphinxincludegraphics{{image032}.png}\hspace*{\fill}}
\begin{enumerate}
\sphinxsetlistlabels{\arabic}{enumi}{enumii}{}{.}%
\setcounter{enumi}{3}
\item {} 
\sphinxAtStartPar
When all the parameters have been defined, click “Calculate”, and the task
will be submitted to Google Earth Engine for computing. When the task is
completed (processing time will vary depending on server usage, but for most
countries it takes only a few minutes most of the time), you’ll receive an
email notifying the successful completion.

\item {} 
\sphinxAtStartPar
When the Google Earth Engine task has completed and you received the email,
click “Refresh List” and the status will show FINISHED. Click on the task
and select “Download results” at the bottom of the window. A pop up window
will open for you to select where to save the layer and to assign it a name.
Then click “Save”. The layer will be saved on your computer and
automatically loaded into yoour current QGIS project.

\end{enumerate}

\noindent{\hspace*{\fill}\sphinxincludegraphics{{output_productivity}.png}\hspace*{\fill}}


\section{Calculate land cover}
\label{\detokenize{Qgis_Plugin/Calculate_sdg15:calculate-land-cover}}
\sphinxAtStartPar
Changes in land cover is one of the indicators used to track potential land
degradation which need to be reported to the UNCCD and to track progress
towards SDG 15.3.1. While some land cover transitions indicate, in most cases,
processes of land degradation, the interpretation of those transitions are for
the most part context specific. For that reason, this indicator requires the
input of the user to identify which changes in land cover will be considered as
degradation, improvement or no change in terms of degradation. The toolbox
allows users to calculate land cover change relative to a baseline period,
enter a transition matrix indicating which transitions indicate degradation,
stability or improvement.

\sphinxAtStartPar
To calculate the land cover change indicator:
\begin{enumerate}
\sphinxsetlistlabels{\arabic}{enumi}{enumii}{}{.}%
\item {} 
\sphinxAtStartPar
Click on the Calculate Indicators button from the toolbox bar, then select
Land cover.

\end{enumerate}

\noindent{\hspace*{\fill}\sphinxincludegraphics{{image033}.png}\hspace*{\fill}}
\begin{enumerate}
\sphinxsetlistlabels{\arabic}{enumi}{enumii}{}{.}%
\setcounter{enumi}{1}
\item {} 
\sphinxAtStartPar
Within the “Land Cover Setup tab” the user selects the baseline and target years

\end{enumerate}

\noindent{\hspace*{\fill}\sphinxincludegraphics{{image034}.png}\hspace*{\fill}}
\begin{enumerate}
\sphinxsetlistlabels{\arabic}{enumi}{enumii}{}{.}%
\setcounter{enumi}{2}
\item {} 
\sphinxAtStartPar
The land cover aggregation can be customized using the ‘Edit definition’
button. The user can define their own aggregation of land cover classes from
the 37 ESA land cover classes to the 7 UNCCD categories.
\begin{enumerate}
\sphinxsetlistlabels{\Alph}{enumii}{enumiii}{}{.}%
\item {} 
\sphinxAtStartPar
Select the dial button for the “Custom” option and select “Create new
definition”

\item {} 
\sphinxAtStartPar
Edit the aggregation suitable for the area of interest

\item {} 
\sphinxAtStartPar
Select “Save definition” and select Next

\end{enumerate}

\end{enumerate}

\noindent{\hspace*{\fill}\sphinxincludegraphics{{image035}.png}\hspace*{\fill}}
\begin{enumerate}
\sphinxsetlistlabels{\arabic}{enumi}{enumii}{}{.}%
\setcounter{enumi}{3}
\item {} 
\sphinxAtStartPar
Within the “Define Degradation tab” user define the meaning of each land
cover transition in terms of degradation. The options are: stable (0),
degradation (\sphinxhyphen{}) or improvement (+). For example, the default for cropland to
cropland is 0 because the land cover stays the same and is therefore stable.
The default for forest to cropland is \sphinxhyphen{}1 because forest is likely cut to
clear way for agriculture and would be considered deforestation. The user is
encouraged to thoroughly evaluate the meaning of each transition based on
their knowledge of the study area, since this matrix will have an important
effect on the land degradation identified by this subindicator.

\sphinxAtStartPar
Users can keep the default values or create unique transition values of
their own.

\end{enumerate}

\noindent{\hspace*{\fill}\sphinxincludegraphics{{image036}.png}\hspace*{\fill}}
\begin{enumerate}
\sphinxsetlistlabels{\arabic}{enumi}{enumii}{}{.}%
\setcounter{enumi}{4}
\item {} 
\sphinxAtStartPar
The next step is to define the study area on which to perform the analysis.
The toolbox allows this task to be completed in one of two ways:
\begin{enumerate}
\sphinxsetlistlabels{\Alph}{enumii}{enumiii}{}{.}%
\item {} 
\sphinxAtStartPar
The user selects first (i.e. country) and second (i.e. province or state)
administrative boundary from a drop\sphinxhyphen{}down menu.

\item {} 
\sphinxAtStartPar
The user can upload a shapefile with an area of interest.

\end{enumerate}

\end{enumerate}

\noindent{\hspace*{\fill}\sphinxincludegraphics{{image037}.png}\hspace*{\fill}}
\begin{enumerate}
\sphinxsetlistlabels{\arabic}{enumi}{enumii}{}{.}%
\setcounter{enumi}{5}
\item {} 
\sphinxAtStartPar
The next step is to add the task name and relevant notes for the analysis.

\end{enumerate}

\noindent{\hspace*{\fill}\sphinxincludegraphics{{image038}.png}\hspace*{\fill}}
\begin{enumerate}
\sphinxsetlistlabels{\arabic}{enumi}{enumii}{}{.}%
\setcounter{enumi}{6}
\item {} 
\sphinxAtStartPar
When all the parameters have been defined, click “Calculate”, and the task
will be submitted to Google Earth Engine for computing. When the task is
completed (processing time will vary depending on server usage, but for most
countries it takes only a few minutes most of the time), you’ll receive an
email notifying the successful completion.

\item {} 
\sphinxAtStartPar
When the Google Earth Engine task has completed and you received the email,
click “Refresh List” and the status will show FINISHED. Click on the task
and select “Download results” at the bottom of the window. A pop up window
will open for you to select where to save the layer and to assign it a name.
Then click “Save”. The layer will be saved on your computer and
automatically loaded into yoour current QGIS project.

\end{enumerate}

\noindent{\hspace*{\fill}\sphinxincludegraphics{{output_landcover}.png}\hspace*{\fill}}


\section{Calculate soil carbon}
\label{\detokenize{Qgis_Plugin/Calculate_sdg15:calculate-soil-carbon}}
\sphinxAtStartPar
Soil Organic Carbon is calculated as a proxy for carbon stocks. It is measured
using soil data and changes in land cover.

\sphinxAtStartPar
To calculate degradation in soil organic carbon:

\noindent{\hspace*{\fill}\sphinxincludegraphics{{image039}.png}\hspace*{\fill}}
\begin{enumerate}
\sphinxsetlistlabels{\arabic}{enumi}{enumii}{}{.}%
\item {} 
\sphinxAtStartPar
Select Soil organic carbon button under Calculate Indicators

\end{enumerate}

\noindent{\hspace*{\fill}\sphinxincludegraphics{{image040}.png}\hspace*{\fill}}
\begin{enumerate}
\sphinxsetlistlabels{\arabic}{enumi}{enumii}{}{.}%
\setcounter{enumi}{1}
\item {} 
\sphinxAtStartPar
The Land Cover Setup tab allows the user to define the period for analysis
with the baseline and target year. Users can select the Edit definition
button to change the land cover aggregation method or upload a datasets.

\end{enumerate}

\noindent{\hspace*{\fill}\sphinxincludegraphics{{image041}.png}\hspace*{\fill}}
\begin{enumerate}
\sphinxsetlistlabels{\arabic}{enumi}{enumii}{}{.}%
\setcounter{enumi}{2}
\item {} 
\sphinxAtStartPar
The “Advanced” tab allows users to specify the Climate regime.

\end{enumerate}

\noindent{\hspace*{\fill}\sphinxincludegraphics{{image042}.png}\hspace*{\fill}}
\begin{enumerate}
\sphinxsetlistlabels{\arabic}{enumi}{enumii}{}{.}%
\setcounter{enumi}{3}
\item {} 
\sphinxAtStartPar
Users can select an area or upload a polygon shapefile for analysis

\end{enumerate}

\noindent{\hspace*{\fill}\sphinxincludegraphics{{image043}.png}\hspace*{\fill}}
\begin{enumerate}
\sphinxsetlistlabels{\arabic}{enumi}{enumii}{}{.}%
\setcounter{enumi}{5}
\item {} 
\sphinxAtStartPar
The next step is to add the task name and relevant notes for the analysis.

\item {} 
\sphinxAtStartPar
When all the parameters have been defined, click “Calculate”, and the task
will be submitted to Google Earth Engine for computing. When the task is
completed (processing time will vary depending on server usage, but for most
countries it takes only a few minutes most of the time), you’ll receive an
email notifying the successful completion.

\item {} 
\sphinxAtStartPar
When the Google Earth Engine task has completed and you received the email,
click “Refresh List” and the status will show FINISHED. Click on the task
and select “Download results” at the bottom of the window. A pop up window
will open for you to select where to save the layer and to assign it a name.
Then click “Save”. The layer will be saved on your computer and
automatically loaded into your current QGIS project.

\end{enumerate}

\noindent{\hspace*{\fill}\sphinxincludegraphics{{output_soc}.png}\hspace*{\fill}}


\section{Compute SDG Indicator 15.3.1}
\label{\detokenize{Qgis_Plugin/Calculate_sdg15:compute-sdg-indicator-15-3-1}}\begin{enumerate}
\sphinxsetlistlabels{\arabic}{enumi}{enumii}{}{.}%
\item {} 
\sphinxAtStartPar
Once you have computed the three sub\sphinxhyphen{}indicators (productivity, land cover
and soil organic carbon), and they are loaded into the QGIS project. Click
on the Calculate icon. This will open up the “Calculate
Indicator” dialog box. This time click on Step 2 “Calculate final SDG 15.3.1
indicator and summary table”.

\item {} 
\sphinxAtStartPar
The input window will open already populated with the correct sub\sphinxhyphen{}indicators
(that if you have them loaded to the QGIS map)

\end{enumerate}

\noindent{\hspace*{\fill}\sphinxincludegraphics{{sdg_input}.png}\hspace*{\fill}}
\begin{enumerate}
\sphinxsetlistlabels{\arabic}{enumi}{enumii}{}{.}%
\setcounter{enumi}{2}
\item {} 
\sphinxAtStartPar
Select the name and location where to save the output raster layer and the
excel file with the areas computed.

\end{enumerate}

\noindent{\hspace*{\fill}\sphinxincludegraphics{{sdg_output}.png}\hspace*{\fill}}
\begin{enumerate}
\sphinxsetlistlabels{\arabic}{enumi}{enumii}{}{.}%
\setcounter{enumi}{3}
\item {} 
\sphinxAtStartPar
Define the area of analysis. In this example, the country boundary.

\end{enumerate}

\noindent{\hspace*{\fill}\sphinxincludegraphics{{sdg_area}.png}\hspace*{\fill}}
\begin{enumerate}
\sphinxsetlistlabels{\arabic}{enumi}{enumii}{}{.}%
\setcounter{enumi}{4}
\item {} 
\sphinxAtStartPar
Give a name to the task and click “Calculate”

\end{enumerate}

\noindent{\hspace*{\fill}\sphinxincludegraphics{{sdg_options}.png}\hspace*{\fill}}
\begin{enumerate}
\sphinxsetlistlabels{\arabic}{enumi}{enumii}{}{.}%
\setcounter{enumi}{5}
\item {} 
\sphinxAtStartPar
This calculation is run on your computer, so depending on the size of the
area and the computing power of your computer, it could take a few minutes.
When completed, the final SDG indicator will be loaded into the QGIS map and
the Excel file with the areas will be saved in the folder you selected. when
done, a message will pop up.

\end{enumerate}

\noindent{\hspace*{\fill}\sphinxincludegraphics{{sdg_done}.png}\hspace*{\fill}}
\begin{enumerate}
\sphinxsetlistlabels{\arabic}{enumi}{enumii}{}{.}%
\setcounter{enumi}{6}
\item {} 
\sphinxAtStartPar
Click OK and two layers will be loaded to your map: the \sphinxstylestrong{5 classes
productivity} and the \sphinxstylestrong{SDG 15.3.1} indicators.

\end{enumerate}

\noindent{\hspace*{\fill}\sphinxincludegraphics{{sdg_maps}.png}\hspace*{\fill}}
\begin{enumerate}
\sphinxsetlistlabels{\arabic}{enumi}{enumii}{}{.}%
\setcounter{enumi}{7}
\item {} 
\sphinxAtStartPar
If you navigate to the folder you selected for storing the files, you can
open the Excel files with the areas computed for each of the sub\sphinxhyphen{}indicators
and the final SDG. NOTE: You may get an error message when opening the file,
just click ok and the file will open regardless. We are working to fix this
error.

\end{enumerate}

\noindent{\hspace*{\fill}\sphinxincludegraphics{{sdg_excel}.png}\hspace*{\fill}}

\sphinxstepscope


\chapter{Calculate Vegetation Degradation}
\label{\detokenize{Qgis_Plugin/Calculate_vegetation:calculate-vegetation-degradation}}\label{\detokenize{Qgis_Plugin/Calculate_vegetation::doc}}

\section{Compute Vegatation Indices}
\label{\detokenize{Qgis_Plugin/Calculate_vegetation:compute-vegatation-indices}}


\sphinxAtStartPar
Land degradation hotspots (LDH) are produced via the analysis of time\sphinxhyphen{}series
vegetation indices data and are used to characterize areas of different sizes,
where the vegetation cover and the soil types are severely degraded. Vegetation
loss/gain hotspots will be calculated based on time series observation of selected
suit of vegetation indices depending on the climatic zones and terrain morphology
of the North African countries

\sphinxAtStartPar
Vegation Indices computed from Landsat 7 ETM+ include:
\begin{enumerate}
\sphinxsetlistlabels{\arabic}{enumi}{enumii}{}{.}%
\item {} 
\sphinxAtStartPar
\sphinxstylestrong{NDVI (humid, sub\sphinxhyphen{}humid and semi\sphinxhyphen{}arid zones)}

\sphinxAtStartPar
DVI is preferable for global vegetation monitoring since it helps to compensate for
changes in lighting conditions, surface slope, exposure, and other external factors.
NDVI is calculated in accordance with the formula:

\noindent{\hspace*{\fill}\sphinxincludegraphics{{ndvi}.png}\hspace*{\fill}}

\sphinxAtStartPar
NIR \textendash{} reflection in the near\sphinxhyphen{}infrared spectrum
RED \textendash{} reflection in the red range of the spectrum

\sphinxAtStartPar
According to this formula, the density of vegetation (NDVI) at a certain point of the
image is equal to the difference in the intensities of reflected light in the red and
infrared range divided by the sum of these intensities.

\sphinxAtStartPar
This index defines values ​​from \sphinxhyphen{}1.0 to 1.0, basically representing greens, where negative
values ​​are mainly formed from clouds, water and snow, and values ​​close to zero are
primarily formed from rocks and bare soil. Very small values ​​(0.1 or less) of the NDVI
function correspond to empty areas of rocks, sand or snow. Moderate values ​​(from 0.2 to 0.3)
represent shrubs and meadows, while large values ​​(from 0.6 to 0.8) indicate temperate and
tropical forests.

\item {} 
\sphinxAtStartPar
\sphinxstylestrong{MSAVI2 (arid and stepic zones)}

\sphinxAtStartPar
MSAVI2 is soil adjusted vegetation indices that seek to address some of the limitation of
NDVI when applied to areas with a high degree of exposed soil surface.It eliminates the need
to find the soil line from a feature\sphinxhyphen{}space plot or even explicitly specify the soil brightness
correction factor:

\noindent{\hspace*{\fill}\sphinxincludegraphics{{msavi2}.png}\hspace*{\fill}}

\item {} 
\sphinxAtStartPar
\sphinxstylestrong{SAVI (desert areas)}

\sphinxAtStartPar
SAVI is used to correct Normalized Difference Vegetation Index (NDVI) for the influence of
soil brightness in areas where vegetative cover is low. Landsat Surface Reflectance\sphinxhyphen{}derived
SAVI is calculated as a ratio between the R and NIR values with a soil brightness correction
factor (L) defined as 0.5 to accommodate most land cover types.

\noindent{\hspace*{\fill}\sphinxincludegraphics{{savi}.png}\hspace*{\fill}}

\end{enumerate}

\sphinxAtStartPar
To compute the above vegetation indices, click on the calculator icon . This will open up the
“Calculate Indicators” dialog box.

\noindent{\hspace*{\fill}\sphinxincludegraphics{{plugin_toolbar_calculate}.png}\hspace*{\fill}}

\sphinxAtStartPar
Select the “Vegetation Indices Time\sphinxhyphen{}Series” to open the window for this analysis.

\noindent{\hspace*{\fill}\sphinxincludegraphics{{vegetation_indices}.png}\hspace*{\fill}}

\sphinxAtStartPar
From the list of vegetation indices provided select your desired index and provide a title to your
plot. Select the area of interest i.e point or polygon, label your task and calculate the index.

\noindent{\hspace*{\fill}\sphinxincludegraphics{{vegetation_indices_calc}.png}\hspace*{\fill}}

\sphinxAtStartPar
To view your final plot go to “Download results from Earth Engine” and refresh the list, then select
the task and download the results. This will plot a graph of your index over time.

\noindent{\hspace*{\fill}\sphinxincludegraphics{{vegetation_indices_task}.png}\hspace*{\fill}}

\noindent{\hspace*{\fill}\sphinxincludegraphics{{vegetation_indices_plot}.png}\hspace*{\fill}}

\sphinxstepscope


\chapter{Calculate Forest Degradation}
\label{\detokenize{Qgis_Plugin/Calculate_forest:calculate-forest-degradation}}\label{\detokenize{Qgis_Plugin/Calculate_forest::doc}}
\begin{sphinxShadowBox}
\sphinxstyletopictitle{Contents}
\begin{itemize}
\item {} 
\sphinxAtStartPar
\phantomsection\label{\detokenize{Qgis_Plugin/Calculate_forest:id1}}{\hyperref[\detokenize{Qgis_Plugin/Calculate_forest:calculate-forest-degradation}]{\sphinxcrossref{Calculate Forest Degradation}}}
\begin{itemize}
\item {} 
\sphinxAtStartPar
\phantomsection\label{\detokenize{Qgis_Plugin/Calculate_forest:id2}}{\hyperref[\detokenize{Qgis_Plugin/Calculate_forest:compute-forest-fires}]{\sphinxcrossref{Compute Forest Fires}}}

\item {} 
\sphinxAtStartPar
\phantomsection\label{\detokenize{Qgis_Plugin/Calculate_forest:id3}}{\hyperref[\detokenize{Qgis_Plugin/Calculate_forest:compute-forest-change-and-total-carbon-summary}]{\sphinxcrossref{Compute Forest Change and Total Carbon \& Summary}}}
\begin{itemize}
\item {} 
\sphinxAtStartPar
\phantomsection\label{\detokenize{Qgis_Plugin/Calculate_forest:id4}}{\hyperref[\detokenize{Qgis_Plugin/Calculate_forest:step-1-compute-forest-change-and-total-carbon}]{\sphinxcrossref{Step 1: Compute Forest Change and Total Carbon}}}

\item {} 
\sphinxAtStartPar
\phantomsection\label{\detokenize{Qgis_Plugin/Calculate_forest:id5}}{\hyperref[\detokenize{Qgis_Plugin/Calculate_forest:step-2-generate-carbon-change-summary}]{\sphinxcrossref{Step 2: Generate Carbon Change Summary}}}

\end{itemize}

\end{itemize}

\end{itemize}
\end{sphinxShadowBox}




\section{Compute Forest Fires}
\label{\detokenize{Qgis_Plugin/Calculate_forest:compute-forest-fires}}
\sphinxAtStartPar
Burnt areas and forest fires are be highlighted and mapped out form remotely sensed \sphinxstylestrong{Landsat 8 /Sentinel 2}
data using the Normalized Burn Ratio (NBR). NBR is designed to highlight burned areas and estimate burn
severity. It uses near\sphinxhyphen{}infrared (NIR) and shortwave\sphinxhyphen{}infrared (SWIR) wavelengths. Before fire events,
healthy vegetation has very high NIR reflectance and a low SWIR reflectance. In contrast, recently
burned areas show low reflectance in the NIR and high reflectance in the SWIR band.

\sphinxAtStartPar
The NBR is be calculated for Landsat/Sentinel images before the fire (pre\sphinxhyphen{}fire NBR) and after
the fire (post\sphinxhyphen{}fire NBR). The \sphinxstylestrong{difference between the pre\sphinxhyphen{}fire NBR and the post\sphinxhyphen{}fire NBR} referred
to as \sphinxstylestrong{delta NBR (dNBR)} is computed to highlight the areas of forest disturbance by fire event.

\sphinxAtStartPar
Classification of the dNBR is be used for burn severity assessment, as areas with higher dNBR
values indicate more severe damage whereas areas with negative dNBR values might show increased
vegetation productivity. dNBR is classified according to burn severity ranges proposed by
the United States Geological Survey(USGS)

\sphinxAtStartPar
To compute the above forest fires, click on the calculator icon . This will open up the
“Calculate Indicators” dialog box.

\noindent{\hspace*{\fill}\sphinxincludegraphics{{plugin_toolbar_calculate}.png}\hspace*{\fill}}

\sphinxAtStartPar
Select the “Forest Degradation Hotspots” to open the window for this analysis then select Forest Fires.

\noindent{\hspace*{\fill}\sphinxincludegraphics{{forest_degradation}.png}\hspace*{\fill}}

\sphinxAtStartPar
Select either “Landsat 8” or  “Sentinel 2”, a Pre\sphinxhyphen{}fire and Post\sphinxhyphen{}fire perio, the area of interest then calculate
your parameters.

\noindent{\hspace*{\fill}\sphinxincludegraphics{{forest_degradation_dataset}.png}\hspace*{\fill}}

\sphinxAtStartPar
To view your final result go to “Download results from Earth Engine” and refresh the list, then select
the task and download the results. This will add 3 datasets to the map view i.e Prefire NBR, Postfire NBR and dNBR imagery.

\noindent{\hspace*{\fill}\sphinxincludegraphics{{vegetation_indices}.png}\hspace*{\fill}}


\section{Compute Forest Change and Total Carbon \& Summary}
\label{\detokenize{Qgis_Plugin/Calculate_forest:compute-forest-change-and-total-carbon-summary}}
\sphinxAtStartPar
The quantification of the forest gain/loss hotspots will be based on pre\sphinxhyphen{}existing high\sphinxhyphen{}resolution
global maps derived from Hansen Global Forest change dataset that can be accessed using Google
Earth Engine API. The maps are produced from time\sphinxhyphen{}series analysis of Landsat images characterizing
forest extent and change over time.


\subsection{Step 1: Compute Forest Change and Total Carbon}
\label{\detokenize{Qgis_Plugin/Calculate_forest:step-1-compute-forest-change-and-total-carbon}}
\sphinxAtStartPar
To compute Forest Change and Total Carbon, click on the calculator icon . This will open up the
“Calculate Indicators” dialog box.

\sphinxAtStartPar
Select Forest Change and Total Carbon and select \sphinxstylestrong{Step 1}, calculate Forest Change and Total Carbon
to open the window for this analysis.

\noindent{\hspace*{\fill}\sphinxincludegraphics{{vegetation_indices}.png}\hspace*{\fill}}

\noindent{\hspace*{\fill}\sphinxincludegraphics{{forest_carbon}.png}\hspace*{\fill}}

\noindent{\hspace*{\fill}\sphinxincludegraphics{{forest_carbon_1}.png}\hspace*{\fill}}

\sphinxAtStartPar
Provide an Initial and Target year for the Hansen Global Forest Change dataset. Also provide
a value considered forest cover percentage.

\noindent{\hspace*{\fill}\sphinxincludegraphics{{forest_carbon_1_year}.png}\hspace*{\fill}}

\sphinxAtStartPar
Next select the above ground biomass dataset to be used and the method for calculating the root to shoot
ratio. Procees to select the area of interest and label your task then calculate.

\noindent{\hspace*{\fill}\sphinxincludegraphics{{forest_carbon_1_method}.png}\hspace*{\fill}}

\sphinxAtStartPar
To view your final result go to “Download results from Earth Engine” and refresh the list, then select
the task and download the results. This will add 2 datasets to the map view i.e Total carbon and Forest loss

\noindent{\hspace*{\fill}\sphinxincludegraphics{{forest_fire_result}.png}\hspace*{\fill}}


\subsection{Step 2: Generate Carbon Change Summary}
\label{\detokenize{Qgis_Plugin/Calculate_forest:step-2-generate-carbon-change-summary}}
\sphinxAtStartPar
To generate a carbon change summary, select \sphinxstylestrong{Step 2}, Calculate carbon change summary table to open the window
for analysis.

\noindent{\hspace*{\fill}\sphinxincludegraphics{{forest_carbon_2}.png}\hspace*{\fill}}

\sphinxAtStartPar
Provide the input datasets generated from step 1 (Auto\sphinxhyphen{}detected if already loaded onto the map view).

\noindent{\hspace*{\fill}\sphinxincludegraphics{{forest_carbon_2_input}.png}\hspace*{\fill}}

\sphinxAtStartPar
Set the output location for the summary table file, select the same area of interest as in step 1 and calculate.

\noindent{\hspace*{\fill}\sphinxincludegraphics{{forest_carbon_2_output}.png}\hspace*{\fill}}

\sphinxAtStartPar
A summary file in xlsx format will be generated on completion similar to the one shown below.

\noindent{\hspace*{\fill}\sphinxincludegraphics{{forest_carbon_excel}.png}\hspace*{\fill}}

\sphinxstepscope


\chapter{Calculate Desertification (MEDALUS)}
\label{\detokenize{Qgis_Plugin/Calculate_medalus:calculate-desertification-medalus}}\label{\detokenize{Qgis_Plugin/Calculate_medalus::doc}}
\sphinxAtStartPar
The Mediterranean Desertification and Land Use (MEDALUS) is the name of a project supported
by Europe to assess, model and understand the desertification phenomena that increasingly
affect the Mediterranean area. It provides a satisfied result about land degradation vulnerability.

\sphinxAtStartPar
The MEDALUS approach identifies environmentally sensitive areas (ESAs) through the Environmentally
Sensitive Area Index (ESAI). This index can be used to obtain an in\sphinxhyphen{}depth understanding of the parameters
causing the desertification threat at a certain point. This approach is simple, robust, widely applicable,
and acceptable to new indicators and parameters and can be adjusted to several level scales. , the method
was used for the analysis of the main indicators identified to be driving forces of land degradation.



\begin{sphinxadmonition}{note}{Note:}
\sphinxAtStartPar
Maintain the same area of interest for all computations within MEDALUS.
\end{sphinxadmonition}

\begin{sphinxShadowBox}
\sphinxstyletopictitle{Contents}
\begin{itemize}
\item {} 
\sphinxAtStartPar
\phantomsection\label{\detokenize{Qgis_Plugin/Calculate_medalus:id3}}{\hyperref[\detokenize{Qgis_Plugin/Calculate_medalus:calculate-desertification-medalus}]{\sphinxcrossref{Calculate Desertification (MEDALUS)}}}
\begin{itemize}
\item {} 
\sphinxAtStartPar
\phantomsection\label{\detokenize{Qgis_Plugin/Calculate_medalus:id4}}{\hyperref[\detokenize{Qgis_Plugin/Calculate_medalus:soil-quality-index-sqi}]{\sphinxcrossref{1. Soil Quality Index (SQI)}}}
\begin{itemize}
\item {} 
\sphinxAtStartPar
\phantomsection\label{\detokenize{Qgis_Plugin/Calculate_medalus:id5}}{\hyperref[\detokenize{Qgis_Plugin/Calculate_medalus:a-using-default-data-computed-on-google-earth-engine}]{\sphinxcrossref{a. Using default data (Computed on Google Earth Engine)}}}

\item {} 
\sphinxAtStartPar
\phantomsection\label{\detokenize{Qgis_Plugin/Calculate_medalus:id6}}{\hyperref[\detokenize{Qgis_Plugin/Calculate_medalus:b-using-custom-data-computed-locally-on-device}]{\sphinxcrossref{b. Using Custom data (Computed locally on device)}}}
\begin{itemize}
\item {} 
\sphinxAtStartPar
\phantomsection\label{\detokenize{Qgis_Plugin/Calculate_medalus:id7}}{\hyperref[\detokenize{Qgis_Plugin/Calculate_medalus:raw-data-download}]{\sphinxcrossref{Raw Data Download}}}

\item {} 
\sphinxAtStartPar
\phantomsection\label{\detokenize{Qgis_Plugin/Calculate_medalus:id8}}{\hyperref[\detokenize{Qgis_Plugin/Calculate_medalus:extracting-soil-drainage-soil-texture-and-rock-fragment-layers-from-hwsd-data}]{\sphinxcrossref{Extracting Soil Drainage, Soil Texture and Rock Fragment Layers from HWSD DATA}}}

\item {} 
\sphinxAtStartPar
\phantomsection\label{\detokenize{Qgis_Plugin/Calculate_medalus:id9}}{\hyperref[\detokenize{Qgis_Plugin/Calculate_medalus:compute-soil-quality-index}]{\sphinxcrossref{Compute Soil Quality Index}}}

\end{itemize}

\end{itemize}

\item {} 
\sphinxAtStartPar
\phantomsection\label{\detokenize{Qgis_Plugin/Calculate_medalus:id10}}{\hyperref[\detokenize{Qgis_Plugin/Calculate_medalus:vegetation-quality-index-vqi}]{\sphinxcrossref{2. Vegetation Quality Index (VQI)}}}

\item {} 
\sphinxAtStartPar
\phantomsection\label{\detokenize{Qgis_Plugin/Calculate_medalus:id11}}{\hyperref[\detokenize{Qgis_Plugin/Calculate_medalus:climate-quality-index-cqi}]{\sphinxcrossref{3. Climate Quality Index (CQI)}}}
\begin{itemize}
\item {} 
\sphinxAtStartPar
\phantomsection\label{\detokenize{Qgis_Plugin/Calculate_medalus:id12}}{\hyperref[\detokenize{Qgis_Plugin/Calculate_medalus:id1}]{\sphinxcrossref{a. Using default data (Computed on Google Earth Engine)}}}

\item {} 
\sphinxAtStartPar
\phantomsection\label{\detokenize{Qgis_Plugin/Calculate_medalus:id13}}{\hyperref[\detokenize{Qgis_Plugin/Calculate_medalus:id2}]{\sphinxcrossref{b. Using Custom data (Computed locally on device)}}}

\end{itemize}

\item {} 
\sphinxAtStartPar
\phantomsection\label{\detokenize{Qgis_Plugin/Calculate_medalus:id14}}{\hyperref[\detokenize{Qgis_Plugin/Calculate_medalus:management-quality-index-mqi}]{\sphinxcrossref{4. Management Quality Index (MQI)}}}

\item {} 
\sphinxAtStartPar
\phantomsection\label{\detokenize{Qgis_Plugin/Calculate_medalus:id15}}{\hyperref[\detokenize{Qgis_Plugin/Calculate_medalus:environmentally-sensitive-area-esa-index-combined-desertification-layer}]{\sphinxcrossref{Environmentally sensitive area (ESA) Index (Combined Desertification Layer)}}}

\end{itemize}

\end{itemize}
\end{sphinxShadowBox}


\section{1. Soil Quality Index (SQI)}
\label{\detokenize{Qgis_Plugin/Calculate_medalus:soil-quality-index-sqi}}
\sphinxAtStartPar
Soil is a crucial factor in evaluating the Environmental Sensitivity of an ecosystem, especially in the arid,
semi\sphinxhyphen{}arid and dry sub\sphinxhyphen{}humid zones. Soil properties related to desertification and degradation phenomena affect two principal parameters:
(i) water storage and retention capacity;
(ii) erosion resistance.

\sphinxAtStartPar
The formula used to compute the SQI is as shown below:

\sphinxAtStartPar
\sphinxstylestrong{SQI  = (Parent material x Depth x Texture x Slope x Drainage x Rock Fragments)\textasciicircum{}1/6}

\sphinxAtStartPar
Default datasets used for sqi are as shown below:


\begin{savenotes}\sphinxattablestart
\centering
\begin{tabulary}{\linewidth}[t]{|T|T|T|}
\hline
\sphinxstyletheadfamily 
\sphinxAtStartPar
\sphinxstylestrong{Indicator}
&\sphinxstyletheadfamily 
\sphinxAtStartPar
\sphinxstylestrong{Variables}
&\sphinxstyletheadfamily 
\sphinxAtStartPar
\sphinxstylestrong{Data Source}
\\
\hline
\sphinxAtStartPar
Soil Quality Index
&
\sphinxAtStartPar
Slope
&
\sphinxAtStartPar
\sphinxhref{https://developers.google.com/earth-engine/datasets/catalog/CGIAR\_SRTM90\_V4}{SRTM Digital Elevation}
\\
\hline&
\sphinxAtStartPar
Soil Depth
&
\sphinxAtStartPar
Custom User Input
\\
\hline&
\sphinxAtStartPar
Rock Fragments
&
\sphinxAtStartPar
\sphinxhref{https://webarchive.iiasa.ac.at/Research/LUC/External-World-soil-database/HTML/HWSD\_Data.html?sb=4}{Harmonized World Soil Database}
\\
\hline&
\sphinxAtStartPar
Parent Material
&
\sphinxAtStartPar
\sphinxhref{http://www.fao.org/geonetwork/srv/en/metadata.show\%3Fid=14116}{Digital Sol Map of the world}
\\
\hline&
\sphinxAtStartPar
Drainage
&
\sphinxAtStartPar
\sphinxhref{https://webarchive.iiasa.ac.at/Research/LUC/External-World-soil-database/HTML/HWSD\_Data.html?sb=4}{Harmonized World Soil Database}
\\
\hline&
\sphinxAtStartPar
Soil Texture
&
\sphinxAtStartPar
\sphinxhref{https://developers.google.com/earth-engine/datasets/catalog/OpenLandMap\_SOL\_SOL\_TEXTURE-CLASS\_USDA-TT\_M\_v02}{OpenLandMap Soil texture class (USDA system)}
\\
\hline
\end{tabulary}
\par
\sphinxattableend\end{savenotes}

\sphinxAtStartPar
Soil Quality Index can be calculated in two ways:
\begin{enumerate}
\sphinxsetlistlabels{\alph}{enumi}{enumii}{}{.}%
\item {} 
\sphinxAtStartPar
Using default data (Computed on Google Earth Engine)

\item {} 
\sphinxAtStartPar
Using Custom data (Computed locally on device)

\end{enumerate}


\subsection{a. Using default data (Computed on Google Earth Engine)}
\label{\detokenize{Qgis_Plugin/Calculate_medalus:a-using-default-data-computed-on-google-earth-engine}}
\sphinxAtStartPar
Inorder to compute Soil Quality Index using default data use the following steps.

\sphinxAtStartPar
First open the calculate indicators toolbox and select MEDALUS then select the Soil Quality Index
option as shown.

\noindent{\hspace*{\fill}\sphinxincludegraphics{{medalus_toolbox}.png}\hspace*{\fill}}

\noindent{\hspace*{\fill}\sphinxincludegraphics{{medalus_sqi}.png}\hspace*{\fill}}

\sphinxAtStartPar
Input soil depth (cm) and edit soil texture aggregation method according to case study as shown. This
will reclassify soil depth values according to the definition selected.

\noindent{\hspace*{\fill}\sphinxincludegraphics{{sqi_toolbox}.png}\hspace*{\fill}}

\noindent{\hspace*{\fill}\sphinxincludegraphics{{sqi_texture_agg}.png}\hspace*{\fill}}

\sphinxAtStartPar
Proceed to select an area of interest and run the computation.


\subsection{b. Using Custom data (Computed locally on device)}
\label{\detokenize{Qgis_Plugin/Calculate_medalus:b-using-custom-data-computed-locally-on-device}}
\sphinxAtStartPar
This step requires the data to be available locally.


\subsubsection{Raw Data Download}
\label{\detokenize{Qgis_Plugin/Calculate_medalus:raw-data-download}}
\sphinxAtStartPar
Inorder to prepare local data, \sphinxcode{\sphinxupquote{Harmonized World Soil Database}}


\subsubsection{Extracting Soil Drainage, Soil Texture and Rock Fragment Layers from HWSD DATA}
\label{\detokenize{Qgis_Plugin/Calculate_medalus:extracting-soil-drainage-soil-texture-and-rock-fragment-layers-from-hwsd-data}}
\sphinxAtStartPar
To extract the drainage, texture and soil group variables from the HWSD data follow these
simple steps:
\begin{enumerate}
\sphinxsetlistlabels{\arabic}{enumi}{enumii}{}{.}%
\item {} 
\sphinxAtStartPar
Open the HWSD data table on Microsoft Access. The variable to be extracted are D DRAINAGE(code),
D USDA TEX CLASS(code), and T GRAVEL. To initiate the extraction, create a query design in the ‘show table’
dialogue that pops up select HWSD DATA, D DRAINAGE and D USDA TEX CLASS tables.

\noindent{\hspace*{\fill}\sphinxincludegraphics{{sqi1}.png}\hspace*{\fill}}

\sphinxAtStartPar
In the ‘show table’ dialogue that pops up select HWSD DATA, D DRAINAGE and D USDA TEX CLASS tables

\noindent{\hspace*{\fill}\sphinxincludegraphics{{sqi2}.png}\hspace*{\fill}}

\item {} 
\sphinxAtStartPar
From the HWSD DATA select the MU GLOBAL and T GRAVEL variables. From the D DRAINAGE and D USDA TEX CLASS tables
select the CODE variable. The resulting query design should have the Fields row populated with the selected variables for
their respective table on the Table row

\noindent{\hspace*{\fill}\sphinxincludegraphics{{sqi3}.png}\hspace*{\fill}}

\item {} 
\sphinxAtStartPar
Click run and the resulting table should have 4 columns populated with data from the
selected variables

\noindent{\hspace*{\fill}\sphinxincludegraphics{{sqi4}.png}\hspace*{\fill}}

\noindent{\hspace*{\fill}\sphinxincludegraphics{{sqi5}.png}\hspace*{\fill}}

\item {} 
\sphinxAtStartPar
Export the table to MS Excel under the ’External Data’ tab as shown below. On the
dialogue that pops up, give the export table an appropriate name and select ’Excel 97 \sphinxhyphen{}
Excel 2003 Workbook’ as the export file format

\noindent{\hspace*{\fill}\sphinxincludegraphics{{sqi6}.png}\hspace*{\fill}}

\noindent{\hspace*{\fill}\sphinxincludegraphics{{sqi7}.png}\hspace*{\fill}}

\item {} 
\sphinxAtStartPar
In MS Excel, open the just exported sheet and in separate sheets copy the MU Global field
alongside the 3 variables of interest (Drainage, Gravel, Texture class); Paste the variables
separately as shown below to facilitate joining them to the raster dataset in
Q\sphinxhyphen{}GIS software.

\noindent{\hspace*{\fill}\sphinxincludegraphics{{sqi8}.png}\hspace*{\fill}}

\noindent{\hspace*{\fill}\sphinxincludegraphics{{sqi9}.png}\hspace*{\fill}}

\item {} 
\sphinxAtStartPar
For each of the newly created sheets, copy the tables in MS Excel and paste to MS Access

\noindent{\hspace*{\fill}\sphinxincludegraphics{{sqi10}.png}\hspace*{\fill}}

\noindent{\hspace*{\fill}\sphinxincludegraphics{{sqi11}.png}\hspace*{\fill}}

\item {} 
\sphinxAtStartPar
Under the Export data tab, select the Export to text file option as shown below.

\noindent{\hspace*{\fill}\sphinxincludegraphics{{sqi12}.png}\hspace*{\fill}}

\item {} 
\sphinxAtStartPar
On the Export text wizard that pops us set the export text as delimited text by checking
the Delimited options then click next.

\noindent{\hspace*{\fill}\sphinxincludegraphics{{sqi13}.png}\hspace*{\fill}}

\end{enumerate}

\sphinxAtStartPar
9. On the Export text wizard that pops us set the export text as delimited text by checking
the Delimited options fig. 2.15 then click next.
\begin{quote}

\noindent{\hspace*{\fill}\sphinxincludegraphics{{sqi14}.png}\hspace*{\fill}}

\sphinxAtStartPar
Click on next to finish the export. Repeat steps 6 to 9 for all the other remaining variables
and save the text files. The text files are used to assign pixel values to the HWSD.bil raster in Q\sphinxhyphen{}GIS.
\end{quote}
\begin{enumerate}
\sphinxsetlistlabels{\arabic}{enumi}{enumii}{}{.}%
\setcounter{enumi}{9}
\item {} 
\sphinxAtStartPar
Open the HWSD.bil raster file on Qgis together with the vector data for the OSS North Africa States in Q\sphinxhyphen{}GIS

\end{enumerate}
\begin{quote}

\noindent{\hspace*{\fill}\sphinxincludegraphics{{sqi15}.png}\hspace*{\fill}}
\end{quote}
\begin{enumerate}
\sphinxsetlistlabels{\arabic}{enumi}{enumii}{}{.}%
\setcounter{enumi}{10}
\item {} 
\sphinxAtStartPar
Under the raster tab on the Q\sphinxhyphen{}GIS menu bar navigate to ‘Extraction’ \textgreater{} ‘Extract by Mask Layer’

\end{enumerate}
\begin{quote}

\noindent{\hspace*{\fill}\sphinxincludegraphics{{sqi16}.png}\hspace*{\fill}}
\end{quote}
\begin{enumerate}
\sphinxsetlistlabels{\arabic}{enumi}{enumii}{}{.}%
\setcounter{enumi}{11}
\item {} 
\sphinxAtStartPar
In the Clip Raster by Mask Layer dialogue box, select hwsd raster as the input layer and

\end{enumerate}
\begin{quote}

\sphinxAtStartPar
the Vector layer as your Mask Layer. Save the output with the desired name to your desired location.

\noindent{\hspace*{\fill}\sphinxincludegraphics{{sqi17}.png}\hspace*{\fill}}

\sphinxAtStartPar
The clipped raster should is as shown below:

\noindent{\hspace*{\fill}\sphinxincludegraphics{{sqi18}.png}\hspace*{\fill}}
\end{quote}
\begin{enumerate}
\sphinxsetlistlabels{\arabic}{enumi}{enumii}{}{.}%
\setcounter{enumi}{12}
\item {} 
\sphinxAtStartPar
On the Q\sphinxhyphen{}GIS Menu bar, click on Processing and select the toolbox (or use keyboard shortcut Ctrl+Alt+T) to

\end{enumerate}
\begin{quote}

\sphinxAtStartPar
open the processing toolbox.

\noindent{\hspace*{\fill}\sphinxincludegraphics{{sqi19}.png}\hspace*{\fill}}
\end{quote}
\begin{enumerate}
\sphinxsetlistlabels{\arabic}{enumi}{enumii}{}{.}%
\setcounter{enumi}{13}
\item {} 
\sphinxAtStartPar
On the processing toolbox, search for the GRASS\sphinxhyphen{}GIS r.recode tool

\end{enumerate}
\begin{quote}

\noindent{\hspace*{\fill}\sphinxincludegraphics{{sqi20}.png}\hspace*{\fill}}
\end{quote}
\begin{enumerate}
\sphinxsetlistlabels{\arabic}{enumi}{enumii}{}{.}%
\setcounter{enumi}{14}
\item {} 
\sphinxAtStartPar
On the r.recode dialogue, select the clipped HWSD data as the input layer and the .txt

\end{enumerate}
\begin{quote}

\sphinxAtStartPar
file previously prepared as the file containing the recode rules(for this example we will use the rock fragment).
Save the output to your desired location and click run.

\noindent{\hspace*{\fill}\sphinxincludegraphics{{sqi21}.png}\hspace*{\fill}}

\sphinxAtStartPar
Using the HWSD cliped raster as input layer, repeat the step 14 and 15 with the appropriate recoding rules to extract
the remaining variables and save the outputs for the other datasets.

\noindent{\hspace*{\fill}\sphinxincludegraphics{{sqi22}.png}\hspace*{\fill}}
\end{quote}


\subsubsection{Compute Soil Quality Index}
\label{\detokenize{Qgis_Plugin/Calculate_medalus:compute-soil-quality-index}}
\sphinxAtStartPar
First Input soil depth (cm) then Select Custom soil
quality datasets instead of default and import sqi datasets as below.

\noindent{\hspace*{\fill}\sphinxincludegraphics{{sqi_custom}.png}\hspace*{\fill}}

\sphinxAtStartPar
Select import and select a raster or vector dataset of interest. Select the band number for the
raster dataset. Input the aggregation definition, study year and the reclassified output destination file as shown:

\noindent{\hspace*{\fill}\sphinxincludegraphics{{sqi_load_raster}.png}\hspace*{\fill}}

\sphinxAtStartPar
Ensure to reclassifiy values correctly according to case study. Once all 4 datasets are imported proceed
to select an area of interest and run the computation. You will be required to select a destination for your
output file.

\noindent{\hspace*{\fill}\sphinxincludegraphics{{sqi_aggregation}.png}\hspace*{\fill}}


\section{2. Vegetation Quality Index (VQI)}
\label{\detokenize{Qgis_Plugin/Calculate_medalus:vegetation-quality-index-vqi}}
\sphinxAtStartPar
The Vegetation Quality index is derived as the geometric mean of the characteristics of the vegetation. Fire Hazard layers (RI),
Fire Resistance (FR), drought (RS), vegetation erosion protection (PE) and cover plant (CV) according to the
following formula:

\sphinxAtStartPar
\sphinxstylestrong{VQI = (RI x PE x RS x CV) \textasciicircum{} ¼}

\sphinxAtStartPar
Default datasets used for vqi are as shown below:


\begin{savenotes}\sphinxattablestart
\centering
\begin{tabulary}{\linewidth}[t]{|T|T|T|}
\hline
\sphinxstyletheadfamily 
\sphinxAtStartPar
\sphinxstylestrong{Indicator}
&\sphinxstyletheadfamily 
\sphinxAtStartPar
\sphinxstylestrong{Variables}
&\sphinxstyletheadfamily 
\sphinxAtStartPar
\sphinxstylestrong{Data Source}
\\
\hline
\sphinxAtStartPar
Vegetation Quality Index
&
\sphinxAtStartPar
Fire Risk
&
\sphinxAtStartPar
\sphinxhref{http://maps.elie.ucl.ac.be/CCI/viewer/}{ESA CCI\textendash{}land cover map v2.0.7\textendash{}2015}
\\
\hline&
\sphinxAtStartPar
Drought Resistance
&
\sphinxAtStartPar
\sphinxhref{http://maps.elie.ucl.ac.be/CCI/viewer/}{ESA CCI\textendash{}land cover map v2.0.7\textendash{}2015}
\\
\hline&
\sphinxAtStartPar
Erosion Protection
&
\sphinxAtStartPar
\sphinxhref{http://maps.elie.ucl.ac.be/CCI/viewer/}{ESA CCI\textendash{}land cover map v2.0.7\textendash{}2015}
\\
\hline&
\sphinxAtStartPar
Plant Cover
&
\sphinxAtStartPar
\sphinxhref{https://developers.google.com/earth-engine/datasets/catalog/VITO\_PROBAV\_C1\_S1\_TOC\_100M}{PROBA\sphinxhyphen{}V C1 Top Of Canopy Daily Synthesis 100m}
\\
\hline
\end{tabulary}
\par
\sphinxattableend\end{savenotes}

\sphinxAtStartPar
To compute vegetation quality index, select Vegetation Quality Index option under the MEDALUS toolbar. For each of the 3 datasets, i.e
\sphinxhyphen{} Fire Risk
\sphinxhyphen{} Drought Resistance
\sphinxhyphen{} Erosion Protection

\noindent{\hspace*{\fill}\sphinxincludegraphics{{vqi_toolbox}.png}\hspace*{\fill}}

\sphinxAtStartPar
Select the land cover year and an aggregation definition or use default set aggregation. This will reclassify
land cover classes based on definition provided.

\noindent{\hspace*{\fill}\sphinxincludegraphics{{vqi_agg}.png}\hspace*{\fill}}

\sphinxAtStartPar
For Plant cover, select a start and end date. Proceed to select an area of interest and run the computation.


\section{3. Climate Quality Index (CQI)}
\label{\detokenize{Qgis_Plugin/Calculate_medalus:climate-quality-index-cqi}}
\sphinxAtStartPar
Climate quality is assessed on the basis of how it influences water availability to the plants. The climate quality index, according to the MEDALUS approach, is obtained by cross\sphinxhyphen{}referencing the three layers of
information namely precipitation and aridity index using the following equation:

\sphinxAtStartPar
\sphinxstylestrong{CQI = (precipitation x aridity index) \textasciicircum{} 1/2}

\sphinxAtStartPar
Default datasets used for CQI are:


\begin{savenotes}\sphinxattablestart
\centering
\begin{tabulary}{\linewidth}[t]{|T|T|T|}
\hline
\sphinxstyletheadfamily 
\sphinxAtStartPar
\sphinxstylestrong{Indicator}
&\sphinxstyletheadfamily 
\sphinxAtStartPar
\sphinxstylestrong{Variables}
&\sphinxstyletheadfamily 
\sphinxAtStartPar
\sphinxstylestrong{Data Source}
\\
\hline
\sphinxAtStartPar
Climate Quality Index
&
\sphinxAtStartPar
Precipitation
&
\sphinxAtStartPar
\sphinxhref{developers.google.com/earth-engine/datasets/catalog/IDAHO\_EPSCOR\_TERRACLIMATE}{TerraClimate Monthly Climate and Climatic Water Balance for Global Terrestrial Surfaces}
\\
\hline&
\sphinxAtStartPar
Potential Evapotranspiration
&
\sphinxAtStartPar
\sphinxhref{developers.google.com/earth-engine/datasets/catalog/IDAHO\_EPSCOR\_TERRACLIMATE}{TerraClimate Monthly Climate and Climatic Water Balance for Global Terrestrial Surfaces}
\\
\hline
\end{tabulary}
\par
\sphinxattableend\end{savenotes}

\sphinxAtStartPar
Climate Quality Index can be calculated in two ways:
\begin{enumerate}
\sphinxsetlistlabels{\alph}{enumi}{enumii}{}{.}%
\item {} 
\sphinxAtStartPar
Using default data (Computed on Google Earth Engine)

\item {} 
\sphinxAtStartPar
Using Custom data (Computed locally on device)

\end{enumerate}


\subsection{a. Using default data (Computed on Google Earth Engine)}
\label{\detokenize{Qgis_Plugin/Calculate_medalus:id1}}
\sphinxAtStartPar
Select a year of study between 1979\sphinxhyphen{}2020. Proceed to select and area of interest and run the computation.

\noindent{\hspace*{\fill}\sphinxincludegraphics{{cqi_toolbox}.png}\hspace*{\fill}}


\subsection{b. Using Custom data (Computed locally on device)}
\label{\detokenize{Qgis_Plugin/Calculate_medalus:id2}}
\sphinxAtStartPar
This step requires the data to be available locally. Load both a potential evapotranspiration and precipitation dataset to the
plugin as shown below.

\noindent{\hspace*{\fill}\sphinxincludegraphics{{cqi_toolbox_custom}.png}\hspace*{\fill}}

\sphinxAtStartPar
Proceed to select an area of interest and run the computation. You will be required to select a destination for your
output file.


\section{4. Management Quality Index (MQI)}
\label{\detokenize{Qgis_Plugin/Calculate_medalus:management-quality-index-mqi}}
\sphinxAtStartPar
The Management quality index, according to the MEDALUS approach, is obtained by cross\sphinxhyphen{}referencing the two layers of information
namely Land\sphinxhyphen{}Use intensity (LU) and Population Density (PD) using the following equation:

\sphinxAtStartPar
\sphinxstylestrong{MQI = (LU X PD)\textasciicircum{}1/2}

\sphinxAtStartPar
Default datasets used for MQI are:


\begin{savenotes}\sphinxattablestart
\centering
\begin{tabulary}{\linewidth}[t]{|T|T|T|}
\hline
\sphinxstyletheadfamily 
\sphinxAtStartPar
\sphinxstylestrong{Indicator}
&\sphinxstyletheadfamily 
\sphinxAtStartPar
\sphinxstylestrong{Variables}
&\sphinxstyletheadfamily 
\sphinxAtStartPar
\sphinxstylestrong{Data Source}
\\
\hline
\sphinxAtStartPar
Management Quality Index
&
\sphinxAtStartPar
Land Use Intensity
&
\sphinxAtStartPar
\sphinxhref{http://maps.elie.ucl.ac.be/CCI/viewer/}{ESA CCI\textendash{}land cover map v2.0.7\textendash{}2015}
\\
\hline&
\sphinxAtStartPar
Population Density
&
\sphinxAtStartPar
\sphinxhref{https://developers.google.com/earth-engine/datasets/catalog/CIESIN\_GPWv411\_GPW\_Population\_Density}{GPWv411: Population Density (Gridded Population of the World Version 4.11)}
\\
\hline
\end{tabulary}
\par
\sphinxattableend\end{savenotes}

\sphinxAtStartPar
To compute vegetation quality index, select Management Quality Index option under the MEDALUS toolbar. Select
the land cover year an and set aggregation definition for Land Use Intensity. This will reclassify land cover classes
based on definition provided.

\noindent{\hspace*{\fill}\sphinxincludegraphics{{mqi_toolbox}.png}\hspace*{\fill}}

\noindent{\hspace*{\fill}\sphinxincludegraphics{{mqi_agg}.png}\hspace*{\fill}}

\sphinxAtStartPar
Proceed to select an area of interest and run the computation. You will be required to select a destination for your
output file.


\section{Environmentally sensitive area (ESA) Index (Combined Desertification Layer)}
\label{\detokenize{Qgis_Plugin/Calculate_medalus:environmentally-sensitive-area-esa-index-combined-desertification-layer}}
\sphinxAtStartPar
The environmentally sensitive area (ESA) index (ESAI) is computed according to the original procedure as a geometric
mean of the four quality values recorded at each location (i.e., in each elementary pixel; Equation 2):

\sphinxAtStartPar
\sphinxstylestrong{ESAI = (SQI x VQI x CQI x MQI) \textasciicircum{} 1/4}

\sphinxAtStartPar
To compute the final desertification layer all MEDALUS subindicator must be already computed i.e SQI, VQI, CQI, MQI.
In the MEDALUS toolbox select Calculate \sphinxstylestrong{final MEDALUS option} as shown below:

\noindent{\hspace*{\fill}\sphinxincludegraphics{{esai}.png}\hspace*{\fill}}

\sphinxAtStartPar
Load all 4 layers to the plugin, select the area of interest and compute the ESAI. You will be required to
select a destination for your output file.

\begin{sphinxadmonition}{note}{Note:}
\sphinxAtStartPar
All layers will be automatically loaded into the plugin if they are available and loaded within QGIS.
\end{sphinxadmonition}

\noindent{\hspace*{\fill}\sphinxincludegraphics{{esai_toolbox}.png}\hspace*{\fill}}

\sphinxstepscope


\chapter{View and download results}
\label{\detokenize{Qgis_Plugin/gee_tasks:view-and-download-results}}\label{\detokenize{Qgis_Plugin/gee_tasks::doc}}


\noindent{\hspace*{\fill}\sphinxincludegraphics{{plugin_toolbar_tasks}.png}\hspace*{\fill}}

\sphinxAtStartPar
Once you have submitted a calculation using MISLAND, it is sent to
Google Earth Engine to run the calculations in the cloud. To view the Google
Earth Engine (GEE) tasks you have running, and to download your results, select
the cloud with the arrow facing down icon highlighted above. This will open up the \sphinxtitleref{Download results
from Earth Engine} dialog box.

\sphinxAtStartPar
Click \sphinxtitleref{Refresh List} to show all the tasks you have submitted and their status.

\sphinxAtStartPar
Users can view their current and previous tasks here. the table shows the
task name provided by the user, which analysis is running (job), the start time
and end time of when the task was started and completed and whether or not the
task was successful. The Details page outlines the different options the user
chose for each task.

\noindent{\hspace*{\fill}\sphinxincludegraphics{{download_results}.png}\hspace*{\fill}}

\sphinxAtStartPar
To download results to the computer once a task has finished, click on the task
you are interested in downloading results for, then click \sphinxtitleref{Download results}.
In the window that appears, choose the location in which to save the download.
Note that some downloads might consist of multiple files \sphinxhyphen{} all of these files
need to be kept together if the results are moved to a different location on
the computer (or saved to a USB stick).

\noindent{\hspace*{\fill}\sphinxincludegraphics{{image0361}.png}\hspace*{\fill}}

\sphinxstepscope


\chapter{Load Custom Data}
\label{\detokenize{Qgis_Plugin/load_data:load-custom-data}}\label{\detokenize{Qgis_Plugin/load_data::doc}}


\noindent{\hspace*{\fill}\sphinxincludegraphics{{plugin_toolbar_loaddata}.png}\hspace*{\fill}}

\sphinxAtStartPar
The “Load data” function allows the user to load data into QGIS and
MISLAND for analysis.

\sphinxAtStartPar
There are two options, to load results of MISLAND analysis or to load
custom datasets which will be used to compute the indicators.

\noindent{\hspace*{\fill}\sphinxincludegraphics{{custom_load}.png}\hspace*{\fill}}


\section{Load a dataset produced by MISLAND}
\label{\detokenize{Qgis_Plugin/load_data:load-a-dataset-produced-by-misland}}
\sphinxAtStartPar
This option lets you load already downloaded results from MISLAND. eg Productivity, Land Cover, Soil Organic Carbon,
Forest Fires, Soil Quality Index, Vegetation Quality Index, Climate Quality Index, Management Quality Index, etc.


\subsection{Productivity}
\label{\detokenize{Qgis_Plugin/load_data:productivity}}
\sphinxAtStartPar
Use this function to load into the QGIS map pre\sphinxhyphen{}computed productivity indicators which had been
processed to identify land degradation.

\sphinxAtStartPar
Navigate to the folder where you stored the downloaded file and select it. The downloaded file is an 8 band raster,
with each band representing the three subindicators (trajectory, performance and state) plus other information which may
help you interpret the trends identified. The layers to be loaded into the QGIS maps are the ones highlighted in gray. By default: trajectory
(degradation and slope), performance and state. If you want to load the other layers, simply select them and click OK.

\noindent{\hspace*{\fill}\sphinxincludegraphics{{loaddata_trendsearth_productivity}.png}\hspace*{\fill}}

\sphinxAtStartPar
The layers will be loaded in the QGIS map with its corresponding symbology.

\noindent{\hspace*{\fill}\sphinxincludegraphics{{loaddata_trendsearth_productivity_loaded}.png}\hspace*{\fill}}


\subsection{Land cover}
\label{\detokenize{Qgis_Plugin/load_data:land-cover}}
\sphinxAtStartPar
This option lets you load pre\sphinxhyphen{}computed land cover indicators which had been
processed to identify land degradation.

\sphinxAtStartPar
Navigate to the folder where you stored the downloaded file and select it. The downloaded file is a multi band raster. The number
of bands will depend on the period of analysis selected and the data source used. If the default ESA CCI land cover was used, for example,
annual land cover maps will be downloaded.

\sphinxAtStartPar
The bands in the stack represent: initial and final land cover (annual if available) both in the original classification scheme and using
the UNCCD 7 class land cover table, land cover transitions and land cover degradation as
identified by this subindicator. If you want to load the other layers not selected by default, simply select them and click OK.

\noindent{\hspace*{\fill}\sphinxincludegraphics{{loaddata_trendsearth_landcover}.png}\hspace*{\fill}}

\sphinxAtStartPar
The layers will be loaded in the QGIS map with its corresponding symbology.

\noindent{\hspace*{\fill}\sphinxincludegraphics{{loaddata_trendsearth_landcover_loaded}.png}\hspace*{\fill}}


\subsection{Soil organic carbon}
\label{\detokenize{Qgis_Plugin/load_data:soil-organic-carbon}}
\sphinxAtStartPar
This option lets you load pre\sphinxhyphen{}computed soil organic carbon indicators which had been
processed to identify land degradation.

\sphinxAtStartPar
Navigate to the folder where you stored the downloaded file and select it. The downloaded file is a multi band raster. The number
of bands will depend on the period of analysis selected and the data source used. If the default ESA CCI land cover was used, for example,
annual soil organic carbon maps will be downloaded.

\sphinxAtStartPar
The bands in the stack represent: initial and final soil organic carbon stocks (annual soc if annual land data is available),
initial and final land cover maps using the UNCCD 7 class land cover classification,and degradation as identified by this
subindicator. The units of the degradation layer are “\% change”, if changes are larger than 10\% for the period, they will be
considered as improvement or degradation depending on the sign of the change. If you want to load the other layers not
selected by default, simply select them and click OK.

\noindent{\hspace*{\fill}\sphinxincludegraphics{{loaddata_trendsearth_soc}.png}\hspace*{\fill}}

\sphinxAtStartPar
The layers will be loaded in the QGIS map with its corresponding symbology.

\noindent{\hspace*{\fill}\sphinxincludegraphics{{loaddata_trendsearth_soc_loaded}.png}\hspace*{\fill}}


\subsection{SDG 15.3.1 indicator}
\label{\detokenize{Qgis_Plugin/load_data:sdg-15-3-1-indicator}}
\sphinxAtStartPar
This option lets you load pre\sphinxhyphen{}computed productivity subindicator (the integration of trajectory,
performance and state) as well as the final SDG 15.3.1 (the integration of productivity,
land cover and soil organic carbon)

\sphinxAtStartPar
Navigate to the folder where you stored the downloaded file and select it. The downloaded file is an 2 band raster, the first one
containing information on the SDG and the second on the 5 classes productivity subindicator.

\noindent{\hspace*{\fill}\sphinxincludegraphics{{loaddata_trendsearth_sdg}.png}\hspace*{\fill}}

\sphinxAtStartPar
The layers will be loaded in the QGIS map with its corresponding symbology.

\noindent{\hspace*{\fill}\sphinxincludegraphics{{loaddata_trendsearth_sdg_loaded}.png}\hspace*{\fill}}


\section{Load a custom input dataset}
\label{\detokenize{Qgis_Plugin/load_data:load-a-custom-input-dataset}}

\subsection{Productivity}
\label{\detokenize{Qgis_Plugin/load_data:id1}}
\sphinxAtStartPar
Use this option to load productivity datasets which have already been generated outside of MISLAND.

\sphinxAtStartPar
Productivity classes in the input data must be coded as follows:

\sphinxAtStartPar
1: Declining
2: Early signs of decline
3: Stable but stressed
4: Stable
5: Increasing
0 or \sphinxhyphen{}32768: No data

\noindent{\hspace*{\fill}\sphinxincludegraphics{{loaddata_landproductivity}.png}\hspace*{\fill}}


\subsection{Land cover}
\label{\detokenize{Qgis_Plugin/load_data:id2}}
\sphinxAtStartPar
Use this option to load land cover datasets which will then be used for land
cover change analysis and/or soil organic carbon change analysis.

\noindent{\hspace*{\fill}\sphinxincludegraphics{{loaddata_landcover}.png}\hspace*{\fill}}

\begin{sphinxadmonition}{note}{Note:}
\sphinxAtStartPar
If you’ll be using the \sphinxhref{https://www.eea.europa.eu/publications/COR0-landcover}{CORINE land cover data} , you can use
\sphinxhref{https://s3.amazonaws.com/trends.earth/sharing/Corine\_Land\_Cover\_to\_UNCCD\_TrendsEarth\_Definition.json}{this definition file}
to pre\sphinxhyphen{}load a suggested aggregation of the land cover classes in Corine to convert them to the 7 UNCCD land cover classes.
\end{sphinxadmonition}


\subsection{Soil organic carbon}
\label{\detokenize{Qgis_Plugin/load_data:id3}}
\sphinxAtStartPar
Processing of custom soil organic carbon data can be handled using this
tool.

\begin{sphinxadmonition}{note}{Note:}
\sphinxAtStartPar
This tool assumes that the units of the raster layer to be imported
are \sphinxstylestrong{Metrics Tons of organic carbon per hectare}. If your layer is in
different units, please make the necessary conversions before using it in
MISLAND.
\end{sphinxadmonition}

\sphinxstepscope


\chapter{Download raw data}
\label{\detokenize{Qgis_Plugin/Download_data:download-raw-data}}\label{\detokenize{Qgis_Plugin/Download_data::doc}}


\noindent{\hspace*{\fill}\sphinxincludegraphics{{plugin_toolbar_download}.png}\hspace*{\fill}}

\sphinxAtStartPar
To download data, select the download icon highlighted above. This will open up the
“Download Data” dialog box:

\sphinxAtStartPar
If you would like to work with the original raw data used in LDMS,
you can select select an area of interest and download the desired data for
further analysis.

\noindent{\hspace*{\fill}\sphinxincludegraphics{{download_data}.png}\hspace*{\fill}}

\sphinxAtStartPar
The table below describes all the data available through the toolbox.
It specifies data sources, resolutions, coverage and the different indicators
for which each data set is used.

\sphinxstepscope


\chapter{Visualization tools}
\label{\detokenize{Qgis_Plugin/visualization:visualization-tools}}\label{\detokenize{Qgis_Plugin/visualization::doc}}


\noindent{\hspace*{\fill}\sphinxincludegraphics{{plugin_toolbar_visualize}.png}\hspace*{\fill}}

\sphinxAtStartPar
To view the “Visualization Tools”, select the report icon highlighted above. This will open up
the “Visualization tools” dialog box:

\noindent{\hspace*{\fill}\sphinxincludegraphics{{image065}.png}\hspace*{\fill}}

\sphinxAtStartPar
Here there are two options: “Add Basemap” or “Create Print Map”. The first allows users to add a Basemap for a first or second level administrative boundary. The second option brings up the Composer window in QGIS to design and export a static map.

\noindent{\hspace*{\fill}\sphinxincludegraphics{{image066}.png}\hspace*{\fill}}

\sphinxAtStartPar
By selecting “Add Basemap”, the user can select the first or second level administrative boundary. The first level is the country boundary. The second level will be the first sub\sphinxhyphen{}division that the country is divided into and will be dependent on the country selected. For example, in the United States of America, the second level will provide a drop down of states. In Kenya, the second level will display provinces.
Please note the disclaimer in the window. Natural Earth provides the spatial layers contained within the dropdown. These boundaries are not official endorsed by CI or other partner organizations and contributors.
After selecting the dropdown, for the first level and second level if applicable, select “Ok”.

\sphinxAtStartPar
After submitting the above message will appear within the QGIS Desktop window. This shows that the Basemap is loading. DO NOT select cancel or attempt another function in QGIS until the Basemap has loaded. The time it takes to load will depend on your Internet connection and computer processor.

\sphinxAtStartPar
If you have a map layer within your QGIS Desktop window, you will now see the Basemap with the administrative level selected clipped out to view the underlying map layer.

\sphinxstepscope


\chapter{Calculate SDG 15.3.1}
\label{\detokenize{Qgis_Plugin/sdg_15_training:calculate-sdg-15-3-1}}\label{\detokenize{Qgis_Plugin/sdg_15_training::doc}}
\noindent{\hspace*{\fill}\sphinxincludegraphics{{plugin_toolbar_calculate}.png}\hspace*{\fill}}

\sphinxAtStartPar
Sustainable Development Goal 15.3 intends to combat desertification, restore
degraded land and soil, including land affected by desertification, drought and
floods, and strive to achieve a land degradation\sphinxhyphen{}neutral world by 2030. In
order to assess the progress to this goal, the agreed\sphinxhyphen{}upon indicator for SDG
15.3 (proportion of land area degraded) is a combination of three
sub\sphinxhyphen{}indicators: change in land productivity, change in land cover and change
in soil organic carbon.

\sphinxAtStartPar
Watch the video below for an introduction to SDG 15.3.1 monitoring using MISLAND
QGIS Plugin.



\sphinxstepscope


\chapter{Calculate Vegetation/Forest Degradation}
\label{\detokenize{Qgis_Plugin/vegetation_degradation_training:calculate-vegetation-forest-degradation}}\label{\detokenize{Qgis_Plugin/vegetation_degradation_training::doc}}
\begin{sphinxShadowBox}
\sphinxstyletopictitle{Contents}
\begin{itemize}
\item {} 
\sphinxAtStartPar
\phantomsection\label{\detokenize{Qgis_Plugin/vegetation_degradation_training:id1}}{\hyperref[\detokenize{Qgis_Plugin/vegetation_degradation_training:calculate-vegetation-forest-degradation}]{\sphinxcrossref{Calculate Vegetation/Forest Degradation}}}
\begin{itemize}
\item {} 
\sphinxAtStartPar
\phantomsection\label{\detokenize{Qgis_Plugin/vegetation_degradation_training:id2}}{\hyperref[\detokenize{Qgis_Plugin/vegetation_degradation_training:compute-vegatation-indices}]{\sphinxcrossref{Compute Vegatation Indices}}}

\item {} 
\sphinxAtStartPar
\phantomsection\label{\detokenize{Qgis_Plugin/vegetation_degradation_training:id3}}{\hyperref[\detokenize{Qgis_Plugin/vegetation_degradation_training:compute-forest-fires}]{\sphinxcrossref{Compute Forest Fires}}}

\item {} 
\sphinxAtStartPar
\phantomsection\label{\detokenize{Qgis_Plugin/vegetation_degradation_training:id4}}{\hyperref[\detokenize{Qgis_Plugin/vegetation_degradation_training:compute-forest-change-and-total-carbon-summary}]{\sphinxcrossref{Compute Forest Change and Total Carbon \& Summary}}}

\end{itemize}

\end{itemize}
\end{sphinxShadowBox}

\sphinxAtStartPar
Watch the video below for an introduction to Vegetation/ Forest degradation
monitoring using MISLAND QGIS Plugin.




\section{Compute Vegatation Indices}
\label{\detokenize{Qgis_Plugin/vegetation_degradation_training:compute-vegatation-indices}}
\sphinxAtStartPar
Land degradation hotspots (LDH) are produced via the analysis of time\sphinxhyphen{}series
vegetation indices data and are used to characterize areas of different sizes,
where the vegetation cover and the soil types are severely degraded. Vegetation
loss/gain hotspots will be calculated based on time series observation of selected
suit of vegetation indices depending on the climatic zones and terrain morphology
of the North African countries

\sphinxAtStartPar
Vegation Indices computed from Landsat 7 ETM+ include:
\begin{enumerate}
\sphinxsetlistlabels{\arabic}{enumi}{enumii}{}{.}%
\item {} 
\sphinxAtStartPar
\sphinxstylestrong{NDVI (humid, sub\sphinxhyphen{}humid and semi\sphinxhyphen{}arid zones)}

\sphinxAtStartPar
DVI is preferable for global vegetation monitoring since it helps to compensate for
changes in lighting conditions, surface slope, exposure, and other external factors.
NDVI is calculated in accordance with the formula:

\noindent{\hspace*{\fill}\sphinxincludegraphics{{ndvi}.png}\hspace*{\fill}}

\sphinxAtStartPar
NIR \textendash{} reflection in the near\sphinxhyphen{}infrared spectrum
RED \textendash{} reflection in the red range of the spectrum

\sphinxAtStartPar
According to this formula, the density of vegetation (NDVI) at a certain point of the
image is equal to the difference in the intensities of reflected light in the red and
infrared range divided by the sum of these intensities.

\sphinxAtStartPar
This index defines values ​​from \sphinxhyphen{}1.0 to 1.0, basically representing greens, where negative
values ​​are mainly formed from clouds, water and snow, and values ​​close to zero are
primarily formed from rocks and bare soil. Very small values ​​(0.1 or less) of the NDVI
function correspond to empty areas of rocks, sand or snow. Moderate values ​​(from 0.2 to 0.3)
represent shrubs and meadows, while large values ​​(from 0.6 to 0.8) indicate temperate and
tropical forests.

\item {} 
\sphinxAtStartPar
\sphinxstylestrong{MSAVI2 (arid and stepic zones)}

\sphinxAtStartPar
MSAVI2 is soil adjusted vegetation indices that seek to address some of the limitation of
NDVI when applied to areas with a high degree of exposed soil surface.It eliminates the need
to find the soil line from a feature\sphinxhyphen{}space plot or even explicitly specify the soil brightness
correction factor:

\noindent{\hspace*{\fill}\sphinxincludegraphics{{msavi2}.png}\hspace*{\fill}}

\item {} 
\sphinxAtStartPar
\sphinxstylestrong{SAVI (desert areas)}

\sphinxAtStartPar
SAVI is used to correct Normalized Difference Vegetation Index (NDVI) for the influence of
soil brightness in areas where vegetative cover is low. Landsat Surface Reflectance\sphinxhyphen{}derived
SAVI is calculated as a ratio between the R and NIR values with a soil brightness correction
factor (L) defined as 0.5 to accommodate most land cover types.

\noindent{\hspace*{\fill}\sphinxincludegraphics{{savi}.png}\hspace*{\fill}}

\end{enumerate}


\section{Compute Forest Fires}
\label{\detokenize{Qgis_Plugin/vegetation_degradation_training:compute-forest-fires}}
\sphinxAtStartPar
Burnt areas and forest fires are be highlighted and mapped out form remotely sensed \sphinxstylestrong{Landsat 8 /Sentinel 2}
data using the Normalized Burn Ratio (NBR). NBR is designed to highlight burned areas and estimate burn
severity. It uses near\sphinxhyphen{}infrared (NIR) and shortwave\sphinxhyphen{}infrared (SWIR) wavelengths. Before fire events,
healthy vegetation has very high NIR reflectance and a low SWIR reflectance. In contrast, recently
burned areas show low reflectance in the NIR and high reflectance in the SWIR band.

\sphinxAtStartPar
The NBR is be calculated for Landsat/Sentinel images before the fire (pre\sphinxhyphen{}fire NBR) and after
the fire (post\sphinxhyphen{}fire NBR). The \sphinxstylestrong{difference between the pre\sphinxhyphen{}fire NBR and the post\sphinxhyphen{}fire NBR} referred
to as \sphinxstylestrong{delta NBR (dNBR)} is computed to highlight the areas of forest disturbance by fire event.

\sphinxAtStartPar
Classification of the dNBR is be used for burn severity assessment, as areas with higher dNBR
values indicate more severe damage whereas areas with negative dNBR values might show increased
vegetation productivity. dNBR is classified according to burn severity ranges proposed by
the United States Geological Survey(USGS)


\section{Compute Forest Change and Total Carbon \& Summary}
\label{\detokenize{Qgis_Plugin/vegetation_degradation_training:compute-forest-change-and-total-carbon-summary}}
\sphinxAtStartPar
The quantification of the forest gain/loss hotspots will be based on pre\sphinxhyphen{}existing high\sphinxhyphen{}resolution
global maps derived from Hansen Global Forest change dataset that can be accessed using Google
Earth Engine API. The maps are produced from time\sphinxhyphen{}series analysis of Landsat images characterizing
forest extent and change over time.


\chapter{Indices and tables}
\label{\detokenize{index:indices-and-tables}}\begin{itemize}
\item {} 
\sphinxAtStartPar
\DUrole{xref,std,std-ref}{genindex}

\item {} 
\sphinxAtStartPar
\DUrole{xref,std,std-ref}{modindex}

\item {} 
\sphinxAtStartPar
\DUrole{xref,std,std-ref}{search}

\end{itemize}



\renewcommand{\indexname}{Index}
\printindex
\end{document}